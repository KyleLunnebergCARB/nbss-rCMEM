\hypertarget{overview}{%
\section{Overview}\label{overview}}

The following vignette walks through the implementation and optionality
of the Cohort Marsh Equilibrium Model R package (version 6.1.4). This
vignette complements the numerical modeling section of the main text of
the manuscript. In each section, we highlight the arguments necessary
for the wrapper function
\protect\hyperlink{runcohortmem}{\texttt{runCohortMem()}} that produces
a simulations of the Cohort Marsh Equilibrium Model and provide
background to help the user inform those arguments. At the end of the
vignette, we also introduce code that can be used to produce graphics
that show partitioning of mass pools across soil cohorts through time.

\hypertarget{predicting-marsh-accretion-and-carbon-accumulation-with-runcohortmem}{%
\section{\texorpdfstring{Predicting marsh accretion and carbon
accumulation with
\protect\hyperlink{runcohortmem}{\texttt{runCohortMem()}}}{Predicting marsh accretion and carbon accumulation with runCohortMem()}}\label{predicting-marsh-accretion-and-carbon-accumulation-with-runcohortmem}}

\hypertarget{about-runcohortmem}{%
\subsection{\texorpdfstring{About
\protect\hyperlink{runcohortmem}{\texttt{runCohortMem()}}}{About runCohortMem()}}\label{about-runcohortmem}}

Overall the function
\protect\hyperlink{runcohortmem}{\texttt{runCohortMem()}} is a wrapper
function that organizes inputs and parameters of CMEM and runs a series
of functions to execute a single simulation (\emph{i.e.} a prediction of
marsh surface elevation and carbon accumulation across a specified time
period). First,
\protect\hyperlink{runcohortmem}{\texttt{runCohortMem()}} will build a
scenario that is characterized by an annualized time series of
sea-level, tidal height, and suspended sediment concentration. Second,
the function will initialize the scenario, building a sediment profile
to serve as the sediment column at time step t = 1. Third, the function
will at each year of the simulation and will lay down a mineral cohort
based on elevation and flooding depth, the suspended sediment
concentration, and flooding time. Simultaneously the function calculates
biomass based on relative tidal elevation, adds roots to each cohort in
the sediment profile, and ages the organic matter cohorts.

For each year the function calculates and tracks net elevation change of
the profile making it a dynamic model. The function has two main output
structures: (1) a table that has reports of inputs and outputs for each
annual time step and (2) a table which tracks the mass of the four mass
pools (\emph{i.e.} mineral, fast organic, slow organic, roots) for each
cohort and each year of the simulation. Finally, output table of each
cohort also reports the depth of the top and bottom of each cohort
relative to the surface and the cumulative volume of the cohorts by
depth.

The \protect\hyperlink{runcohortmem}{\texttt{runCohortMem()}} function
can be used to create both forecasts and hindcasts depending on the
amount and type of data that are use to inform the simulation as we
highlight below. For example, a user can input a historical tide gauge
record or use information from a core sampled from marsh soil to create
a hindcast of marsh accretion and carbon accumulation.

\hypertarget{building-a-tidal-scenario}{%
\subsection{Building a tidal scenario}\label{building-a-tidal-scenario}}

\begin{longtable}[]{@{}
  >{\raggedright\arraybackslash}p{(\columnwidth - 6\tabcolsep) * \real{0.2759}}
  >{\centering\arraybackslash}p{(\columnwidth - 6\tabcolsep) * \real{0.1724}}
  >{\centering\arraybackslash}p{(\columnwidth - 6\tabcolsep) * \real{0.1724}}
  >{\raggedright\arraybackslash}p{(\columnwidth - 6\tabcolsep) * \real{0.3793}}@{}}
\toprule
\begin{minipage}[b]{\linewidth}\raggedright
argument
\end{minipage} & \begin{minipage}[b]{\linewidth}\centering
class
\end{minipage} & \begin{minipage}[b]{\linewidth}\centering
units
\end{minipage} & \begin{minipage}[b]{\linewidth}\raggedright
description
\end{minipage} \\
\midrule
\endhead
\texttt{startYear} & integer & - & the start year of the scenario
(YYYY) \\
\texttt{endYear} & integer & - & the end year of the scenario (YYYY) \\
\texttt{relSeaLevelRiseInit} & numeric & cm \(\text{yr}^{-1}\) & initial
rate of of sea-level rise \\
\texttt{relSeaLevelRiseTotal} & numeric & cm & total relative sea-level
rise over the course of the simulation \\
\texttt{meanSeaLevel} & numeric & cm & initial mean sea level \\
& vector & cm & mean sea level for each year of the simulation \\
\texttt{meanSeaLevelDatum} & numeric & cm & mean sea level over the last
datum period \\
\texttt{meanHighWaterDatum} & numeric & cm & mean high water level over
the last datum period \\
\bottomrule
\end{longtable}

The first step in the workflow is to build the tidal scenario that the
simulation will iterate over. This works in two phases: building a
sea-level curve and building a high-tide scenario. The function
\protect\hyperlink{runcohortmem}{\texttt{runCohortMem()}} takes a
\texttt{startYear} and \texttt{endYear} (which is ported to the function
\protect\hyperlink{buildscenariocurve}{\texttt{buildScenarioCurve()}}),
in the form YYYY. There is some more flexibility in this R-package than
in previous online interfaces (\emph{e.g.}
\href{http://129.252.139.226/model/marsh/mem2.asp?t_zero=1991\&centu_sea_level=24\&mean_high_water=70\&mean_sea_level=-2\&init_rate_slr=0.24\&susp_sed_conc=20\&marsh_elev=43\&max_elev=90\&min_elev=-20\&max_peak_biomass=1017\&om_decay_rate=-0.8\&bg_input_mult=3\&bg_turnover=3\&Drmax=10\&LN_amp=3.1\&kr=0.02\&ks=.0322\&q=.0015\&chkSeas_Bio=on\&default=1}{MEM
online interface}). Note that this is not fixed at a 100 year forecast
as previous versions of CMEM.

The user then should set the initial rate of sea-level rise
\texttt{relSeaLevelRiseInit} and the total relative sea-level rise over
the course of the scenario \texttt{relSeaLevelRiseTotal}. These two
inputs control how the rate of sea-level rise accelerates over the
course of the simulation, or if it stays constant, depending on the
number of years of the simulation (\texttt{endYear} -
\texttt{startYear}). These inputs are only required if the simulation is
informed by a single value of mean sea-level at the beginning of the
simulation to then predict sea-level at further time points in the
simulation (see below). The function
\protect\hyperlink{buildscenariocurve}{\texttt{buildScenarioCurve()}}
will build a sea-level rise scenario assuming acceleration of sea-level
rise {[}@sweet2017global{]}:

\begin{align}
S_{\mu}(t) &= S_{\mu}(0) + At + Bt^2\\
A &= [S_{\mu}(1) - S_{\mu}(0)] - B\\
B &= \frac{\frac{S_{\mu}(T) - S_{\mu}(0)}{T} - [S_{\mu}(1) - S_{\mu}(0)]}{T - 1}
\end{align}

where \(S_{\mu}(t)\) is mean sea-level at time \(t\) and \(A\) and \(B\)
are coefficients of a quadratic regression with a specified initial mean
sea-level \(S_{\mu}(0)\).

Given the previous equations and specified inputs, we can substitute
\texttt{relSeaLevelRiseInit} = \(S_{\mu}(1) - S_{\mu}(0)\),
\texttt{relSeaLevelRiseTotal} = \(S_{\mu}(T) - S_{\mu}(0)\), and
\texttt{endYear} - \texttt{startYear} = \(T\).

\begin{align}
B &= \frac{\frac{\textbf{relSeaLevelRiseTotal}}{\textbf{endYear} - \textbf{startYear}} - [\textbf{relSeaLevelRiseInit}]}{\textbf{endYear} - \textbf{startYear} - 1}\\
A &= \textbf{relSeaLevelRiseInit} - B
\end{align}

Conversely, if a vector of mean sea-levels is to be used to inform the
tidal record (see below), \texttt{relSeaLevelRiseInit} and
\texttt{relSeaLevelRiseTotal} should not be specified.

The user will then input \texttt{meanSeaLevel} as either a numeric or a
vector. The function
\protect\hyperlink{buildscenariocurve}{\texttt{buildScenarioCurve()}}
will make a decision about what type of analysis you are trying to run
based on the structure of the input \texttt{meanSeaLevel}.
\texttt{meanSeaLevel} can either be an initial mean sea-level
(\(S_{\mu}(0)\)) or a vector of mean sea-levels, as long as the vector
is the same length as the number of years in the simulation. The vector
can be useful for hindcasting using a tide gauge record.

Finally, the user will input the mean sea level over the last datum
period \texttt{meanSeaLevelDatum} and the mean high water over the last
datum period \texttt{meanHighWaterDatum}. Tidal datums should represent
long term averages over a datum period. The last complete epoch was
1981-2001, released by NOAA
\href{https://tidesandcurrents.noaa.gov/datums.html?datum=NAVD88\&units=1\&epoch=0\&id=8575512\&name=Annapolis\&state=MD}{for
example}. In our examples we present tidal datums from NOAA tide gauges
as cm relative to the North American Vertical Datum of 1988 (NAVD88). If
high tide datums represent tidal amplitudes, then the user should set
\texttt{meanSeaLevelDatum} to 0.

\hypertarget{modifying-the-tidal-scenario-to-account-for-mean-high-high-water-and-spring-tides-optional}{%
\subsection{Modifying the tidal scenario to account for mean high high
water and spring tides
(optional)}\label{modifying-the-tidal-scenario-to-account-for-mean-high-high-water-and-spring-tides-optional}}

\begin{longtable}[]{@{}
  >{\raggedright\arraybackslash}p{(\columnwidth - 6\tabcolsep) * \real{0.2759}}
  >{\centering\arraybackslash}p{(\columnwidth - 6\tabcolsep) * \real{0.1724}}
  >{\centering\arraybackslash}p{(\columnwidth - 6\tabcolsep) * \real{0.1724}}
  >{\raggedright\arraybackslash}p{(\columnwidth - 6\tabcolsep) * \real{0.3793}}@{}}
\toprule
\begin{minipage}[b]{\linewidth}\raggedright
argument
\end{minipage} & \begin{minipage}[b]{\linewidth}\centering
class
\end{minipage} & \begin{minipage}[b]{\linewidth}\centering
units
\end{minipage} & \begin{minipage}[b]{\linewidth}\raggedright
description
\end{minipage} \\
\midrule
\endhead
\texttt{meanHighHighWaterDatum} & numeric & cm & (optional) mean higher
high tide water level over the last datum period \\
\texttt{meanHighHighWaterSpringDatum} & numeric & cm & (optional) mean
higher high spring tide water level over the last datum period \\
\bottomrule
\end{longtable}

The occurrence of exceptionally high tides within a day or across a
month provide more detail about a region's tidal cycle and thus will
produce a more accurate representation of the flooding experienced by
the marsh surface. If a user specifies only the average high tide level
\texttt{meanHighWaterDatum}, then all the tidal floods in the simulation
are apportioned to that datum. If a user specifies separate daily high
tides (\texttt{meanHighWaterDatum} and \texttt{meanHighHighWaterDatum}),
then half the tidal flood events are apportioned to each. The user can
also specify spring tides (\emph{i.e.} monthly tidal events where there
is the greatest differences between high and low tide) via
\texttt{meanHighHighWaterSpringDatum}. If all three types of tides are
defined (\texttt{meanHighWaterDatum}, \texttt{meanHighHighWaterDatum},
and \texttt{meanHighHighWaterSpringDatum}) then 50\% of the flood events
are apportioned to the lower daily high tides, 46.5\% apportioned to
daily higher high tides, and 3.5\% apportioned to bimonthly spring
tides. As before, all high tide datums should be relative to long-term
gauge averages or if they represent tidal amplitudes relative to mean
sea level, the user should make sure that \texttt{meanSeaLevelDatum} is
set to 0.

The choice of how many tidal data to useseem to have a marginal effect
on annual sedimentation at the higher end of the elevation gradient.
Given the parameter set used on this paper, more sedimentation is
modeled higher in the tidal frame when all three tidal datums are
defined as inputs compared to simply using only the weighted average
mean high tide.

\hypertarget{suspended-sediment-the-lunar-nodal-cycle-and-annualized-flooding-events}{%
\subsection{Suspended sediment, the lunar nodal cycle, and annualized
flooding
events}\label{suspended-sediment-the-lunar-nodal-cycle-and-annualized-flooding-events}}

\begin{longtable}[]{@{}
  >{\raggedright\arraybackslash}p{(\columnwidth - 6\tabcolsep) * \real{0.2759}}
  >{\centering\arraybackslash}p{(\columnwidth - 6\tabcolsep) * \real{0.1724}}
  >{\centering\arraybackslash}p{(\columnwidth - 6\tabcolsep) * \real{0.1724}}
  >{\raggedright\arraybackslash}p{(\columnwidth - 6\tabcolsep) * \real{0.3793}}@{}}
\toprule
\begin{minipage}[b]{\linewidth}\raggedright
argument
\end{minipage} & \begin{minipage}[b]{\linewidth}\centering
class
\end{minipage} & \begin{minipage}[b]{\linewidth}\centering
units
\end{minipage} & \begin{minipage}[b]{\linewidth}\raggedright
description
\end{minipage} \\
\midrule
\endhead
\texttt{suspendedSediment} & numeric & g \(\text{cm}^{-3}\) & suspended
sediment concentration of the water column \\
\texttt{lunarNodalAmp} & numeric & - & amplitude of the 18-year lunar
nodal cycle \\
\texttt{lunarNodalPhase} & numeric & decimal years & the start year of
the sine wave representing the lunar nodal cycle (YYYY) \\
\texttt{nFloods} & numeric & - & number of tidal flooding events per
year \\
\texttt{floodTime.fn} & function & - & method used to calculate flooding
time per tidal cycle (\texttt{floodTimeLinear} or
\texttt{floodTimeTrig}) \\
\texttt{captureRate} & numeric & - & number of times a water column will
clear per tidal cycle \\
\bottomrule
\end{longtable}

The user can specify the suspended sediment concentration (g
\(\text{cm}^{-3}\)) \texttt{suspendedSediment} to be as constant across
the course of the simulation. Similar to mean sea-level, suspended
sediment concentration can be either a single value, if assumed the same
for each year of the scenario, or a vector of values, as long as it is
the same length as number of years in the simulation. This could be
useful for hindcasting using a record of suspended sediment
concentration, using random walks to hindcast or forecast, or to build
scenarios such as those involving sediment augmentation, dam removals,
or sediment diversions. The function
\protect\hyperlink{buildscenariocurve}{\texttt{buildScenarioCurve()}}
outputs a table with a row for each year of the simulation that will be
used to track inputs, as well as stores some annualized outputs.

Parameters for the lunar nodal cycle (\texttt{lunarNodalAmp},
\texttt{lunarNodalPhase}) need to be specified locally. A reasonable
range for lunar nodal amplitudes is 2 to 10 cm {[}@peng2019tide{]}.
Absolute lunar nodal amplitude generally increases with latitude and
with tidal range, but decreases when considered as a fraction of tidal
range {[}@peng2019tide{]}. @peng2019tide found two clusters of phase for
the 18.61 lunar nodal cycle one for semi-diurnal and mixed tides, which
last peaked October of 2015, and one for diurnal tides, which last
peaked in June of 2006 {[}@peng2019tide{]}. Reasonable phase values to
set for \texttt{lunarNodalPhase} are 2001.848 +/- 0.5169 for diurnal
tides and 2011.181 +/- 0.56 for mixed and semi-diurnal tides
{[}@peng2019tide{]}.

Here is an example of a high tide scenario taking into account the lunar
nodal cycle (Fig \ref{fig:highTide}).

In CMEM the user inputs the number of floods in a year \texttt{nFloods}
rather than having the number of tides be a fixed constant (as in
previous versions of MEM) because the number of tides each year can vary
regionally. The \texttt{nFloods} input can be specified as any numeric
value we recommend the following values: for diurnal tides we recommend
352.89 tidal cycles per year. There are 365.25 days per year (accounting
for leap years), 24 hours per day, and a diurnal tidal cycle which is
24.84 hours long (\(\frac{365.25 \times 24}{24.84} \approx 352.89\)).
For semi-diurnal or mixed semi-diurnal tide regimes, we recommend 705.80
tidal cycles per year, since a mixed or semi-diurnal tidal cycle is
12.42 hours long (\(\frac{365.25 \times 24}{12.42} \approx 705.80\)).

There are two methods for calculating fractional flooding time in this
R-package. Users can specify a flood time function by setting the input
\texttt{floodTime.fn} to the function name; users use the generic linear
method \protect\hyperlink{floodtimelinear}{\texttt{floodTimeLinear()}},
or specify the trigonometric function
\protect\hyperlink{floodtimetrig}{\texttt{floodTimeTrig()}}. This
formulation is used in XX versions of MEM (need particulars). Both
functions apply a tidal flooding function \(f(Z)\), where \(Z\) is the
elevation of the marsh surface, if the elevation is between the flood
and ebb levels. For both methods, fractional flooding time is capped at
1 if \(Z\) is lower than lowest flood elevation (100\% flooded), and 0
(0\% flooded) if the elevation is greater than the highest water level.
For the simpler linear method, fractional inundation time is assumed to
have a linear relationship to the relative tidal range, 0 above the
average flood depth, 1 below the average ebb depth, and a linear
interpolation in between.

\begin{equation}
\label{eqn:tidalflood}
f(Z) = \frac{H_{i} - Z}{H_i-L_i}
\end{equation}

Fraction flooding time for tidal class \(i\) (\(t_{f,i}\)) is calculated
as:

\begin{equation} 
\label{eqn:fracflood}
t_{f,i} = \left\{
        \begin{array}{ll}
            0 & \quad  Z_{H,i} < Z \\
             f(Z) & \quad Z_{L,i} \leq  Z \leq Z_{H,i} \\
             1 & \quad  Z < Z_{L,i} 
        \end{array}
    \right.  
\end{equation}

Alternatively, we present the trigonometric function option
{[}\textbackslash emph\{sensu\} @hickey2019tidal{]} which allows
sedimentation processes to be more sensitive for high marshes (that are
not frequently flooded) without substantially affecting run time.
Flooding time is calculated as the product of the absolute value of the
rising time \(t_r\) minus \(\phi\), which is the time of one half of a
tidal cycle, and the falling time \(t_l\) for tidal cycle class \(i\).

\begin{align} 
t_{f,i} &= f(Z) = |t_{r,i} - \phi| + t_{l,i} \\
t_{r,i} &=  \sin\left(\phi \left[\frac{A_{1,i}}{\pi}\right] - 1 \right) \\
A_{1,i} &= 2 \pi - \arccos\left(2 \left[\frac{Z-L_i}{H_i-L_i}\right] - 1\right)\\
t_{l,i} &=  \sin\left(\phi \left[\frac{A_{2,i}}{\pi}\right] - 1\right) \\
A_{2,i} &= 2 \pi - \arccos\left(2 \left[\frac{Z-H_i}{L_i-H_i}\right] - 1\right)
\end{align}

Here, we \(\phi\) to 0.5 (\emph{i.e.} half of a tidal cycle), which
allows the output to be equal to the fraction for which an elevation is
flooded (\emph{i.e.} fractional flooding time). The
\protect\hyperlink{floodtimetrig}{\texttt{floodTimeTrig()}} function
calculates flooding time as the product of the absolute value of the
rising time \(t_r\) minus \(\Phi\) (the time of one half of a tidal
cycle), and the falling time \(t_l\) for tidal cycle class \(i\).

The function
\protect\hyperlink{deliversediment}{\texttt{deliverSediment()}}
organizes inputs and in the cases where daily higher high tides and
monthly high spring tides are specified, proportions the number of tidal
floods among the types of tides specified. It also uses a user-specified
capture rate \texttt{captureRate} (the number of times a water column
will clear per tidal cycle) multiplied by the the amount of time the
surface is flooded to calculate the amount of available sediment that is
delivered.

Here, we show the calculated sediment delivered as a function of mean
sea level and four different possible elevations (10cm, 20cm, 30cm,
40cm) using the tidal data generated in high tide scenario assuming a
simple tidal scenario (\(i.e.\) the user did not input a value for
\texttt{meanHighHighWaterDatum} or
\texttt{meanHighHighWaterSpringDatum}). This does \emph{not} reflect the
suspended sediment trajectory of a simulation because elevation changes
as a function of vegetation (see below), but rather shows how the amount
of sediment delivered depends both on the tidal scenario and the
elevation. In particular, for supratidal marshes, the amount of sediment
delivered is zero.

If the user does specify a \texttt{meanHighHighWaterDatum} and
\texttt{meanHighHighWaterSpringDatum}, then the high tides are
partitioned before the amount of sediment delivered is calculated.

As seen below with comparison to the previous sediment graph, there is
not considerable change between the amount of sediment delivered when
using the simple sediment delivery function versus one that takes into
account high high tide and spring high high tide.

\hypertarget{determining-initial-cohorts}{%
\subsection{Determining initial
cohorts}\label{determining-initial-cohorts}}

For the numerical model produce realistic predictions, the model
requires an initialization of cohorts at equilibrium (\emph{i.e.} until
the oldest cohorts have mass pools that are not significantly different)
given that the oldest cohort is above a minimum age (default = 50
years), below a maximum age (default = 12000 years), and spans up to a
minimum depth. In practice, this initialization is completed by the
\protect\hyperlink{runtoequilibrium}{\texttt{runToEquilibrium()}}
function, where in new cohorts are repeatedly added to a soil profile.
The initialization of soil cohorts is informed by the initial
belowground biomass (applying the function
\protect\hyperlink{predictbiomass}{\texttt{predictBiomass()}} at the
initial elevation \texttt{initElv} and multiplying by the root-to-shoot
ratio \texttt{rootToShoot}), fast and slow pool decay rates
\texttt{omDecayRate}, recalcitrant fraction of organic material
\texttt{recalcitrantFrac}, the organic and mineral self-packing
densities (\texttt{omPackingDensity}, \texttt{mineralPackingDensity}),
the shape of the belowground biomass distribution (\texttt{shape};
``linear'' or ``exponential''), the initial mineral input
(\texttt{mineralInput\_g\_per\_yr}; calculated using the
\protect\hyperlink{deliversediment}{\texttt{deliverSediment()}}
function), and the minimum depth for which the cohorts must be
initialize which set to a default of the maximum rooting depth
\texttt{rootDepthMax} plus 0.5cm, rounded to the nearest cm.

If the initial elevation is above the highest tidal flooding elevation,
but below the maximum growing elevation \texttt{zVegMax}, then there
will be no mineral sedimentation to lay down annualized layers, and
\protect\hyperlink{runtoequilibrium}{\texttt{runToEquilibrium()}} will
fail with an error message. In this case
\protect\hyperlink{determineinitialcohorts}{\texttt{determineInitialCohorts()}}
will form a supertidal peat. The user has some options; a user can
provide their own supertidal sediment input
(\texttt{superTidalSedimentInput}; g \(\text{cm}^{-2}\)
\(\text{year}^{-1}\)). If this is not supplied, then
\protect\hyperlink{determineinitialcohorts}{\texttt{determineInitialCohorts()}}
will calculate sediment input at 1 cm below the highest tide and use
that. The
\protect\hyperlink{runtoequilibrium}{\texttt{runToEquilibrium()}}
function then runs the same as if initial elevation were an elevation in
which there was flooding elevation. The user also has an option to use
their own set of cohorts for super tidal peat, each with mineral, root,
fast and slow pool organic matter pool masses via the
\texttt{superTidalCohorts} input.

If the initial elevation is above the both the highest tidal flooding
elevation and the maximum growing elevation \texttt{zVegMax} of plants,
then the initial condition is an upland soil. This is possible when
modeling a marsh traversing into an adjacent upland as sea-level rises
(\emph{i.e.} horizontal transgression). A user can supply their own
upland soil cohorts as an input, as long as it takes the form of a table
with minimum and maximum depths, and ages of a stack of soil cohorts,
each with mineral, root, fast and slow pool organic matter pool masses.
If this argument is not provided, then
\protect\hyperlink{determineinitialcohorts}{\texttt{determineInitialCohorts()}}
will initialize an adjacent upland as a stack of 1 cm wide cohorts, each
with 0 as their age, and each with 50\% slow pool organic matter and
50\% mineral matter.

Finally, an alternative to letting
\protect\hyperlink{determineinitialcohorts}{\texttt{determineInitialCohorts()}}
make the decisions about how to initialize a sediment column, a user can
specify their own \texttt{initialCohorts}. This will cause the function
to jump over all of the previously described decisions and return the
\texttt{initialCohorts} file. This is useful when you are pairing a
hindcast with a forecast.

\hypertarget{predicting-plant-biomass-given-elevation-and-flooding}{%
\subsection{Predicting plant biomass given elevation and
flooding}\label{predicting-plant-biomass-given-elevation-and-flooding}}

\begin{longtable}[]{@{}
  >{\raggedright\arraybackslash}p{(\columnwidth - 6\tabcolsep) * \real{0.2759}}
  >{\centering\arraybackslash}p{(\columnwidth - 6\tabcolsep) * \real{0.1724}}
  >{\centering\arraybackslash}p{(\columnwidth - 6\tabcolsep) * \real{0.1724}}
  >{\raggedright\arraybackslash}p{(\columnwidth - 6\tabcolsep) * \real{0.3793}}@{}}
\toprule
\begin{minipage}[b]{\linewidth}\raggedright
argument
\end{minipage} & \begin{minipage}[b]{\linewidth}\centering
class
\end{minipage} & \begin{minipage}[b]{\linewidth}\centering
units
\end{minipage} & \begin{minipage}[b]{\linewidth}\raggedright
description
\end{minipage} \\
\midrule
\endhead
\texttt{bMax} & numeric & g \(\text{cm}^2\) & peak aboveground
biomass \\
\texttt{zVegMin} & numeric & cm* & lower elevation of biomass limit \\
\texttt{zVegMax} & numeric & cm* & upper elevation of biomass limit \\
\texttt{zVegPeak} & numeric & cm* & (optional) elevation of peak
biomass \\
\texttt{plantElevationType} & character & - & ``orthometric'' or
``dimensionless'', specifying elevation reference of the vegetation
growing elevations (*) \\
\bottomrule
\end{longtable}

The user inputs multiple parameters that are related to the relationship
between elevation and biomass, such that a given time step, the amount
of biomass can be predicted for a given square meter area (10000
\(\text{cm}^2\)). There is assumed to be a parabolic relationship
between elevation and biomass {[}@morris2002responses{]} which is
parameterized (at a minimum) by a peak biomass \texttt{bMax}, a lower
elevation limit of biomass \texttt{zVegMin}, and a upper elevation limit
of biomass \texttt{zVegMax}. This specification assumes a symmetrical
inverted parabola.

The user also has the option to provide an additional parameter
\texttt{zVegPeak} which allows for the elevation of peak biomass to be
different than the midpoint elevation of the lower and upper elevation
limits (\emph{i.e.} an asymmetrical parabola). Finally, the user
specifies \texttt{plantElevationType} as either ``orthometric'', which
means that the values of \texttt{zVegMin}, \texttt{zVegMax}, and
\texttt{zVegPeak} (if provided) are in cm NAVD88, or ``dimensionless''
which means that the values of \texttt{zVegMin}, \texttt{zVegMax}, and
\texttt{zVegPeak} (if provided) are relative to mean sea-level.

The function
\protect\hyperlink{predictbiomass}{\texttt{predictBiomass()}} takes the
inputs \texttt{bMax}, \texttt{zVegMin}, \texttt{zVegMax}, and
\texttt{zVegPeak} (if provided) as well as the elevation at the current
timestep to calculate aboveground biomass. Because regional tidal
regimes influence the relationship between elevation and biomass
production, elevation values for \texttt{zVegMin}, \texttt{zVegMax}, and
\texttt{zVegPeak} (if provided) are converted to dimensionless
elevations (\(Z^*\)) if the elevations are specified orthometrically
(\texttt{plantElevationType} = ``orthometric''). The function
\protect\hyperlink{convertztozstar}{\texttt{convertZToZstar()}}
completes this conversion before the aboveground biomass is calculated.
Similarly, the current surface elevation at the timestep in which the
biomass prediction is being made is converted to \(Z^*\) using the same
function
\protect\hyperlink{convertztozstar}{\texttt{convertZToZstar()}}.

Here we show both a symmetric and asymmetric parabola first with biomass
as a function of elevation in cm NAVD88 and then as a function of
\(Z^*\) to illustrate that the functional relationship does not change
with the relative tidal elevation transformation.

\hypertarget{specifying-additional-plant-trait-parameters}{%
\subsection{Specifying additional plant trait
parameters}\label{specifying-additional-plant-trait-parameters}}

\begin{longtable}[]{@{}
  >{\raggedright\arraybackslash}p{(\columnwidth - 6\tabcolsep) * \real{0.2759}}
  >{\centering\arraybackslash}p{(\columnwidth - 6\tabcolsep) * \real{0.1724}}
  >{\centering\arraybackslash}p{(\columnwidth - 6\tabcolsep) * \real{0.1724}}
  >{\raggedright\arraybackslash}p{(\columnwidth - 6\tabcolsep) * \real{0.3793}}@{}}
\toprule
\begin{minipage}[b]{\linewidth}\raggedright
argument
\end{minipage} & \begin{minipage}[b]{\linewidth}\centering
class
\end{minipage} & \begin{minipage}[b]{\linewidth}\centering
units
\end{minipage} & \begin{minipage}[b]{\linewidth}\raggedright
description
\end{minipage} \\
\midrule
\endhead
\texttt{rootToShoot} & numeric & g/g & root and rhizome to shoot
ratio \\
\texttt{rootTurnover} & numeric & \(\text{yr}^{-1}\) & belowground
biomass annual turnover rate \\
\texttt{abovegroundTurnover} & numeric & \(\text{yr}^{-1}\) &
aboveground biomass annual turnover rate \\
\texttt{speciesCode} & character & - & (optional) species names or codes
associated with biological inputs \\
\texttt{rootDepthMax} & numeric & cm & depth of 95\% cumulative
belowground biomass \\
\texttt{shape} & numeric & - & ``linear'' or ``exponential'' to describe
the shape of the relationship between depth and belowground biomass \\
\texttt{omDecayRate} & numeric & \(\text{yr}^{-1}\) & annual fractional
mass lost due to decay \\
\texttt{recalcitrantFrac} & numeric & g/g & fraction of organic matter
resistant to decay \\
\bottomrule
\end{longtable}

Within \protect\hyperlink{runcohortmem}{\texttt{runCohortMem()}} the
predicted aboveground biomass is converted to belowground biomass via a
user-provided root-to-shoot ratio \texttt{rootToShoot}. Note that this
ratio actually represents the root \emph{and} rhizome to shoot ratio and
therefore can more accurately be described as a ratio of belowground to
aboveground biomass allocation. We use the term root-to-shoot ratio and
the argument \texttt{rootToShoot} to reflect what is in previous
publications and versions of CMEM and similar models.

Within the \protect\hyperlink{massliveroots}{\texttt{massLiveRoots()}}
function, the belowground biomass is distributed across the depth of the
core (across the cohorts) which is mediated both by the user-specified
depth at which 95\% of the belowground biomass has accumulated
\texttt{rootDepthMax} and the \texttt{shape} of the belowground biomass
distribution (``linear'' or ``exponential''). Within each cohort, the
belowground biomass is turned over via the \texttt{rootTurnover}
parameter. Finally, a fraction of the turned over biomass in each cohort
(1 - \texttt{recalcitrantFrac}) is decayed at a user-specified organic
matter decay rate \texttt{omDecayRate} (\(\text{yr}^{-1}\)). As a
default, the recalcitrant fraction of biomass does not decay.

The user also has an option to specify an aboveground turnover rate
(\(\text{yr}^{-1}\)), however this is not required.

\hypertarget{multi-species-functionality-optional}{%
\subsection{Multi-species functionality
(optional)}\label{multi-species-functionality-optional}}

The \texttt{rCMEM} package allows for multiple species with different
elevation limits and biomass parameters to be included in the modeling
framework. The
\protect\hyperlink{runmultispeciesbiomass}{\texttt{runMultiSpeciesBiomass()}}
function will accept vectors of biological inputs, run
\protect\hyperlink{predictbiomass}{\texttt{predictBiomass()}} on each
set, and will return a single set of parameters based on a competition
function (\emph{sensu} @morris2006competition). In short,
\protect\hyperlink{runmultispeciesbiomass}{\texttt{runMultiSpeciesBiomass()}}
predicts aboveground biomass for multiple species given the relative
tidal elevation at the current timestep and then selects the species
that has the largest aboveground biomass (\emph{i.e.} the dominant
species). Then the rest of the plant trait parameters (\emph{e.g.}
root-to-shoot ratio, root turnover) are defined by the dominant species.
In the case where multiple species have the same predicted aboveground
biomass at a given elevation, for example, at an intersection of the two
parabolas, the rest of the plant trait parameters are an average across
the number of species that share the same predicted biomass.

In practice, for a multi-species prediction, the following inputs in
\protect\hyperlink{runcohortmem}{\texttt{runCohortMem()}} should be a
vector with a length equal to the number of species and including either
numeric or character values based on the argument description:

\begin{itemize}
\tightlist
\item
  \texttt{bMax}
\item
  \texttt{zVegMin}
\item
  \texttt{zVegMax}
\item
  \texttt{zVegPeak} (optional)
\item
  \texttt{rootToShoot}
\item
  \texttt{rootTurnover}
\item
  \texttt{rootDepthMax}
\item
  \texttt{abovegroundTurnover} (optional)
\item
  \texttt{speciesCode} (optional)
\end{itemize}

Users can specify species names or codes if they want to explicitly
track the change in the dominant species over the course of the
simulation via the argument \texttt{speciesCode}. If these are not
specified, they will either be labeled sequentially \emph{species1},
\emph{species2}, etc. or \emph{unvegetated}. Users can specify their own
competition function as long as they result in a one-row table with each
column representing a biological parameter; users could potentially
simulate competitive exclusion or facilitation by using species weighted
averages of the biological parameters based on aboveground biomass,
rather than selecting the set of parameters that produces the maximum.
Currently there is not an example implementation of custom competition
function.

\hypertarget{adding-a-cohort}{%
\subsection{Adding a Cohort}\label{adding-a-cohort}}

\begin{longtable}[]{@{}
  >{\raggedright\arraybackslash}p{(\columnwidth - 6\tabcolsep) * \real{0.2759}}
  >{\centering\arraybackslash}p{(\columnwidth - 6\tabcolsep) * \real{0.1724}}
  >{\centering\arraybackslash}p{(\columnwidth - 6\tabcolsep) * \real{0.1724}}
  >{\raggedright\arraybackslash}p{(\columnwidth - 6\tabcolsep) * \real{0.3793}}@{}}
\toprule
\begin{minipage}[b]{\linewidth}\raggedright
argument
\end{minipage} & \begin{minipage}[b]{\linewidth}\centering
class
\end{minipage} & \begin{minipage}[b]{\linewidth}\centering
units
\end{minipage} & \begin{minipage}[b]{\linewidth}\raggedright
description
\end{minipage} \\
\midrule
\endhead
\texttt{omPackingDensity} & numeric & g \(\text{cm}^{-3}\) & bulk
density of pure organic matter \\
\texttt{mineralPackingDensity} & numeric & g \(\text{cm}^{-3}\) & bulk
density of pure mineral matter \\
\texttt{rootPackingDensity} & numeric & g \(\text{cm}^{-3}\) & bulk
density of pure root matter \\
\bottomrule
\end{longtable}

At each time step, soil cohorts are aged by one timestep (default = 1
year) and then are modified in the following steps within the
\protect\hyperlink{addcohort}{\texttt{addCohort()}} function:

\begin{enumerate}
\def\labelenumi{\arabic{enumi}.}
\tightlist
\item
  Roots are turned over and organic matter is decayed.
\end{enumerate}

\begin{itemize}
\tightlist
\item
  During this step, the new fast \(C_f\) and slow \(C_s\) carbon mass
  pools of age \(a\) at timestep \(t\) are calculated:
  \[C_f(a,t+1) = \text{exp}(k_f)(C_f(a,t) + B_{bg}(a,t)f_f k_r)\]
  \[C_s(a,t+1) = \text{exp}(k_s)(C_s(a,t) + B_{bg}(a,t)(1-f_f)k_r)\]
\end{itemize}

where \(k_f\) and \(k_s\) are the decay rates of fast pool and slow pool
organic material (\texttt{omDecayRate}), \(f_f\) is the fractions of
organic matter resistant to decay (\texttt{recalcitrantFrac}), and
\(B_{bg}\) is the mass of live belowground biomass.

\begin{itemize}
\tightlist
\item
  Additionally, respired carbon is calculated as:
  \[C_r(a,t+1) = (1 - \text{exp}(k_f))\times (C_f(a,t) + B_{bg}(a,t)f_f k_r) \\ 
    + (1 - \text{exp}(k_s))\times (C_s(a,t) + B_{bg}(a,t)(1-f_f) k_r)\]
\end{itemize}

\begin{enumerate}
\def\labelenumi{\arabic{enumi}.}
\setcounter{enumi}{1}
\tightlist
\item
  If there are non-zero mineral inputs, a new cohort of mineral sediment
  is added to the top of the soil profile.
\item
  The top and bottom depths of the cohort layers are recalculated using
  the function
  \protect\hyperlink{calculatedepthofnonrootvolume}{\texttt{calculateDepthOfNonRootVolume()}}
  after changes in mass of the cohorts are transformed into volumes via
  user-specified self-packing densities (\texttt{omPackingDensity},
  \texttt{mineralPackingDensity}, \texttt{rootPackingDensity}).
\item
  A new root profile is constructed based on the modified cohort layers
  using the functions
  \protect\hyperlink{massliveroots}{\texttt{massLiveRoots()}}.
\end{enumerate}

\pagebreak

\hypertarget{reference-code}{%
\section{Reference code}\label{reference-code}}

\hypertarget{addcohort}{%
\subsection{\texorpdfstring{\texttt{addCohort()}}{addCohort()}}\label{addcohort}}

\begin{Shaded}
\begin{Highlighting}[]
\FunctionTok{options}\NormalTok{(}\AttributeTok{width =} \DecValTok{60}\NormalTok{)}
\NormalTok{addCohort}\OtherTok{\textless{}{-}}\ControlFlowTok{function}\NormalTok{(massPools,}
\NormalTok{                    rootTurnover, rootOmFrac, omDecayRate, }\CommentTok{\#decay paraemters}
\NormalTok{                    packing, }\CommentTok{\#packing densities}
                    \AttributeTok{mineralInput =} \ConstantTok{NA}\NormalTok{,}
                    \AttributeTok{mineralInput.fn =}\NormalTok{ sedimentInputs, }
                      \AttributeTok{massLiveRoots.fn =}\NormalTok{ massLiveRoots,}
                      \AttributeTok{calculateDepthOfNonRootVolume.fn=}\NormalTok{calculateDepthOfNonRootVolume,}
                      \AttributeTok{timeStep=}\DecValTok{1}\NormalTok{, ...)\{}
  
  \CommentTok{\#Sanity check the inputs}
  \ControlFlowTok{if}\NormalTok{(}\SpecialCharTok{!}\FunctionTok{all}\NormalTok{(}\FunctionTok{c}\NormalTok{(}\StringTok{\textquotesingle{}age\textquotesingle{}}\NormalTok{, }\StringTok{\textquotesingle{}fast\_OM\textquotesingle{}}\NormalTok{, }\StringTok{\textquotesingle{}slow\_OM\textquotesingle{}}\NormalTok{, }\StringTok{\textquotesingle{}mineral\textquotesingle{}}\NormalTok{, }\StringTok{\textquotesingle{}layer\_top\textquotesingle{}}\NormalTok{, }\StringTok{\textquotesingle{}layer\_bottom\textquotesingle{}}\NormalTok{,}
            \StringTok{\textquotesingle{}root\_mass\textquotesingle{}}\NormalTok{) }\SpecialCharTok{\%in\%} 
          \FunctionTok{names}\NormalTok{(massPools)))\{}
    \FunctionTok{stop}\NormalTok{(}\StringTok{\textquotesingle{}Badly named massPools\textquotesingle{}}\NormalTok{)}
\NormalTok{  \}}
  
  \ControlFlowTok{if}\NormalTok{(}\SpecialCharTok{!}\FunctionTok{all}\NormalTok{(}\FunctionTok{c}\NormalTok{(}\StringTok{\textquotesingle{}organic\textquotesingle{}}\NormalTok{, }\StringTok{\textquotesingle{}mineral\textquotesingle{}}\NormalTok{) }\SpecialCharTok{\%in\%} \FunctionTok{names}\NormalTok{(packing)))\{}
    \FunctionTok{stop}\NormalTok{(}\StringTok{\textquotesingle{}Can not find expected packing densities.\textquotesingle{}}\NormalTok{)}
\NormalTok{  \}}
  
  \ControlFlowTok{if}\NormalTok{(}\SpecialCharTok{!}\FunctionTok{all}\NormalTok{(}\FunctionTok{c}\NormalTok{(}\StringTok{\textquotesingle{}fast\textquotesingle{}}\NormalTok{, }\StringTok{\textquotesingle{}slow\textquotesingle{}}\NormalTok{) }\SpecialCharTok{\%in\%} \FunctionTok{names}\NormalTok{(rootOmFrac)))\{}
    \FunctionTok{stop}\NormalTok{(}\StringTok{\textquotesingle{}Can not find expected root fraction splits.\textquotesingle{}}\NormalTok{)}
\NormalTok{  \}}
  
  \ControlFlowTok{if}\NormalTok{(}\SpecialCharTok{!}\FunctionTok{all}\NormalTok{(}\FunctionTok{c}\NormalTok{(}\StringTok{\textquotesingle{}fast\textquotesingle{}}\NormalTok{, }\StringTok{\textquotesingle{}slow\textquotesingle{}}\NormalTok{) }\SpecialCharTok{\%in\%} \FunctionTok{names}\NormalTok{(omDecayRate)))\{}
    \FunctionTok{stop}\NormalTok{(}\StringTok{\textquotesingle{}Can not find expected organic matter decay rates.\textquotesingle{}}\NormalTok{)}
\NormalTok{  \}}
  
  \CommentTok{\#copy over to an answer data frame}
\NormalTok{  ans }\OtherTok{\textless{}{-}}\NormalTok{ massPools}
  
\NormalTok{  ans}\SpecialCharTok{$}\NormalTok{age }\OtherTok{\textless{}{-}}\NormalTok{ ans}\SpecialCharTok{$}\NormalTok{age }\SpecialCharTok{+}\NormalTok{ timeStep }\CommentTok{\#age the cohorts}
  
  \CommentTok{\# Convert decay rates to decay coefficients (k) in case in the future we want}
  \DocumentationTok{\#\# to model decay at sub{-}annual time steps}
  \DocumentationTok{\#\# C\_t = C0 * exp(kt)}
\NormalTok{  k\_fast}\OtherTok{\textless{}{-}}\FunctionTok{log}\NormalTok{(}\DecValTok{1}\SpecialCharTok{{-}}\NormalTok{omDecayRate}\SpecialCharTok{$}\NormalTok{fast)}
\NormalTok{  k\_slow}\OtherTok{\textless{}{-}}\FunctionTok{log}\NormalTok{(}\DecValTok{1}\SpecialCharTok{{-}}\NormalTok{omDecayRate}\SpecialCharTok{$}\NormalTok{slow)}

  \CommentTok{\# track respiration}
  
  \DocumentationTok{\#\# total respired belowground OM = ((cumulative fast OM from previous time}
  \DocumentationTok{\#\# steps + fast OM from this time step) * fraction lost to decay) +}
  \DocumentationTok{\#\# ((sumulative slow pool OM from previous time steps + slow pool from this}
  \DocumentationTok{\#\# time step) * fraction lost to decay)}
  
  \DocumentationTok{\#\# Fraction slow pool lost to decay will be 0 the way the inputs to this}
  \DocumentationTok{\#\# function are set but this formulation sets us up to integrate slow pool}
  \DocumentationTok{\#\# organic matter decay in a future iteration.}
\NormalTok{  ans}\SpecialCharTok{$}\NormalTok{respired\_OM }\OtherTok{\textless{}{-}}\NormalTok{ ((ans}\SpecialCharTok{$}\NormalTok{fast\_OM }\SpecialCharTok{+}
\NormalTok{    (ans}\SpecialCharTok{$}\NormalTok{root\_mass }\SpecialCharTok{*}\NormalTok{ rootOmFrac}\SpecialCharTok{$}\NormalTok{fast }\SpecialCharTok{*}\NormalTok{ rootTurnover }\SpecialCharTok{*}\NormalTok{ timeStep)) }\SpecialCharTok{*} 
\NormalTok{    (}\DecValTok{1}\SpecialCharTok{{-}}\FunctionTok{exp}\NormalTok{(k\_fast}\SpecialCharTok{*}\NormalTok{timeStep))) }\SpecialCharTok{+}
\NormalTok{    ((ans}\SpecialCharTok{$}\NormalTok{slow\_OM }\SpecialCharTok{+} 
\NormalTok{       ans}\SpecialCharTok{$}\NormalTok{root\_mass }\SpecialCharTok{*}\NormalTok{ rootOmFrac}\SpecialCharTok{$}\NormalTok{slow }\SpecialCharTok{*}\NormalTok{ rootTurnover }\SpecialCharTok{*}\NormalTok{ timeStep) }\SpecialCharTok{*}
\NormalTok{    (}\DecValTok{1}\SpecialCharTok{{-}}\FunctionTok{exp}\NormalTok{(k\_slow}\SpecialCharTok{*}\NormalTok{timeStep)))}
  
  \CommentTok{\#add and decay the organic matter \# New fast pool OM = (cumulative fast OM}
  \CommentTok{\#from previous time steps + new fast OM) * fraction remaining after decay}
\NormalTok{  ans}\SpecialCharTok{$}\NormalTok{fast\_OM }\OtherTok{\textless{}{-}}\NormalTok{ (ans}\SpecialCharTok{$}\NormalTok{fast\_OM }\SpecialCharTok{+}
\NormalTok{                  (ans}\SpecialCharTok{$}\NormalTok{root\_mass }\SpecialCharTok{*}\NormalTok{ rootOmFrac}\SpecialCharTok{$}\NormalTok{fast }\SpecialCharTok{*}\NormalTok{ rootTurnover }\SpecialCharTok{*}\NormalTok{ timeStep)) }\SpecialCharTok{*} 
    \FunctionTok{exp}\NormalTok{(k\_fast}\SpecialCharTok{*}\NormalTok{timeStep)}
  
  \CommentTok{\#New slow pool OM = (cumulative slow OM from previous time steps + new slow}
  \CommentTok{\#OM) * fraction remaining after decay \# Fraction remaining after decay will be}
  \CommentTok{\#1 unless we add flexibiltiy in inputs upstream of this function in a future}
  \CommentTok{\#version}
\NormalTok{  ans}\SpecialCharTok{$}\NormalTok{slow\_OM }\OtherTok{\textless{}{-}}\NormalTok{ (ans}\SpecialCharTok{$}\NormalTok{slow\_OM }\SpecialCharTok{+} 
\NormalTok{             ans}\SpecialCharTok{$}\NormalTok{root\_mass }\SpecialCharTok{*}\NormalTok{ rootOmFrac}\SpecialCharTok{$}\NormalTok{slow }\SpecialCharTok{*}\NormalTok{ rootTurnover }\SpecialCharTok{*}\NormalTok{ timeStep) }\SpecialCharTok{*}
    \FunctionTok{exp}\NormalTok{(k\_slow}\SpecialCharTok{*}\NormalTok{timeStep)}
             
  \CommentTok{\# Check to see if mineral input is a static value or a function}
  \ControlFlowTok{if}\NormalTok{ (}\FunctionTok{is.na}\NormalTok{(mineralInput)) \{}
\NormalTok{    mineralInput }\OtherTok{\textless{}{-}} \FunctionTok{mineralInput.fn}\NormalTok{(...)}
\NormalTok{  \}}
  
  \CommentTok{\#if we are laying down a new cohort}
  \ControlFlowTok{if}\NormalTok{(mineralInput }\SpecialCharTok{\textgreater{}} \DecValTok{0}\NormalTok{)\{}
    \ControlFlowTok{if}\NormalTok{(}\SpecialCharTok{!}\FunctionTok{any}\NormalTok{(}\FunctionTok{is.na}\NormalTok{(ans}\SpecialCharTok{$}\NormalTok{age)))\{ }\CommentTok{\#if there aren\textquotesingle{}t any empty cohort slots}
\NormalTok{      bufferAns }\OtherTok{\textless{}{-}}\NormalTok{ ans}
\NormalTok{      bufferAns[}\ConstantTok{TRUE}\NormalTok{] }\OtherTok{\textless{}{-}} \ConstantTok{NA}
\NormalTok{      ans }\OtherTok{\textless{}{-}} \FunctionTok{rbind}\NormalTok{(bufferAns, ans) }\CommentTok{\#double the number of slots}
\NormalTok{    \}}
    
    \CommentTok{\#Take the last empty cohort slot}
    
    \DocumentationTok{\#\#Note that by buffering the profile with empty cohort slots we do not need}
    \DocumentationTok{\#\#to copy over the ...entire data frame when adding a single cohort. This}
    \DocumentationTok{\#\#increases runtime when this function is ...embedided in interative runs.}
\NormalTok{    newCohortIndex }\OtherTok{\textless{}{-}} \FunctionTok{max}\NormalTok{(}\FunctionTok{which}\NormalTok{(}\FunctionTok{is.na}\NormalTok{(ans}\SpecialCharTok{$}\NormalTok{age)))}
    
    \CommentTok{\#initalize the age and organic carbon pools to 0}
\NormalTok{    ans}\SpecialCharTok{$}\NormalTok{age[newCohortIndex] }\OtherTok{\textless{}{-}} \DecValTok{0}
\NormalTok{    ans}\SpecialCharTok{$}\NormalTok{fast\_OM[newCohortIndex] }\OtherTok{\textless{}{-}} \DecValTok{0}
\NormalTok{    ans}\SpecialCharTok{$}\NormalTok{slow\_OM[newCohortIndex] }\OtherTok{\textless{}{-}} \DecValTok{0}

\NormalTok{    ans}\SpecialCharTok{$}\NormalTok{respired\_OM[newCohortIndex] }\OtherTok{\textless{}{-}} \DecValTok{0}
    
    \CommentTok{\#lay down the new mineral inputs}
\NormalTok{    ans}\SpecialCharTok{$}\NormalTok{mineral[newCohortIndex] }\OtherTok{\textless{}{-}}\NormalTok{ mineralInput }\SpecialCharTok{*}\NormalTok{ timeStep}
    
\NormalTok{  \}}
  
  \CommentTok{\#calculate the volume of each cohort}
\NormalTok{  temp\_Vol }\OtherTok{\textless{}{-}}\NormalTok{ (ans}\SpecialCharTok{$}\NormalTok{fast\_OM }\SpecialCharTok{+}\NormalTok{ ans}\SpecialCharTok{$}\NormalTok{slow\_OM)}\SpecialCharTok{/}\NormalTok{packing}\SpecialCharTok{$}\NormalTok{organic }\SpecialCharTok{+}
\NormalTok{    ans}\SpecialCharTok{$}\NormalTok{mineral}\SpecialCharTok{/}\NormalTok{packing}\SpecialCharTok{$}\NormalTok{mineral}
  \CommentTok{\#replace NA with 0 so we can calculate cumulative volume}
\NormalTok{  temp\_Vol[}\FunctionTok{is.na}\NormalTok{(temp\_Vol)] }\OtherTok{\textless{}{-}} \DecValTok{0} 
\NormalTok{  ans}\SpecialCharTok{$}\NormalTok{cumCohortVol }\OtherTok{\textless{}{-}} \FunctionTok{cumsum}\NormalTok{(temp\_Vol)}
  
  \CommentTok{\#calculate depth profile}
\NormalTok{  ans}\SpecialCharTok{$}\NormalTok{layer\_bottom }\OtherTok{\textless{}{-}} \FunctionTok{calculateDepthOfNonRootVolume.fn}\NormalTok{(}\AttributeTok{nonRootVolume =}
\NormalTok{                                                         ans}\SpecialCharTok{$}\NormalTok{cumCohortVol,}
                          \AttributeTok{massLiveRoots.fn=}\NormalTok{massLiveRoots.fn,}
                          \AttributeTok{soilLength=}\DecValTok{1}\NormalTok{, }\AttributeTok{soilWidth=}\DecValTok{1}\NormalTok{,}
                          \AttributeTok{relTol =} \FloatTok{1e{-}6}\NormalTok{,}
\NormalTok{                                              ...)}
\NormalTok{  ans}\SpecialCharTok{$}\NormalTok{layer\_top }\OtherTok{\textless{}{-}} \FunctionTok{c}\NormalTok{(}\DecValTok{0}\NormalTok{, ans}\SpecialCharTok{$}\NormalTok{layer\_bottom[}\SpecialCharTok{{-}}\FunctionTok{length}\NormalTok{(ans}\SpecialCharTok{$}\NormalTok{layer\_bottom)])}
  
  \CommentTok{\#recalculate the root mass}
\NormalTok{  ans}\SpecialCharTok{$}\NormalTok{root\_mass }\OtherTok{\textless{}{-}} \FunctionTok{massLiveRoots.fn}\NormalTok{(}\AttributeTok{layerBottom =}\NormalTok{ ans}\SpecialCharTok{$}\NormalTok{layer\_bottom,}
                                    \AttributeTok{layerTop =}\NormalTok{ ans}\SpecialCharTok{$}\NormalTok{layer\_top, ...)}
  
  \FunctionTok{return}\NormalTok{(ans)}
\NormalTok{\}}
\end{Highlighting}
\end{Shaded}

\hypertarget{animatecohorttransect}{%
\subsection{\texorpdfstring{\texttt{animateCohortTransect()}}{animateCohortTransect()}}\label{animatecohorttransect}}

\begin{Shaded}
\begin{Highlighting}[]
\NormalTok{animateCohortTransect }\OtherTok{\textless{}{-}} \ControlFlowTok{function}\NormalTok{(scenarioTransect, cohortsTransect,}
                                  \AttributeTok{duration =} \DecValTok{30}\NormalTok{, }\AttributeTok{width =} \DecValTok{5}\NormalTok{, }\AttributeTok{height =} \DecValTok{3}\NormalTok{,}
                                  \AttributeTok{savePath =} \FunctionTok{getwd}\NormalTok{(),}
                                  \AttributeTok{filename =} \StringTok{"MEM{-}transect{-}animation.gif"}\NormalTok{) \{}
  \FunctionTok{require}\NormalTok{(tidyverse, }\AttributeTok{quietly =} \ConstantTok{TRUE}\NormalTok{)}
  \FunctionTok{require}\NormalTok{(gganimate, }\AttributeTok{quietly =} \ConstantTok{TRUE}\NormalTok{)}
  \FunctionTok{require}\NormalTok{(gifski, }\AttributeTok{quietly =} \ConstantTok{TRUE}\NormalTok{)}
  \FunctionTok{require}\NormalTok{(png, }\AttributeTok{quietly =} \ConstantTok{TRUE}\NormalTok{)}
  
\NormalTok{  surfaceElevations }\OtherTok{\textless{}{-}}\NormalTok{ scenarioTransect }\SpecialCharTok{\%\textgreater{}\%} 
\NormalTok{    dplyr}\SpecialCharTok{::}\FunctionTok{select}\NormalTok{(year, elvMax, elvMin, surfaceElevation) }\CommentTok{\# \%\textgreater{}\% }
    \CommentTok{\# }\AlertTok{TODO}\CommentTok{ rename years as year up stream so we don\textquotesingle{}t have to rename it here}
    \CommentTok{\# dplyr::rename(year = years)}
  
\NormalTok{  animateCohorts }\OtherTok{\textless{}{-}}\NormalTok{ cohortsTransect }\SpecialCharTok{\%\textgreater{}\%} 
\NormalTok{    dplyr}\SpecialCharTok{::}\FunctionTok{mutate}\NormalTok{(}\AttributeTok{fraction\_om =}\NormalTok{ (fast\_OM }\SpecialCharTok{+}\NormalTok{ slow\_OM }\SpecialCharTok{+}\NormalTok{ root\_mass) }\SpecialCharTok{/} 
\NormalTok{                    (fast\_OM }\SpecialCharTok{+}\NormalTok{ slow\_OM }\SpecialCharTok{+}\NormalTok{ root\_mass }\SpecialCharTok{+}\NormalTok{ mineral)) }\SpecialCharTok{\%\textgreater{}\%} 
\NormalTok{    dplyr}\SpecialCharTok{::}\FunctionTok{select}\NormalTok{(year, layer\_top, layer\_bottom, elvMin, elvMax, fraction\_om)}\SpecialCharTok{\%\textgreater{}\%} 
\NormalTok{    dplyr}\SpecialCharTok{::}\FunctionTok{left\_join}\NormalTok{(surfaceElevations, }\AttributeTok{by =} \FunctionTok{c}\NormalTok{(}\StringTok{"year"}\NormalTok{, }\StringTok{"elvMin"}\NormalTok{, }\StringTok{"elvMax"}\NormalTok{)) }\SpecialCharTok{\%\textgreater{}\%} 
\NormalTok{    dplyr}\SpecialCharTok{::}\FunctionTok{mutate}\NormalTok{(}\AttributeTok{layer\_top =}\NormalTok{ surfaceElevation }\SpecialCharTok{{-}}\NormalTok{ layer\_top,}
                  \AttributeTok{layer\_bottom =}\NormalTok{ surfaceElevation }\SpecialCharTok{{-}}\NormalTok{ layer\_bottom)}
  
\NormalTok{  waterLevel }\OtherTok{\textless{}{-}}\NormalTok{ scenarioTransect }\SpecialCharTok{\%\textgreater{}\%} 
\NormalTok{    dplyr}\SpecialCharTok{::}\FunctionTok{select}\NormalTok{(year, meanSeaLevel, meanHighWater) }\SpecialCharTok{\%\textgreater{}\%} 
    \CommentTok{\# dplyr::rename(year = years) \%\textgreater{}\% }
    \FunctionTok{group\_by}\NormalTok{(year) }\SpecialCharTok{\%\textgreater{}\%}
    \FunctionTok{summarise}\NormalTok{(}\AttributeTok{meanSeaLevel=}\FunctionTok{first}\NormalTok{(meanSeaLevel),}
              \AttributeTok{meanHighWater=}\FunctionTok{first}\NormalTok{(meanHighWater)) }\SpecialCharTok{\%\textgreater{}\%} 
    \FunctionTok{gather}\NormalTok{(}\AttributeTok{value=}\StringTok{"waterLevel"}\NormalTok{, }\AttributeTok{key=}\StringTok{"datum"}\NormalTok{, }\SpecialCharTok{{-}}\NormalTok{year) }
  
\NormalTok{  soil\_transect }\OtherTok{\textless{}{-}} \FunctionTok{ggplot}\NormalTok{(}\AttributeTok{data =}\NormalTok{ animateCohorts, }\FunctionTok{aes}\NormalTok{(}\AttributeTok{xmin =}\NormalTok{ elvMin, }
                                                     \AttributeTok{xmax =}\NormalTok{ elvMax,}
                                                     \AttributeTok{ymin =}\NormalTok{ layer\_bottom,}
                                                     \AttributeTok{ymax =}\NormalTok{ layer\_top, }
                                                     \AttributeTok{frame =}\NormalTok{ year}
\NormalTok{  )) }\SpecialCharTok{+}
    \FunctionTok{geom\_rect}\NormalTok{(}\FunctionTok{aes}\NormalTok{(}\AttributeTok{fill =}\NormalTok{ fraction\_om), }\AttributeTok{color =} \FunctionTok{rgb}\NormalTok{(}\DecValTok{0}\NormalTok{,}\DecValTok{0}\NormalTok{,}\DecValTok{0}\NormalTok{, }\AttributeTok{alpha =} \FloatTok{0.1}\NormalTok{)) }\SpecialCharTok{+}
    \FunctionTok{scale\_fill\_gradient2}\NormalTok{(}\AttributeTok{low =} \StringTok{"darkgrey"}\NormalTok{, }\AttributeTok{mid =} \StringTok{"lightgrey"}\NormalTok{,}
                         \AttributeTok{high =} \StringTok{"darkgreen"}\NormalTok{, }\AttributeTok{midpoint =} \FloatTok{0.5}\NormalTok{,}
                         \AttributeTok{name =} \StringTok{"Organic Matter (fraction)"}\NormalTok{) }\SpecialCharTok{+} 
    \FunctionTok{theme\_minimal}\NormalTok{() }\SpecialCharTok{+}
    \FunctionTok{geom\_hline}\NormalTok{(}\AttributeTok{data=}\NormalTok{waterLevel, }\FunctionTok{aes}\NormalTok{(}\AttributeTok{yintercept=}\NormalTok{waterLevel, }\AttributeTok{lty=}\NormalTok{datum),}
               \AttributeTok{color=}\StringTok{"blue"}\NormalTok{) }\SpecialCharTok{+}
    \FunctionTok{ylab}\NormalTok{(}\StringTok{"Depth (cm NAVD88)"}\NormalTok{) }\SpecialCharTok{+}
    \FunctionTok{xlab}\NormalTok{(}\StringTok{"Initial elevation (cm NAVD88)"}\NormalTok{) }\SpecialCharTok{+}
    \FunctionTok{labs}\NormalTok{(}\AttributeTok{title =} \StringTok{\textquotesingle{}Year: \{round(frame\_time,0)\}\textquotesingle{}}\NormalTok{) }\SpecialCharTok{+}
    \FunctionTok{transition\_time}\NormalTok{(year) }\SpecialCharTok{+}
    \FunctionTok{ease\_aes}\NormalTok{(}\StringTok{\textquotesingle{}linear\textquotesingle{}}\NormalTok{)}
  
\NormalTok{tempAnimation}\OtherTok{\textless{}{-}}\NormalTok{gganimate}\SpecialCharTok{::}\FunctionTok{animate}\NormalTok{(soil\_transect, }
                                  \AttributeTok{duration =}\NormalTok{ duration,}
                                  \AttributeTok{nframes=}\FunctionTok{length}\NormalTok{(}\FunctionTok{unique}\NormalTok{(scenarioTransect}\SpecialCharTok{$}\NormalTok{year)),}
                                  \AttributeTok{renderer =} \FunctionTok{gifski\_renderer}\NormalTok{(),}
                                  \AttributeTok{width =}\NormalTok{ width, }
                                  \AttributeTok{height =}\NormalTok{ height, }
                                  \AttributeTok{units =} \StringTok{"in"}\NormalTok{, }
                                  \AttributeTok{res =} \DecValTok{300}\NormalTok{)}
\NormalTok{  (tempAnimation)}
  
\NormalTok{  gganimate}\SpecialCharTok{::}\FunctionTok{anim\_save}\NormalTok{(}\AttributeTok{filename=}\NormalTok{filename,}
                       \AttributeTok{animation=}\NormalTok{tempAnimation,}
                       \AttributeTok{path=}\NormalTok{savePath) }
\NormalTok{\}}
\end{Highlighting}
\end{Shaded}

\hypertarget{buildhightidescenario}{%
\subsection{\texorpdfstring{\texttt{buildHighTideScenario()}}{buildHighTideScenario()}}\label{buildhightidescenario}}

\begin{Shaded}
\begin{Highlighting}[]
\NormalTok{buildHighTideScenario}\OtherTok{\textless{}{-}}\ControlFlowTok{function}\NormalTok{(scenarioCurve, }
                                \AttributeTok{meanSeaLevelDatum=}\NormalTok{scenarioCurve}\SpecialCharTok{$}\NormalTok{meanSeaLevel[}\DecValTok{1}\NormalTok{], }
\NormalTok{                                meanHighWaterDatum, meanHighHighWaterDatum, }
\NormalTok{                                meanHighHighWaterSpringDatum, lunarNodalAmp) \{}
  
  \CommentTok{\# In scenarioCurve object create a meanHighWater and add it to the scenario}
\NormalTok{  scenarioCurve}\SpecialCharTok{$}\NormalTok{meanHighWater}\OtherTok{\textless{}{-}}
    \FunctionTok{predictLunarNodalCycle}\NormalTok{(}\AttributeTok{year =}\NormalTok{scenarioCurve}\SpecialCharTok{$}\NormalTok{years,}
                           \AttributeTok{meanSeaLevel=}\NormalTok{ scenarioCurve}\SpecialCharTok{$}\NormalTok{meanSeaLevel, }
                           \AttributeTok{meanSeaLevelDatum =}\NormalTok{ meanSeaLevelDatum,}
                           \AttributeTok{floodElv=}\NormalTok{meanHighWaterDatum, }
                           \AttributeTok{lunarNodalAmp=}\NormalTok{lunarNodalAmp)}
  
  \CommentTok{\# If meanHighHighWater and meanHighHighWaterSpring are arguments add them to}
  \CommentTok{\# the scenario table too}
  \ControlFlowTok{if}\NormalTok{ (}\SpecialCharTok{!}\FunctionTok{missing}\NormalTok{(meanHighHighWaterDatum)}\SpecialCharTok{\&} \SpecialCharTok{!}\FunctionTok{missing}\NormalTok{(meanHighHighWaterSpringDatum))\{}
\NormalTok{    scenarioCurve}\SpecialCharTok{$}\NormalTok{meanHighHighWater }\OtherTok{\textless{}{-}} 
      \FunctionTok{predictLunarNodalCycle}\NormalTok{(}\AttributeTok{year =}\NormalTok{ scenarioCurve}\SpecialCharTok{$}\NormalTok{years,}
                             \AttributeTok{meanSeaLevel=}\NormalTok{scenarioCurve}\SpecialCharTok{$}\NormalTok{meanSeaLevel,}
                             \AttributeTok{meanSeaLevelDatum =}\NormalTok{ meanSeaLevelDatum,}
                             \AttributeTok{floodElv=}\NormalTok{meanHighHighWaterDatum,}
                             \AttributeTok{lunarNodalAmp=}\NormalTok{lunarNodalAmp)}
    
\NormalTok{    scenarioCurve}\SpecialCharTok{$}\NormalTok{meanHighHighWaterSpring }\OtherTok{\textless{}{-}} 
      \FunctionTok{predictLunarNodalCycle}\NormalTok{(}\AttributeTok{year =}\NormalTok{ scenarioCurve}\SpecialCharTok{$}\NormalTok{years,}
      \AttributeTok{meanSeaLevel=}\NormalTok{ scenarioCurve}\SpecialCharTok{$}\NormalTok{meanSeaLevel,}
      \AttributeTok{meanSeaLevelDatum =}\NormalTok{ meanSeaLevelDatum,}
      \AttributeTok{floodElv=}\NormalTok{meanHighHighWaterSpringDatum, }
      \AttributeTok{lunarNodalAmp=}\NormalTok{lunarNodalAmp)}
\NormalTok{  \}}
  \FunctionTok{return}\NormalTok{(scenarioCurve)}
\NormalTok{\}}
\end{Highlighting}
\end{Shaded}

\hypertarget{buildscenariocurve}{%
\subsection{\texorpdfstring{\texttt{buildScenarioCurve()}}{buildScenarioCurve()}}\label{buildscenariocurve}}

\begin{Shaded}
\begin{Highlighting}[]
\NormalTok{buildScenarioCurve }\OtherTok{\textless{}{-}} \ControlFlowTok{function}\NormalTok{(startYear, endYear, meanSeaLevel, }
\NormalTok{                               relSeaLevelRiseInit, relSeaLevelRiseTotal, }
\NormalTok{                               suspendedSediment) \{}
  
  \CommentTok{\# Create a sequence of the years of the simulation}
\NormalTok{  years }\OtherTok{\textless{}{-}}\NormalTok{ startYear}\SpecialCharTok{:}\NormalTok{endYear}
  \CommentTok{\# Calculate the total number of years of the simulation}
\NormalTok{  nYearsOfSim }\OtherTok{\textless{}{-}} \FunctionTok{length}\NormalTok{(years) }
  
  \CommentTok{\# Create an empty dataframe to hold values of mean sea{-}level and suspended}
  \CommentTok{\# sediment concentration}
\NormalTok{  scenario }\OtherTok{\textless{}{-}} \FunctionTok{data.frame}\NormalTok{(}\AttributeTok{index =} \DecValTok{1}\SpecialCharTok{:}\NormalTok{nYearsOfSim, }\AttributeTok{years =}\NormalTok{ years,}
                         \AttributeTok{meanSeaLevel =} \FunctionTok{rep}\NormalTok{(}\ConstantTok{NA}\NormalTok{, nYearsOfSim),}
                         \AttributeTok{suspendedSediment =} \FunctionTok{rep}\NormalTok{(}\ConstantTok{NA}\NormalTok{, nYearsOfSim))}
  
  \DocumentationTok{\#\# Build the mean sea level scenario}
  \CommentTok{\# If the input only specifies an initial mean sea level at time = 0 ... }
  \ControlFlowTok{if}\NormalTok{ (}\FunctionTok{length}\NormalTok{(meanSeaLevel) }\SpecialCharTok{==} \DecValTok{1}\NormalTok{) \{}
    \CommentTok{\# ... create a sea{-}level rise scenario based on total sea{-}level rise and}
    \CommentTok{\# initial relative sea{-}level rise rate }
\NormalTok{    scenario}\SpecialCharTok{$}\NormalTok{meanSeaLevel[}\DecValTok{1}\NormalTok{] }\OtherTok{\textless{}{-}}\NormalTok{ meanSeaLevel}
\NormalTok{    B }\OtherTok{\textless{}{-}}\NormalTok{ (relSeaLevelRiseTotal}\SpecialCharTok{/}\NormalTok{nYearsOfSim}\SpecialCharTok{{-}}\NormalTok{relSeaLevelRiseInit)}\SpecialCharTok{/}\NormalTok{(nYearsOfSim}\DecValTok{{-}1}\NormalTok{)}
\NormalTok{    A }\OtherTok{\textless{}{-}}\NormalTok{ relSeaLevelRiseInit }\SpecialCharTok{{-}}\NormalTok{ B}
    \CommentTok{\# Calculate the mean sea{-}level for each year of the simulation given A and B}
\NormalTok{    scenario}\SpecialCharTok{$}\NormalTok{meanSeaLevel[}\DecValTok{2}\SpecialCharTok{:}\NormalTok{nYearsOfSim] }\OtherTok{\textless{}{-}}\NormalTok{ scenario}\SpecialCharTok{$}\NormalTok{meanSeaLevel[}\DecValTok{1}\NormalTok{] }\SpecialCharTok{+}
\NormalTok{      A}\SpecialCharTok{*}\NormalTok{scenario}\SpecialCharTok{$}\NormalTok{index[}\DecValTok{2}\SpecialCharTok{:}\NormalTok{nYearsOfSim] }\SpecialCharTok{+}\NormalTok{ B}\SpecialCharTok{*}\NormalTok{scenario}\SpecialCharTok{$}\NormalTok{index[}\DecValTok{2}\SpecialCharTok{:}\NormalTok{nYearsOfSim]}\SpecialCharTok{\^{}}\DecValTok{2}
    \CommentTok{\# If the user enters in a vector of meanSeaLevel that is equal to the number}
    \CommentTok{\# of years in the simulation.}
\NormalTok{  \} }\ControlFlowTok{else} \ControlFlowTok{if}\NormalTok{ (}\FunctionTok{length}\NormalTok{(meanSeaLevel) }\SpecialCharTok{==} \FunctionTok{length}\NormalTok{(years)) \{}
\NormalTok{    scenario}\SpecialCharTok{$}\NormalTok{meanSeaLevel }\OtherTok{\textless{}{-}}\NormalTok{ meanSeaLevel}
\NormalTok{  \} }\ControlFlowTok{else}\NormalTok{ \{}
    \FunctionTok{stop}\NormalTok{(}\StringTok{"RSLR input is incorrect. Either enter a value at for }
\StringTok{         the beginning and ending of the scenario, or a vector}
\StringTok{         of RSLR one for each year of the scenario."}\NormalTok{)}
\NormalTok{  \}}
  
  \CommentTok{\# Add suspended sediment concentration as either a single value...}
  \ControlFlowTok{if}\NormalTok{ (}\FunctionTok{length}\NormalTok{(suspendedSediment) }\SpecialCharTok{==} \DecValTok{1}\NormalTok{) \{}
    \CommentTok{\# If suspendedSediment is a single value.}
\NormalTok{    scenario}\SpecialCharTok{$}\NormalTok{suspendedSediment }\OtherTok{\textless{}{-}} \FunctionTok{rep}\NormalTok{(suspendedSediment, nYearsOfSim)}
    \CommentTok{\# ... or as a vector}
\NormalTok{  \} }\ControlFlowTok{else} \ControlFlowTok{if}\NormalTok{ (}\FunctionTok{length}\NormalTok{(suspendedSediment) }\SpecialCharTok{==}\NormalTok{ nYearsOfSim) \{}
    \CommentTok{\# If suspendedSediment is a vector in equal length to the number of years of}
    \CommentTok{\# the simulation.}
\NormalTok{    scenario}\SpecialCharTok{$}\NormalTok{suspendedSediment }\OtherTok{\textless{}{-}}\NormalTok{ suspendedSediment}
\NormalTok{  \} }\ControlFlowTok{else}\NormalTok{ \{}
    \FunctionTok{stop}\NormalTok{(}\StringTok{"SSC input is incorrect. Either enter a single average value,}
\StringTok{         or a vector equal in length to the number of years in the scenario."}\NormalTok{)}
\NormalTok{  \}}
  \FunctionTok{return}\NormalTok{(scenario)}
\NormalTok{\}}
\end{Highlighting}
\end{Shaded}

\hypertarget{calculatedepthofnonrootvolume}{%
\subsection{\texorpdfstring{\texttt{calculateDepthOfNonRootVolume()}}{calculateDepthOfNonRootVolume()}}\label{calculatedepthofnonrootvolume}}

\begin{Shaded}
\begin{Highlighting}[]
\NormalTok{calculateDepthOfNonRootVolume }\OtherTok{\textless{}{-}} \ControlFlowTok{function}\NormalTok{(nonRootVolume.arr, }
                                 \AttributeTok{massLiveRoots.fn =} \ConstantTok{NULL}\NormalTok{,}
\NormalTok{                                 totalRootMassPerArea, }
\NormalTok{                                 rootDepthMax, }
\NormalTok{                                 rootDensity,}
                                 \AttributeTok{shape =} \StringTok{\textquotesingle{}linear\textquotesingle{}}\NormalTok{,}
                                 \AttributeTok{soilLength =} \DecValTok{1}\NormalTok{, }\AttributeTok{soilWidth =} \DecValTok{1}\NormalTok{,}
                                 \AttributeTok{relTol =} \FloatTok{1e{-}4}\NormalTok{,}
                                 \AttributeTok{verbose =} \ConstantTok{FALSE}\NormalTok{,}
\NormalTok{                                 ...)\{}
  
  \DocumentationTok{\#\#\#\#}
\NormalTok{  totalRootMass }\OtherTok{\textless{}{-}}\NormalTok{ soilLength}\SpecialCharTok{*}\NormalTok{soilWidth}\SpecialCharTok{*}\NormalTok{totalRootMassPerArea}

  \ControlFlowTok{if}\NormalTok{(verbose) }\FunctionTok{print}\NormalTok{(}\FunctionTok{paste}\NormalTok{(}\StringTok{\textquotesingle{}totalRootMass = \textquotesingle{}}\NormalTok{, totalRootMass))}
  
\NormalTok{  totalRootVolume }\OtherTok{\textless{}{-}}\NormalTok{ totalRootMass}\SpecialCharTok{/}\NormalTok{rootDensity}
  \ControlFlowTok{if}\NormalTok{(verbose) }\FunctionTok{print}\NormalTok{(}\FunctionTok{paste}\NormalTok{(}\StringTok{\textquotesingle{}totalRootVolume = \textquotesingle{}}\NormalTok{, totalRootVolume))}
  
  \ControlFlowTok{if}\NormalTok{(totalRootVolume }\SpecialCharTok{\textgreater{}}\NormalTok{ soilLength}\SpecialCharTok{*}\NormalTok{soilWidth}\SpecialCharTok{*}\NormalTok{rootDepthMax)\{}
    \FunctionTok{stop}\NormalTok{(}\StringTok{\textquotesingle{}Bad root volume\textquotesingle{}}\NormalTok{)}
\NormalTok{  \}}
  

  \ControlFlowTok{if}\NormalTok{(totalRootMassPerArea }\SpecialCharTok{==} \DecValTok{0}\NormalTok{)\{}
    \FunctionTok{return}\NormalTok{(nonRootVolume.arr}\SpecialCharTok{*}\NormalTok{soilLength}\SpecialCharTok{*}\NormalTok{soilWidth)}
\NormalTok{  \}}
  
  \DocumentationTok{\#\#If the shape is linear then solve the non{-}root volume depth explicitly }
  \ControlFlowTok{if}\NormalTok{(shape }\SpecialCharTok{==} \StringTok{\textquotesingle{}linear\textquotesingle{}}\NormalTok{)\{}
    \ControlFlowTok{if}\NormalTok{(verbose) }\FunctionTok{print}\NormalTok{(}\StringTok{\textquotesingle{}We are linear.\textquotesingle{}}\NormalTok{)}
\NormalTok{    rootWidth }\OtherTok{\textless{}{-}}\NormalTok{ totalRootVolume}\SpecialCharTok{*}\DecValTok{2}\SpecialCharTok{/}\NormalTok{(rootDepthMax}\SpecialCharTok{*}\NormalTok{soilLength)}
    
    \CommentTok{\#nonRootVolume = ((rootWidth/rootDepthMax*depth\^{}2)/2 +}
    \CommentTok{\#depth*(soilWidth{-}rootWidth))*soilLength 0 = rootWidth / (2*rootDepthMax) *}
    \CommentTok{\#depth \^{}2 + (soilWidth{-}rootWidth) * depth {-} nonRootVolume/soilLength }
    \DocumentationTok{\#\#solve for depth}
\NormalTok{    coef1 }\OtherTok{\textless{}{-}}\NormalTok{ rootWidth }\SpecialCharTok{/}\NormalTok{ (}\DecValTok{2}\SpecialCharTok{*}\NormalTok{rootDepthMax)}
\NormalTok{    coef2 }\OtherTok{\textless{}{-}}\NormalTok{ soilWidth}\SpecialCharTok{{-}}\NormalTok{rootWidth}
\NormalTok{    coef3 }\OtherTok{\textless{}{-}} \SpecialCharTok{{-}}\NormalTok{nonRootVolume.arr}\SpecialCharTok{/}\NormalTok{soilLength}
\NormalTok{    ansDepth }\OtherTok{\textless{}{-}}\NormalTok{ (}\SpecialCharTok{{-}}\NormalTok{coef2 }\SpecialCharTok{+} \FunctionTok{sqrt}\NormalTok{(coef2}\SpecialCharTok{\^{}}\DecValTok{2{-}4}\SpecialCharTok{*}\NormalTok{coef1}\SpecialCharTok{*}\NormalTok{coef3))}\SpecialCharTok{/}\NormalTok{(}\DecValTok{2}\SpecialCharTok{*}\NormalTok{coef1)}
    
    \CommentTok{\#correct for beyond root zone}
\NormalTok{    behondRootZone }\OtherTok{\textless{}{-}}\NormalTok{ nonRootVolume.arr }\SpecialCharTok{\textgreater{}}\NormalTok{ soilLength}\SpecialCharTok{*}\NormalTok{soilWidth}\SpecialCharTok{*}\NormalTok{rootDepthMax }\SpecialCharTok{{-}}
\NormalTok{      totalRootVolume}

    
    \ControlFlowTok{if}\NormalTok{(verbose) }\FunctionTok{print}\NormalTok{(}\FunctionTok{paste0}\NormalTok{(}\StringTok{\textquotesingle{}nonRootVolume.arr: [\textquotesingle{}}\NormalTok{, }\FunctionTok{paste0}\NormalTok{(nonRootVolume.arr,}
                                                            \AttributeTok{collapse =} \StringTok{\textquotesingle{}, \textquotesingle{}}\NormalTok{),}
                             \StringTok{\textquotesingle{}]\textquotesingle{}}\NormalTok{))}
    \ControlFlowTok{if}\NormalTok{(verbose) }\FunctionTok{print}\NormalTok{(}\FunctionTok{paste0}\NormalTok{(}\StringTok{\textquotesingle{}nrv comparison: [\textquotesingle{}}\NormalTok{,}
                             \FunctionTok{paste0}\NormalTok{(soilLength}\SpecialCharTok{*}\NormalTok{soilWidth}\SpecialCharTok{*}\NormalTok{rootDepthMax }\SpecialCharTok{{-}}
\NormalTok{                                      totalRootVolume, }\AttributeTok{collapse =} \StringTok{\textquotesingle{}, \textquotesingle{}}\NormalTok{), }\StringTok{\textquotesingle{}]\textquotesingle{}}\NormalTok{))}
    \ControlFlowTok{if}\NormalTok{(verbose) }\FunctionTok{print}\NormalTok{(}\FunctionTok{paste0}\NormalTok{(}\StringTok{\textquotesingle{}beyondRootZon: [\textquotesingle{}}\NormalTok{, }\FunctionTok{paste0}\NormalTok{(behondRootZone,}
                                                        \AttributeTok{collapse =} \StringTok{\textquotesingle{}, \textquotesingle{}}\NormalTok{), }\StringTok{\textquotesingle{}]\textquotesingle{}}\NormalTok{))}

\NormalTok{    ansDepth[behondRootZone] }\OtherTok{\textless{}{-}}\NormalTok{ (rootDepthMax }\SpecialCharTok{+}
\NormalTok{        (nonRootVolume.arr}\SpecialCharTok{{-}}\NormalTok{soilLength}\SpecialCharTok{*}\NormalTok{soilWidth}\SpecialCharTok{*}\NormalTok{rootDepthMax}\SpecialCharTok{+}\NormalTok{totalRootVolume )}\SpecialCharTok{/}
\NormalTok{                                   (soilLength}\SpecialCharTok{*}\NormalTok{soilWidth)) [behondRootZone] }
    \FunctionTok{return}\NormalTok{(ansDepth)}
    
\NormalTok{  \}}\ControlFlowTok{else}\NormalTok{\{ }\DocumentationTok{\#\#Otherwise apply a general binary search algorithm}
    
    \CommentTok{\#What is the non rooting volumne down to rooting max}
\NormalTok{    nonRootVolumeToRootMax }\OtherTok{\textless{}{-}}\NormalTok{ (soilLength}\SpecialCharTok{*}\NormalTok{soilWidth}\SpecialCharTok{*}\NormalTok{rootDepthMax) }\SpecialCharTok{{-}}
\NormalTok{      totalRootVolume}
    
\NormalTok{    previousDepth }\OtherTok{\textless{}{-}} \DecValTok{0}
\NormalTok{    possibleDepth.arr }\OtherTok{\textless{}{-}} \FunctionTok{rep}\NormalTok{(}\ConstantTok{NA}\NormalTok{, }\AttributeTok{length=}\FunctionTok{length}\NormalTok{(nonRootVolume.arr))}
    
    \ControlFlowTok{for}\NormalTok{(nonRootVolume.index }\ControlFlowTok{in}\NormalTok{ (}\DecValTok{1}\SpecialCharTok{:}\FunctionTok{length}\NormalTok{(nonRootVolume.arr)))\{}
      \ControlFlowTok{if}\NormalTok{(verbose)}\FunctionTok{cat}\NormalTok{(}\FunctionTok{paste}\NormalTok{(}\StringTok{\textquotesingle{}layerIndex: \textquotesingle{}}\NormalTok{, nonRootVolume.index, }\StringTok{\textquotesingle{}{-}{-}{-}{-}}\SpecialCharTok{\textbackslash{}n}\StringTok{\textquotesingle{}}\NormalTok{))}
\NormalTok{      nonRootVolume }\OtherTok{\textless{}{-}}\NormalTok{ nonRootVolume.arr[nonRootVolume.index]}
      \ControlFlowTok{if}\NormalTok{(verbose)}\FunctionTok{cat}\NormalTok{(}\FunctionTok{paste}\NormalTok{(}\StringTok{\textquotesingle{}}\SpecialCharTok{\textbackslash{}t}\StringTok{target nonRootVolume=\textquotesingle{}}\NormalTok{, nonRootVolume, }\StringTok{\textquotesingle{}}\SpecialCharTok{\textbackslash{}n}\StringTok{\textquotesingle{}}\NormalTok{))}
      \ControlFlowTok{if}\NormalTok{(nonRootVolume }\SpecialCharTok{\textless{}}\NormalTok{ relTol)\{}
        \ControlFlowTok{if}\NormalTok{(verbose)}\FunctionTok{cat}\NormalTok{(}\StringTok{\textquotesingle{}no volume}\SpecialCharTok{\textbackslash{}n}\StringTok{\textquotesingle{}}\NormalTok{)}
\NormalTok{        possibleDepth.arr[nonRootVolume] }\OtherTok{\textless{}{-}}\NormalTok{ previousDepth}
        \ControlFlowTok{next}
\NormalTok{      \}}
      \CommentTok{\#If we overfill that root zone volume}
      \ControlFlowTok{if}\NormalTok{(nonRootVolumeToRootMax }\SpecialCharTok{\textless{}=}\NormalTok{ nonRootVolume)\{}
        \ControlFlowTok{if}\NormalTok{(verbose)}\FunctionTok{cat}\NormalTok{(}\StringTok{\textquotesingle{}outside root zone}\SpecialCharTok{\textbackslash{}n}\StringTok{\textquotesingle{}}\NormalTok{)}
        \CommentTok{\#Subtract the root zone volumne, find the depth beyond and add to the}
        \CommentTok{\#max rooting depth}
\NormalTok{        possibleDepth.arr[nonRootVolume.index] }\OtherTok{\textless{}{-}} 
\NormalTok{          (nonRootVolume}\SpecialCharTok{{-}}\NormalTok{nonRootVolumeToRootMax)}\SpecialCharTok{/}\NormalTok{(soilLength}\SpecialCharTok{*}\NormalTok{soilWidth) }\SpecialCharTok{+}
\NormalTok{          rootDepthMax}
\NormalTok{        previousDepth }\OtherTok{\textless{}{-}}\NormalTok{ possibleDepth.arr[nonRootVolume.index]}
        \ControlFlowTok{next}
\NormalTok{      \}}
      
      \CommentTok{\#increment \textless{}{-} (min(rootDepthMax, nonRootVolume/(soilLength*soilWidth)) {-}}
      \CommentTok{\#previousDepth)}
      
      \CommentTok{\#possibleDepth \textless{}{-} min(rootDepthMax, nonRootVolume/(soilLength*soilWidth))}
      \CommentTok{\#{-} incrament / 2}
\NormalTok{      incrament }\OtherTok{\textless{}{-}}\NormalTok{ rootDepthMax }\SpecialCharTok{{-}}\NormalTok{ previousDepth}
\NormalTok{      possibleDepth }\OtherTok{\textless{}{-}}\NormalTok{ previousDepth }\SpecialCharTok{+}\NormalTok{ incrament}\SpecialCharTok{/}\DecValTok{2}
      
\NormalTok{      maxSearchDepth }\OtherTok{\textless{}{-}} \FunctionTok{ceiling}\NormalTok{(}\FunctionTok{log}\NormalTok{(relTol, }\AttributeTok{base=}\FloatTok{0.5}\NormalTok{)) }
      \CommentTok{\#search down to relTol*depth}
      \ControlFlowTok{for}\NormalTok{(ii }\ControlFlowTok{in} \DecValTok{1}\SpecialCharTok{:}\NormalTok{maxSearchDepth)\{}
        \ControlFlowTok{if}\NormalTok{(verbose)}\FunctionTok{cat}\NormalTok{(}\FunctionTok{paste}\NormalTok{(}\StringTok{\textquotesingle{}ii=\textquotesingle{}}\NormalTok{, ii, }\StringTok{\textquotesingle{}}\SpecialCharTok{\textbackslash{}n\textbackslash{}t}\StringTok{incrament=\textquotesingle{}}\NormalTok{, incrament,}
                             \StringTok{\textquotesingle{}}\SpecialCharTok{\textbackslash{}n\textbackslash{}t}\StringTok{possibleDepth=\textquotesingle{}}\NormalTok{, possibleDepth))}
\NormalTok{        rootVolume }\OtherTok{\textless{}{-}}\FunctionTok{massLiveRoots.fn}\NormalTok{(}\AttributeTok{layerTop =} \DecValTok{0}\NormalTok{,}
                                      \AttributeTok{layerBottom =}\NormalTok{ possibleDepth,}
                                      \AttributeTok{totalRootMassPerArea=}\NormalTok{totalRootMassPerArea, }
                                      \AttributeTok{rootDepthMax=}\NormalTok{rootDepthMax, }
                                      \AttributeTok{soilLength=}\NormalTok{soilLength,}
                                      \AttributeTok{soilWidth=}\NormalTok{soilWidth, }\AttributeTok{shape=}\NormalTok{shape }
\NormalTok{        ) }\SpecialCharTok{/}\NormalTok{ rootDensity}
        \ControlFlowTok{if}\NormalTok{(verbose)}\FunctionTok{cat}\NormalTok{(}\FunctionTok{paste}\NormalTok{(}\StringTok{\textquotesingle{}}\SpecialCharTok{\textbackslash{}n\textbackslash{}t}\StringTok{rootVolume\_test=\textquotesingle{}}\NormalTok{, rootVolume))}
\NormalTok{        nonRootVolume\_test}\OtherTok{\textless{}{-}}\NormalTok{(possibleDepth}\SpecialCharTok{*}\NormalTok{soilLength }\SpecialCharTok{*}\NormalTok{ soilWidth) }\SpecialCharTok{{-}}\NormalTok{ rootVolume}
        \ControlFlowTok{if}\NormalTok{(verbose)}\FunctionTok{cat}\NormalTok{(}\FunctionTok{paste}\NormalTok{(}\StringTok{\textquotesingle{}}\SpecialCharTok{\textbackslash{}n\textbackslash{}t}\StringTok{nonRootVolume\_test=\textquotesingle{}}\NormalTok{, nonRootVolume\_test))}
        \ControlFlowTok{if}\NormalTok{( (}\FunctionTok{abs}\NormalTok{(nonRootVolume}\SpecialCharTok{{-}}\NormalTok{nonRootVolume\_test) }\SpecialCharTok{/}\NormalTok{ nonRootVolume) }\SpecialCharTok{\textless{}}\NormalTok{ relTol)\{}
          \ControlFlowTok{if}\NormalTok{(verbose)}\FunctionTok{cat}\NormalTok{(}\StringTok{\textquotesingle{}}\SpecialCharTok{\textbackslash{}n\textbackslash{}t}\StringTok{done}\SpecialCharTok{\textbackslash{}n}\StringTok{\textquotesingle{}}\NormalTok{)}
          \ControlFlowTok{break}
\NormalTok{        \}}\ControlFlowTok{else}\NormalTok{\{}
\NormalTok{          incrament }\OtherTok{\textless{}{-}}\NormalTok{ incrament }\SpecialCharTok{/} \DecValTok{2}
          
          \ControlFlowTok{if}\NormalTok{(verbose)}\FunctionTok{cat}\NormalTok{(}\FunctionTok{paste}\NormalTok{(}\StringTok{\textquotesingle{}}\SpecialCharTok{\textbackslash{}n\textbackslash{}t}\StringTok{relTol=\textquotesingle{}}\NormalTok{,}
                               \FunctionTok{abs}\NormalTok{(nonRootVolume}\SpecialCharTok{{-}}\NormalTok{nonRootVolume\_test) }\SpecialCharTok{/}
\NormalTok{                                 nonRootVolume))}
          \CommentTok{\#Should we go up?}
          \ControlFlowTok{if}\NormalTok{(nonRootVolume }\SpecialCharTok{\textgreater{}}\NormalTok{ nonRootVolume\_test)\{}
            \ControlFlowTok{if}\NormalTok{(verbose)}\FunctionTok{cat}\NormalTok{(}\StringTok{\textquotesingle{}}\SpecialCharTok{\textbackslash{}n\textbackslash{}t}\StringTok{up}\SpecialCharTok{\textbackslash{}n}\StringTok{\textquotesingle{}}\NormalTok{)}
\NormalTok{            possibleDepth }\OtherTok{\textless{}{-}}\NormalTok{ possibleDepth }\SpecialCharTok{+}\NormalTok{ incrament }\SpecialCharTok{/} \DecValTok{2}
\NormalTok{          \}}\ControlFlowTok{else}\NormalTok{\{}
            \ControlFlowTok{if}\NormalTok{(verbose)}\FunctionTok{cat}\NormalTok{(}\StringTok{\textquotesingle{}}\SpecialCharTok{\textbackslash{}n\textbackslash{}t}\StringTok{down}\SpecialCharTok{\textbackslash{}n}\StringTok{\textquotesingle{}}\NormalTok{)}
\NormalTok{            possibleDepth }\OtherTok{\textless{}{-}}\NormalTok{ possibleDepth }\SpecialCharTok{{-}}\NormalTok{ incrament }\SpecialCharTok{/} \DecValTok{2}
            
\NormalTok{          \}}\CommentTok{\#if{-}else up/down}
\NormalTok{        \}}\CommentTok{\#if{-}else relTol}
        
\NormalTok{      \}}\CommentTok{\#for{-}loop refining search}
      
      \ControlFlowTok{if}\NormalTok{(ii }\SpecialCharTok{==}\NormalTok{ maxSearchDepth)\{}
        \ControlFlowTok{if}\NormalTok{(verbose) }
          \FunctionTok{warning}\NormalTok{(}\FunctionTok{paste}\NormalTok{(}\StringTok{\textquotesingle{}Target volumne precision [\textquotesingle{}}\NormalTok{,relTol,}\StringTok{\textquotesingle{}] is not reached\textquotesingle{}}\NormalTok{))}
\NormalTok{      \}}
      
\NormalTok{      previousDepth }\OtherTok{\textless{}{-}}\NormalTok{ possibleDepth}
\NormalTok{      possibleDepth.arr[nonRootVolume.index] }\OtherTok{\textless{}{-}}\NormalTok{ possibleDepth}
\NormalTok{    \}}\CommentTok{\#for{-}loop across array}
    
    \FunctionTok{return}\NormalTok{(possibleDepth.arr)}
\NormalTok{  \}}\CommentTok{\#if checking for unknown shape}
  
\NormalTok{\}}
\end{Highlighting}
\end{Shaded}

\hypertarget{convertprofileagetodepth}{%
\subsection{\texorpdfstring{\texttt{convertProfileAgeToDepth()}}{convertProfileAgeToDepth()}}\label{convertprofileagetodepth}}

\begin{Shaded}
\begin{Highlighting}[]
\NormalTok{convertProfileAgeToDepth }\OtherTok{\textless{}{-}} \ControlFlowTok{function}\NormalTok{(ageCohort, layerTop, layerBottom)\{}
  
\NormalTok{  ans }\OtherTok{\textless{}{-}}\NormalTok{ plyr}\SpecialCharTok{::}\FunctionTok{ddply}\NormalTok{(}\FunctionTok{data.frame}\NormalTok{(}\AttributeTok{bottom=}\NormalTok{layerBottom, }\AttributeTok{top=}\NormalTok{layerTop),}
                     \FunctionTok{c}\NormalTok{(}\StringTok{\textquotesingle{}top\textquotesingle{}}\NormalTok{, }\StringTok{\textquotesingle{}bottom\textquotesingle{}}\NormalTok{), }\ControlFlowTok{function}\NormalTok{(xx)\{}
    
\NormalTok{    layerWeights }\OtherTok{\textless{}{-}} \FunctionTok{pmax}\NormalTok{( }\FunctionTok{pmin}\NormalTok{(ageCohort}\SpecialCharTok{$}\NormalTok{layer\_bottom, xx}\SpecialCharTok{$}\NormalTok{bottom) }\SpecialCharTok{{-}}
                            \FunctionTok{pmax}\NormalTok{(ageCohort}\SpecialCharTok{$}\NormalTok{layer\_top, xx}\SpecialCharTok{$}\NormalTok{top), }
                          \DecValTok{0}\NormalTok{) }\SpecialCharTok{/}\NormalTok{ (ageCohort}\SpecialCharTok{$}\NormalTok{layer\_bottom }\SpecialCharTok{{-}}\NormalTok{ ageCohort}\SpecialCharTok{$}\NormalTok{layer\_top)}
    
\NormalTok{    ans }\OtherTok{\textless{}{-}}\NormalTok{ base}\SpecialCharTok{::}\FunctionTok{lapply}\NormalTok{(ageCohort[,}\FunctionTok{c}\NormalTok{(}\StringTok{\textquotesingle{}fast\_OM\textquotesingle{}}\NormalTok{, }\StringTok{\textquotesingle{}slow\_OM\textquotesingle{}}\NormalTok{,}
                                     \StringTok{\textquotesingle{}mineral\textquotesingle{}}\NormalTok{, }\StringTok{\textquotesingle{}root\_mass\textquotesingle{}}\NormalTok{)],}
                        \ControlFlowTok{function}\NormalTok{(yy)}\FunctionTok{sum}\NormalTok{(yy}\SpecialCharTok{*}\NormalTok{layerWeights))}
    
\NormalTok{    ans}\SpecialCharTok{$}\NormalTok{age }\OtherTok{\textless{}{-}} \FunctionTok{weighted.mean}\NormalTok{(ageCohort}\SpecialCharTok{$}\NormalTok{age, layerWeights)}
\NormalTok{    ans}\SpecialCharTok{$}\NormalTok{input\_yrs }\OtherTok{\textless{}{-}} \FunctionTok{sum}\NormalTok{(layerWeights)}
\NormalTok{    ans}\SpecialCharTok{$}\NormalTok{layer\_bottom }\OtherTok{\textless{}{-}}\NormalTok{ xx}\SpecialCharTok{$}\NormalTok{bottom}
\NormalTok{    ans}\SpecialCharTok{$}\NormalTok{layer\_top }\OtherTok{\textless{}{-}}\NormalTok{ xx}\SpecialCharTok{$}\NormalTok{top}
    
    \FunctionTok{return}\NormalTok{(}\FunctionTok{as.data.frame}\NormalTok{(ans))}
\NormalTok{  \})}
  
  \FunctionTok{return}\NormalTok{(ans[,}\FunctionTok{c}\NormalTok{(}\StringTok{\textquotesingle{}layer\_top\textquotesingle{}}\NormalTok{, }\StringTok{\textquotesingle{}layer\_bottom\textquotesingle{}}\NormalTok{, }\StringTok{\textquotesingle{}age\textquotesingle{}}\NormalTok{, }\StringTok{\textquotesingle{}input\_yrs\textquotesingle{}}\NormalTok{,}
                \StringTok{\textquotesingle{}fast\_OM\textquotesingle{}}\NormalTok{, }\StringTok{\textquotesingle{}slow\_OM\textquotesingle{}}\NormalTok{, }\StringTok{\textquotesingle{}mineral\textquotesingle{}}\NormalTok{, }\StringTok{\textquotesingle{}root\_mass\textquotesingle{}}\NormalTok{)])}
\NormalTok{\}}
\end{Highlighting}
\end{Shaded}

\hypertarget{convertzstartoz}{%
\subsection{\texorpdfstring{\texttt{convertZStarToZ()}}{convertZStarToZ()}}\label{convertzstartoz}}

\begin{Shaded}
\begin{Highlighting}[]
\NormalTok{convertZStarToZ }\OtherTok{\textless{}{-}} \ControlFlowTok{function}\NormalTok{(zStar, meanHighWater, meanSeaLevel) \{ }
\NormalTok{  (zStar }\SpecialCharTok{*}\NormalTok{ ((meanHighWater}\SpecialCharTok{{-}}\NormalTok{meanSeaLevel))) }\SpecialCharTok{+}\NormalTok{ meanSeaLevel }
\NormalTok{  \}}
\end{Highlighting}
\end{Shaded}

\hypertarget{convertztozstar}{%
\subsection{\texorpdfstring{\texttt{convertZToZstar()}}{convertZToZstar()}}\label{convertztozstar}}

\begin{Shaded}
\begin{Highlighting}[]
\NormalTok{convertZToZstar }\OtherTok{\textless{}{-}} \ControlFlowTok{function}\NormalTok{(z, meanHighWater, meanSeaLevel) \{ }
\NormalTok{  (z}\SpecialCharTok{{-}}\NormalTok{meanSeaLevel)}\SpecialCharTok{/}\NormalTok{(meanHighWater}\SpecialCharTok{{-}}\NormalTok{meanSeaLevel) }
\NormalTok{\}}
\end{Highlighting}
\end{Shaded}

\hypertarget{deliversediment}{%
\subsection{\texorpdfstring{\texttt{deliverSediment()}}{deliverSediment()}}\label{deliversediment}}

\begin{Shaded}
\begin{Highlighting}[]
\NormalTok{deliverSediment }\OtherTok{\textless{}{-}} \ControlFlowTok{function}\NormalTok{(z, suspendedSediment, }\AttributeTok{nFloods =} \FloatTok{705.79}\NormalTok{,}
\NormalTok{                            meanSeaLevel, meanHighWater, }\AttributeTok{meanHighHighWater=}\ConstantTok{NA}\NormalTok{,}
                            \AttributeTok{meanHighHighWaterSpring=}\ConstantTok{NA}\NormalTok{,}
                            \AttributeTok{meanLowWater=}\NormalTok{meanSeaLevel}\SpecialCharTok{{-}}\NormalTok{meanHighWater, }
                            \AttributeTok{meanLowLowWater=}\NormalTok{meanSeaLevel}\SpecialCharTok{{-}}\NormalTok{meanHighHighWater, }
                            \AttributeTok{meanLowLowWaterSpring=}
\NormalTok{                              meanSeaLevel}\SpecialCharTok{{-}}\NormalTok{meanHighHighWaterSpring,}
\NormalTok{                            captureRate,}
\NormalTok{                            floodTime.fn) \{}


  \CommentTok{\# If all three tidal datums are present}
  \ControlFlowTok{if}\NormalTok{ (}\FunctionTok{all}\NormalTok{(}\SpecialCharTok{!}\FunctionTok{is.na}\NormalTok{(}\FunctionTok{c}\NormalTok{(meanHighWater,meanHighHighWater,meanHighHighWaterSpring)))) \{}
    \CommentTok{\# Create a data frame operation so we can use tidy functions to speed up}
\NormalTok{    tidalCycles }\OtherTok{\textless{}{-}} \FunctionTok{data.frame}\NormalTok{(}\AttributeTok{datumHigh =} \FunctionTok{c}\NormalTok{(meanHighWater, meanHighHighWater,}
\NormalTok{                                            meanHighHighWaterSpring), }
                              \AttributeTok{datumLow =} \FunctionTok{c}\NormalTok{(meanLowWater, meanLowLowWater,}
\NormalTok{                                           meanLowLowWaterSpring),}
                              \AttributeTok{nTides =} \FunctionTok{c}\NormalTok{(}\FloatTok{0.5}\NormalTok{, }\FloatTok{0.46497542}\NormalTok{, }\FloatTok{0.03502458}\NormalTok{)) }
\NormalTok{  \} }\ControlFlowTok{else} \ControlFlowTok{if}\NormalTok{ (}\FunctionTok{all}\NormalTok{(}\SpecialCharTok{!} \FunctionTok{is.na}\NormalTok{(}\FunctionTok{c}\NormalTok{(meanHighWater, meanHighHighWater)))) \{}
    \CommentTok{\# If only MHW and MHHW are present}
\NormalTok{    tidalCycles }\OtherTok{\textless{}{-}} \FunctionTok{data.frame}\NormalTok{(}\AttributeTok{datumHigh =} \FunctionTok{c}\NormalTok{(meanHighWater, meanHighHighWater), }
                              \AttributeTok{datumLow =} \FunctionTok{c}\NormalTok{(meanLowWater, meanLowLowWater),}
                              \AttributeTok{nTides =} \FunctionTok{c}\NormalTok{(}\FloatTok{0.5}\NormalTok{, }\FloatTok{0.5}\NormalTok{)) }
\NormalTok{  \} }\ControlFlowTok{else}\NormalTok{ \{}
    \CommentTok{\# If only MHW is present}
\NormalTok{    tidalCycles }\OtherTok{\textless{}{-}} \FunctionTok{data.frame}\NormalTok{(}\AttributeTok{datumHigh =} \FunctionTok{c}\NormalTok{(meanHighWater), }
                              \AttributeTok{datumLow =} \FunctionTok{c}\NormalTok{(meanLowWater),}
                              \AttributeTok{nTides =} \FunctionTok{c}\NormalTok{(}\DecValTok{1}\NormalTok{)) }
\NormalTok{  \}}
  
  \CommentTok{\# Convert fraction of tides to counts of tides based on input}
\NormalTok{  tidalCycles}\SpecialCharTok{$}\NormalTok{nTides }\OtherTok{\textless{}{-}}\NormalTok{ tidalCycles}\SpecialCharTok{$}\NormalTok{nTides }\SpecialCharTok{*}\NormalTok{ nFloods}
  
\NormalTok{  tidalCycles }\OtherTok{\textless{}{-}}\NormalTok{ tidalCycles }\SpecialCharTok{\%\textgreater{}\%}
    \CommentTok{\# Set tidal properties to 0 if surface is above tidal range in each case}
\NormalTok{    dplyr}\SpecialCharTok{::}\FunctionTok{mutate}\NormalTok{(}\AttributeTok{nTides =} \FunctionTok{ifelse}\NormalTok{(z}\SpecialCharTok{\textgreater{}}\NormalTok{datumHigh, }\DecValTok{0}\NormalTok{, nTides), }\CommentTok{\# number of tides}
                  \CommentTok{\# Tidal height relative to surface}
                  \AttributeTok{tidalHeight =} \FunctionTok{ifelse}\NormalTok{(z}\SpecialCharTok{\textgreater{}}\NormalTok{datumHigh, }\DecValTok{0}\NormalTok{, (datumHigh}\SpecialCharTok{{-}}\NormalTok{z)}\SpecialCharTok{*}\FloatTok{0.5}\NormalTok{), }
                  \CommentTok{\# Call flood time function.}
                  \AttributeTok{floodTime =} \FunctionTok{floodTime.fn}\NormalTok{(}\AttributeTok{z=}\NormalTok{z, }
                              \AttributeTok{datumHigh=}\NormalTok{datumHigh, }
                              \AttributeTok{datumLow=}\NormalTok{datumLow),}
                  \CommentTok{\# Calculate fraction of sediment captured}
                        \CommentTok{\# if the sediment column IS NOT able to clear}
                  \AttributeTok{fractionCaptured =} \FunctionTok{ifelse}\NormalTok{(floodTime }\SpecialCharTok{\textless{}} \DecValTok{1}\SpecialCharTok{/}\NormalTok{captureRate, }
                        \CommentTok{\# available suspendedSediment is total possible capture }
\NormalTok{                                             captureRate}\SpecialCharTok{*}\NormalTok{floodTime, }
                                      \CommentTok{\# if the sediment column IS able to clear}
                                             \DecValTok{1}\NormalTok{),}
                  \CommentTok{\# Calculate available sediment as a cumulative block of water}
                  \AttributeTok{availableSediment =}\NormalTok{ suspendedSediment }\SpecialCharTok{*}\NormalTok{ nTides }\SpecialCharTok{*}\NormalTok{ tidalHeight,}
                  \CommentTok{\# Calculated delivered sediment}
                  \AttributeTok{deliveredSediment =}\NormalTok{ availableSediment }\SpecialCharTok{*}\NormalTok{ fractionCaptured) }
  
  \CommentTok{\# Sum delivered sediment across tidal cycles}
\NormalTok{  totalDeliveredSediment }\OtherTok{\textless{}{-}} \FunctionTok{sum}\NormalTok{(tidalCycles}\SpecialCharTok{$}\NormalTok{deliveredSediment) }
  
  \FunctionTok{return}\NormalTok{(totalDeliveredSediment)}
\NormalTok{\}}
\end{Highlighting}
\end{Shaded}

\hypertarget{determineinitialcohorts}{%
\subsection{\texorpdfstring{\texttt{determineInitialCohorts()}}{determineInitialCohorts()}}\label{determineinitialcohorts}}

\begin{Shaded}
\begin{Highlighting}[]
\NormalTok{determineInitialCohorts }\OtherTok{\textless{}{-}} \ControlFlowTok{function}\NormalTok{(initElv,}
\NormalTok{                                 meanSeaLevel, meanHighWater,}
                                 \AttributeTok{meanHighHighWater=}\ConstantTok{NA}\NormalTok{, }
                                 \AttributeTok{meanHighHighWaterSpring=}\ConstantTok{NA}\NormalTok{, }
\NormalTok{                                 suspendedSediment,}
                                 \AttributeTok{nFloods =} \FloatTok{705.79}\NormalTok{,}
                                 \AttributeTok{floodTime.fn =}\NormalTok{ floodTimeLinear,}
\NormalTok{                                 bMax, zVegMin, zVegMax, zVegPeak,}
\NormalTok{                                 plantElevationType,}
\NormalTok{                                 rootToShoot, rootTurnover, rootDepthMax,}
                                 \AttributeTok{shape=}\StringTok{"linear"}\NormalTok{,}
                                 \AttributeTok{abovegroundTurnover=}\ConstantTok{NA}\NormalTok{, }\AttributeTok{speciesCode=}\ConstantTok{NA}\NormalTok{,}
\NormalTok{                                 omDecayRate, recalcitrantFrac, captureRate,}
                                 \AttributeTok{omPackingDensity=}\FloatTok{0.085}\NormalTok{,}
                                 \AttributeTok{mineralPackingDensity=}\FloatTok{1.99}\NormalTok{,}
                                 \AttributeTok{rootPackingDensity=}\NormalTok{omPackingDensity,}
                                 \AttributeTok{initialCohorts=}\ConstantTok{NA}\NormalTok{,}
                                 \AttributeTok{uplandCohorts=}\ConstantTok{NA}\NormalTok{,}
                                 \AttributeTok{supertidalCohorts=}\ConstantTok{NA}\NormalTok{,}
                                 \AttributeTok{supertidalSedimentInput=}\ConstantTok{NA}\NormalTok{,}
\NormalTok{                                 ...) \{}
  
  \CommentTok{\# If initialCohorts is defined as an input then it overrides all other}
  \CommentTok{\# arguments.}
  \ControlFlowTok{if}\NormalTok{ (}\FunctionTok{is.data.frame}\NormalTok{(initialCohorts)) \{}
    \CommentTok{\# If it does,}
    \CommentTok{\# Return initial cohorts}
\NormalTok{    cohorts }\OtherTok{\textless{}{-}}\NormalTok{ initialCohorts}
\NormalTok{    bio\_table }\OtherTok{\textless{}{-}} \FunctionTok{data.frame}\NormalTok{(}\AttributeTok{speciesCode=}\ConstantTok{NA}\NormalTok{,}
                            \AttributeTok{rootToShoot=}\ConstantTok{NA}\NormalTok{,}
                            \AttributeTok{rootTurnover=}\ConstantTok{NA}\NormalTok{,}
                            \AttributeTok{abovegroundTurnover=}\ConstantTok{NA}\NormalTok{,}
                            \AttributeTok{rootDepthMax=}\ConstantTok{NA}\NormalTok{,}
                            \AttributeTok{aboveground\_biomass=}\ConstantTok{NA}\NormalTok{,}
                            \AttributeTok{belowground\_biomass=}\ConstantTok{NA}\NormalTok{)}
\NormalTok{    initSediment }\OtherTok{\textless{}{-}} \ConstantTok{NA}
\NormalTok{  \} }\ControlFlowTok{else}\NormalTok{ \{ }\CommentTok{\# If initial cohorts are not supplied}
    
    \CommentTok{\# Convert real growing elevations to dimensionless growing elevations}
    \ControlFlowTok{if}\NormalTok{ (}\SpecialCharTok{!}\NormalTok{ plantElevationType }\SpecialCharTok{\%in\%} \FunctionTok{c}\NormalTok{(}\StringTok{"dimensionless"}\NormalTok{, }\StringTok{"zStar"}\NormalTok{, }\StringTok{"Z*"}\NormalTok{, }\StringTok{"zstar"}\NormalTok{)) \{}
\NormalTok{      zStarVegMin }\OtherTok{\textless{}{-}} \FunctionTok{convertZToZstar}\NormalTok{(zVegMin, meanHighWater, meanSeaLevel)}
\NormalTok{      zStarVegMax }\OtherTok{\textless{}{-}} \FunctionTok{convertZToZstar}\NormalTok{(zVegMax, meanHighWater, meanSeaLevel)}
\NormalTok{      zStarVegPeak }\OtherTok{\textless{}{-}} \FunctionTok{convertZToZstar}\NormalTok{(zVegPeak, meanHighWater, meanSeaLevel)}
\NormalTok{    \} }\ControlFlowTok{else}\NormalTok{ \{}
\NormalTok{      zStarVegMin }\OtherTok{\textless{}{-}}\NormalTok{ zVegMin}
\NormalTok{      zStarVegMax }\OtherTok{\textless{}{-}}\NormalTok{ zVegMax}
\NormalTok{      zStarVegPeak }\OtherTok{\textless{}{-}}\NormalTok{ zVegPeak}
\NormalTok{    \}}
    
    \CommentTok{\# Convert dimensionless plant growing elevations to real growing elevations}
    \ControlFlowTok{if}\NormalTok{ (plantElevationType }\SpecialCharTok{\%in\%} \FunctionTok{c}\NormalTok{(}\StringTok{"dimensionless"}\NormalTok{, }\StringTok{"zStar"}\NormalTok{, }\StringTok{"Z*"}\NormalTok{, }\StringTok{"zstar"}\NormalTok{)) \{}
\NormalTok{      zVegMin }\OtherTok{\textless{}{-}} \FunctionTok{convertZStarToZ}\NormalTok{(zVegMin, meanHighWater, meanSeaLevel)}
\NormalTok{      zVegMax }\OtherTok{\textless{}{-}} \FunctionTok{convertZStarToZ}\NormalTok{(zVegMax, meanHighWater, meanSeaLevel)}
\NormalTok{      zVegPeak }\OtherTok{\textless{}{-}} \FunctionTok{convertZStarToZ}\NormalTok{(zVegPeak, meanHighWater, meanSeaLevel)}
\NormalTok{    \}}
    
    \CommentTok{\# Set initial conditions}
    \CommentTok{\# Calculate initial z star}
\NormalTok{    initElvStar }\OtherTok{\textless{}{-}} \FunctionTok{convertZToZstar}\NormalTok{(}\AttributeTok{z=}\NormalTok{initElv, }\AttributeTok{meanSeaLevel=}\NormalTok{meanSeaLevel,}
                                   \AttributeTok{meanHighWater=}\NormalTok{meanHighWater)}
    
    \CommentTok{\# Initial Above Ground Biomass}
\NormalTok{    bio\_table }\OtherTok{\textless{}{-}} \FunctionTok{runMultiSpeciesBiomass}\NormalTok{(initElvStar, }\AttributeTok{bMax =}\NormalTok{ bMax,}
                                        \AttributeTok{zVegMax =}\NormalTok{ zStarVegMax, }
                                        \AttributeTok{zVegMin =}\NormalTok{ zStarVegMin,}
                                        \AttributeTok{zVegPeak =}\NormalTok{ zStarVegPeak,}
                                        \AttributeTok{rootToShoot=}\NormalTok{rootToShoot,}
                                        \AttributeTok{rootTurnover=}\NormalTok{rootTurnover, }
                                        \AttributeTok{abovegroundTurnover=}\NormalTok{abovegroundTurnover, }
                                        \AttributeTok{rootDepthMax=}\NormalTok{rootDepthMax,}
                                        \AttributeTok{speciesCode=}\NormalTok{speciesCode)}
    
    \CommentTok{\# If elevation is lower than highest tide provided, and lower than maximum}
    \CommentTok{\# growing elevation Generate 1 m or more of sediment given equilibrium}
    \CommentTok{\# conditions}
    \ControlFlowTok{if}\NormalTok{ ((initElv }\SpecialCharTok{\textless{}=} \FunctionTok{max}\NormalTok{(meanHighWater, meanHighHighWater,}
\NormalTok{                        meanHighHighWaterSpring, }\AttributeTok{na.rm=}\NormalTok{T)) }\SpecialCharTok{\&}
\NormalTok{        (initElv }\SpecialCharTok{\textless{}=} \FunctionTok{max}\NormalTok{(zVegMax))) \{}

      \CommentTok{\# Initial Sediment}
\NormalTok{      initSediment }\OtherTok{\textless{}{-}} \FunctionTok{deliverSediment}\NormalTok{(}\AttributeTok{z=}\NormalTok{initElv, }
                                      \AttributeTok{suspendedSediment=}\NormalTok{suspendedSediment,}
                                      \AttributeTok{nFloods=}\NormalTok{nFloods,}
                                      \AttributeTok{meanSeaLevel=}\NormalTok{meanSeaLevel,}
                                      \AttributeTok{meanHighWater=}\NormalTok{meanHighWater,}
                                      \AttributeTok{meanHighHighWater=}\NormalTok{meanHighHighWater,}
                                      \AttributeTok{meanHighHighWaterSpring=}
\NormalTok{                                        meanHighHighWaterSpring,}
                                      \AttributeTok{captureRate=}\NormalTok{captureRate,}
                                      \AttributeTok{floodTime.fn=}\NormalTok{floodTime.fn)}
      
      \CommentTok{\# Run initial conditions to equilibrium}
\NormalTok{      cohorts }\OtherTok{\textless{}{-}} \FunctionTok{runToEquilibrium}\NormalTok{(}\AttributeTok{totalRootMassPerArea=}
\NormalTok{                                    bio\_table}\SpecialCharTok{$}\NormalTok{belowground\_biomass[}\DecValTok{1}\NormalTok{], }
                                  \AttributeTok{rootDepthMax=}\NormalTok{bio\_table}\SpecialCharTok{$}\NormalTok{rootDepthMax[}\DecValTok{1}\NormalTok{],}
                                  \AttributeTok{rootTurnover=}\NormalTok{bio\_table}\SpecialCharTok{$}\NormalTok{rootTurnover,}
                                  \AttributeTok{omDecayRate =} \FunctionTok{list}\NormalTok{(}\AttributeTok{fast=}\NormalTok{omDecayRate, }\AttributeTok{slow=}\DecValTok{0}\NormalTok{),}
                                  \AttributeTok{rootOmFrac=}\FunctionTok{list}\NormalTok{(}\AttributeTok{fast=}\DecValTok{1}\SpecialCharTok{{-}}\NormalTok{recalcitrantFrac,}
                                                  \AttributeTok{slow=}\NormalTok{recalcitrantFrac),}
                                  \AttributeTok{packing=}\FunctionTok{list}\NormalTok{(}\AttributeTok{organic=}\NormalTok{omPackingDensity,}
                                               \AttributeTok{mineral=}\NormalTok{mineralPackingDensity),}
                                  \AttributeTok{rootDensity=}\NormalTok{rootPackingDensity, }\AttributeTok{shape=}\NormalTok{shape, }
                                  \AttributeTok{mineralInput =}\NormalTok{ initSediment,}
                                  \AttributeTok{minDepth =} \FunctionTok{round}\NormalTok{(}\FunctionTok{max}\NormalTok{(rootDepthMax)}\SpecialCharTok{+}\FloatTok{0.5}\NormalTok{))}
      
\NormalTok{    \} }\ControlFlowTok{else} \ControlFlowTok{if}\NormalTok{ ((initElv }\SpecialCharTok{\textgreater{}=} \FunctionTok{max}\NormalTok{(meanHighWater, meanHighHighWater,}
\NormalTok{                               meanHighHighWaterSpring, }\AttributeTok{na.rm=}\NormalTok{T)) }\SpecialCharTok{\&} 
\NormalTok{               (initElv }\SpecialCharTok{\textless{}=} \FunctionTok{max}\NormalTok{(zVegMax))) \{ }
      \CommentTok{\# If elevation is greater than highest tide provided, but lower than}
      \CommentTok{\# maximum growing elevation Then form super{-}tidal peat}
      
      \CommentTok{\# Check to see if supertidal peat is defined as an input}
      \ControlFlowTok{if}\NormalTok{ (}\FunctionTok{is.data.frame}\NormalTok{(supertidalCohorts)) \{}
        \CommentTok{\# If it is, than pass the input staight to the output}
\NormalTok{        cohorts }\OtherTok{\textless{}{-}}\NormalTok{ supertidalCohorts}
\NormalTok{        bio\_table }\OtherTok{\textless{}{-}} \FunctionTok{data.frame}\NormalTok{(}\AttributeTok{speciesCode=}\ConstantTok{NA}\NormalTok{,}
                                \AttributeTok{rootToShoot=}\ConstantTok{NA}\NormalTok{,}
                                \AttributeTok{rootTurnover=}\ConstantTok{NA}\NormalTok{,}
                                \AttributeTok{abovegroundTurnover=}\ConstantTok{NA}\NormalTok{,}
                                \AttributeTok{rootDepthMax=}\ConstantTok{NA}\NormalTok{,}
                                \AttributeTok{aboveground\_biomass=}\ConstantTok{NA}\NormalTok{,}
                                \AttributeTok{belowground\_biomass=}\ConstantTok{NA}\NormalTok{)}
\NormalTok{                                initSediment }\OtherTok{\textless{}{-}} \ConstantTok{NA}
\NormalTok{      \} }\ControlFlowTok{else}\NormalTok{ \{}
        \CommentTok{\# If supertidalSedimentInput is defined}
        \ControlFlowTok{if}\NormalTok{ (}\FunctionTok{is.data.frame}\NormalTok{(supertidalSedimentInput)) \{}
          \CommentTok{\# Run initial conditions to equilibrium}
\NormalTok{          initSediment }\OtherTok{\textless{}{-}}\NormalTok{ supertidalSedimentInput}
\NormalTok{          cohorts }\OtherTok{\textless{}{-}} \FunctionTok{runToEquilibrium}\NormalTok{(}\AttributeTok{totalRootMassPerArea=}
\NormalTok{                              bio\_table}\SpecialCharTok{$}\NormalTok{belowground\_biomass[}\DecValTok{1}\NormalTok{],}
                              \AttributeTok{rootDepthMax=}\NormalTok{bio\_table}\SpecialCharTok{$}\NormalTok{rootDepthMax[}\DecValTok{1}\NormalTok{],}
                              \AttributeTok{rootTurnover=}\NormalTok{bio\_table}\SpecialCharTok{$}\NormalTok{rootTurnover,}
                              \AttributeTok{omDecayRate =} \FunctionTok{list}\NormalTok{(}\AttributeTok{fast=}\NormalTok{omDecayRate,}\AttributeTok{slow=}\DecValTok{0}\NormalTok{),}
                              \AttributeTok{rootOmFrac=}\FunctionTok{list}\NormalTok{(}\AttributeTok{fast=}\DecValTok{1}\SpecialCharTok{{-}}\NormalTok{recalcitrantFrac,}
                                              \AttributeTok{slow=}\NormalTok{recalcitrantFrac),}
                              \AttributeTok{packing=}\FunctionTok{list}\NormalTok{(}\AttributeTok{organic=}\NormalTok{omPackingDensity,}
                                           \AttributeTok{mineral=}\NormalTok{mineralPackingDensity),}
                              \AttributeTok{rootDensity=}\NormalTok{rootPackingDensity, }\AttributeTok{shape=}\NormalTok{shape, }
                              \AttributeTok{mineralInput =}\NormalTok{ initSediment,}
                              \AttributeTok{minDepth =} \FunctionTok{round}\NormalTok{(}\FunctionTok{max}\NormalTok{(rootDepthMax)}\SpecialCharTok{+}\FloatTok{0.5}\NormalTok{))}
\NormalTok{        \} }\ControlFlowTok{else}\NormalTok{ \{}
          \CommentTok{\# If not, come up with a set with a column of of peat generated with}
          \CommentTok{\# biomass inputs, and any assumed 0 sediment input.}
          
          \CommentTok{\# Initial Sediment, an arbitrary low number. Here I use the annual}
          \CommentTok{\# sediment delivered 1 cm below the highest tide line}
\NormalTok{          initSediment }\OtherTok{\textless{}{-}} \FunctionTok{deliverSediment}\NormalTok{(}\AttributeTok{z=}\FunctionTok{max}\NormalTok{(meanHighWater,}
\NormalTok{                                                meanHighHighWater,}
\NormalTok{                                                meanHighHighWaterSpring,}
                                                \AttributeTok{na.rm=}\NormalTok{T)}\SpecialCharTok{{-}}\DecValTok{1}\NormalTok{, }
                                          \AttributeTok{suspendedSediment=}\NormalTok{suspendedSediment, }
                                          \AttributeTok{meanSeaLevel=}\NormalTok{meanSeaLevel, }
                                          \AttributeTok{meanHighWater=}\NormalTok{meanHighWater, }
                                          \AttributeTok{meanHighHighWater=}\NormalTok{meanHighHighWater, }
                                          \AttributeTok{meanHighHighWaterSpring=}
\NormalTok{                                            meanHighHighWaterSpring,}
                                          \AttributeTok{nFloods =}\NormalTok{ nFloods,}
                                          \AttributeTok{captureRate=}\NormalTok{captureRate,}
                                          \AttributeTok{floodTime.fn =}\NormalTok{ floodTime.fn)}
          
\NormalTok{          cohorts }\OtherTok{\textless{}{-}} \FunctionTok{runToEquilibrium}\NormalTok{(}\AttributeTok{totalRootMassPerArea=}
\NormalTok{                              bio\_table}\SpecialCharTok{$}\NormalTok{belowground\_biomass[}\DecValTok{1}\NormalTok{],}
                              \AttributeTok{rootDepthMax=}\NormalTok{bio\_table}\SpecialCharTok{$}\NormalTok{rootDepthMax[}\DecValTok{1}\NormalTok{],}
                              \AttributeTok{rootTurnover=}\NormalTok{bio\_table}\SpecialCharTok{$}\NormalTok{rootTurnover,}
                              \AttributeTok{omDecayRate =} \FunctionTok{list}\NormalTok{(}\AttributeTok{fast=}\NormalTok{omDecayRate, }\AttributeTok{slow=}\DecValTok{0}\NormalTok{),}
                              \AttributeTok{rootOmFrac=}\FunctionTok{list}\NormalTok{(}\AttributeTok{fast=}\DecValTok{1}\SpecialCharTok{{-}}\NormalTok{recalcitrantFrac, }
                                              \AttributeTok{slow=}\NormalTok{recalcitrantFrac),}
                              \AttributeTok{packing=}\FunctionTok{list}\NormalTok{(}\AttributeTok{organic=}\NormalTok{omPackingDensity, }
                                           \AttributeTok{mineral=}\NormalTok{mineralPackingDensity),}
                              \AttributeTok{rootDensity=}\NormalTok{rootPackingDensity, }\AttributeTok{shape=}\NormalTok{shape, }
                              \AttributeTok{mineralInput =}\NormalTok{ initSediment,}
                              \AttributeTok{minDepth =} \FunctionTok{round}\NormalTok{(}\FunctionTok{max}\NormalTok{(rootDepthMax)}\SpecialCharTok{+}\FloatTok{0.5}\NormalTok{))}
\NormalTok{        \}}
\NormalTok{      \}}
\NormalTok{    \} }\ControlFlowTok{else} \ControlFlowTok{if}\NormalTok{ ((initElv }\SpecialCharTok{\textgreater{}=} \FunctionTok{max}\NormalTok{(meanHighWater, meanHighHighWater,}
\NormalTok{                               meanHighHighWaterSpring, }\AttributeTok{na.rm=}\NormalTok{T)) }\SpecialCharTok{\&}
\NormalTok{               (initElv }\SpecialCharTok{\textgreater{}}\NormalTok{ zVegMax)) \{}
      
      \CommentTok{\# If elevation is greater than maximum growing elevation}
      \CommentTok{\# Then assign it an upland soil}
      \CommentTok{\# If an upland soil is provided use it.}
      \ControlFlowTok{if}\NormalTok{ (}\SpecialCharTok{!} \FunctionTok{is.na}\NormalTok{(uplandCohorts)) \{}
\NormalTok{        cohorts }\OtherTok{\textless{}{-}}\NormalTok{ uplandCohorts}
\NormalTok{        bio\_table }\OtherTok{\textless{}{-}} \FunctionTok{data.frame}\NormalTok{(}\AttributeTok{speciesCode=}\ConstantTok{NA}\NormalTok{,}
                                \AttributeTok{rootToShoot=}\ConstantTok{NA}\NormalTok{,}
                                \AttributeTok{rootTurnover=}\ConstantTok{NA}\NormalTok{,}
                                \AttributeTok{abovegroundTurnover=}\ConstantTok{NA}\NormalTok{,}
                                \AttributeTok{rootDepthMax=}\ConstantTok{NA}\NormalTok{,}
                                \AttributeTok{aboveground\_biomass=}\ConstantTok{NA}\NormalTok{,}
                                \AttributeTok{belowground\_biomass=}\ConstantTok{NA}\NormalTok{)}
\NormalTok{        initSediment }\OtherTok{\textless{}{-}} \ConstantTok{NA}
\NormalTok{      \} }\ControlFlowTok{else}\NormalTok{ \{}
        \CommentTok{\# If not assign it an arbitrary 50\% organic matter soil}
\NormalTok{        cohorts }\OtherTok{\textless{}{-}} \FunctionTok{data.frame}\NormalTok{(}\AttributeTok{age=}\FunctionTok{rep}\NormalTok{(}\DecValTok{0}\NormalTok{, }\FunctionTok{round}\NormalTok{(rootDepthMax}\FloatTok{+0.6}\NormalTok{)),}
                              \AttributeTok{fast\_OM=}\FunctionTok{rep}\NormalTok{(}\DecValTok{0}\NormalTok{, }\FunctionTok{round}\NormalTok{(rootDepthMax}\FloatTok{+0.6}\NormalTok{)),}
        \AttributeTok{slow\_OM=}\FunctionTok{rep}\NormalTok{(}\FloatTok{0.5}\SpecialCharTok{*}\NormalTok{(}\DecValTok{1}\SpecialCharTok{/}\NormalTok{(}\FloatTok{0.5}\SpecialCharTok{/}\NormalTok{mineralPackingDensity}\FloatTok{+0.5}\SpecialCharTok{/}\NormalTok{omPackingDensity)),}
                    \FunctionTok{round}\NormalTok{(rootDepthMax}\FloatTok{+0.6}\NormalTok{)),}
                              \AttributeTok{respired\_OM=}\FunctionTok{rep}\NormalTok{(}\DecValTok{0}\NormalTok{, }\FunctionTok{round}\NormalTok{(rootDepthMax}\FloatTok{+0.6}\NormalTok{)),}
        \AttributeTok{mineral=}\FunctionTok{rep}\NormalTok{(}\FloatTok{0.5}\SpecialCharTok{*}\NormalTok{(}\DecValTok{1}\SpecialCharTok{/}\NormalTok{(}\FloatTok{0.5}\SpecialCharTok{/}\NormalTok{mineralPackingDensity}\FloatTok{+0.5}\SpecialCharTok{/}\NormalTok{omPackingDensity)),}
                                          \FunctionTok{round}\NormalTok{(rootDepthMax}\FloatTok{+0.6}\NormalTok{)),}
                              \AttributeTok{root\_mass=}\FunctionTok{rep}\NormalTok{(}\DecValTok{0}\NormalTok{,}\FunctionTok{round}\NormalTok{(rootDepthMax}\FloatTok{+0.6}\NormalTok{)),}
                              \AttributeTok{layer\_top=}\DecValTok{0}\SpecialCharTok{:}\NormalTok{((}\FunctionTok{round}\NormalTok{(rootDepthMax}\FloatTok{+0.6}\NormalTok{)}\SpecialCharTok{{-}}\DecValTok{1}\NormalTok{)),}
                              \AttributeTok{layer\_bottom=}\DecValTok{1}\SpecialCharTok{:}\FunctionTok{round}\NormalTok{(rootDepthMax}\FloatTok{+0.6}\NormalTok{)) }\SpecialCharTok{\%\textgreater{}\%} 
\NormalTok{          dplyr}\SpecialCharTok{::}\FunctionTok{mutate}\NormalTok{(}\AttributeTok{cumCohortVol =} \FunctionTok{cumsum}\NormalTok{(layer\_bottom}\SpecialCharTok{{-}}\NormalTok{layer\_top))}
        
\NormalTok{        bio\_table }\OtherTok{\textless{}{-}} \FunctionTok{data.frame}\NormalTok{(}\AttributeTok{speciesCode=}\ConstantTok{NA}\NormalTok{,}
                                \AttributeTok{rootToShoot=}\ConstantTok{NA}\NormalTok{,}
                                \AttributeTok{rootTurnover=}\ConstantTok{NA}\NormalTok{,}
                                \AttributeTok{abovegroundTurnover=}\ConstantTok{NA}\NormalTok{,}
                                \AttributeTok{rootDepthMax=}\ConstantTok{NA}\NormalTok{,}
                                \AttributeTok{aboveground\_biomass=}\ConstantTok{NA}\NormalTok{,}
                                \AttributeTok{belowground\_biomass=}\ConstantTok{NA}\NormalTok{)}
\NormalTok{        initSediment }\OtherTok{\textless{}{-}} \ConstantTok{NA}
\NormalTok{      \}}
\NormalTok{    \} }\ControlFlowTok{else}\NormalTok{ \{}
      \FunctionTok{stop}\NormalTok{(}\StringTok{"Elevations are invalid for creating initial cohorts."}\NormalTok{)}
\NormalTok{    \}}
\NormalTok{  \}}
  
  \CommentTok{\# Check to make sure it has the right column names,}
  \CommentTok{\# If it does then return it,}
  \CommentTok{\# If not throw an error message}
  \ControlFlowTok{if}\NormalTok{ (}\SpecialCharTok{!} \FunctionTok{all}\NormalTok{(}\FunctionTok{c}\NormalTok{(}\StringTok{"age"}\NormalTok{, }\StringTok{"fast\_OM"}\NormalTok{, }\StringTok{"slow\_OM"}\NormalTok{, }
              \StringTok{"mineral"}\NormalTok{, }\StringTok{"root\_mass"}\NormalTok{, }
              \StringTok{"layer\_top"}\NormalTok{, }\StringTok{"layer\_bottom"}\NormalTok{, }\StringTok{"cumCohortVol"}\NormalTok{)}\SpecialCharTok{\%in\%}\FunctionTok{names}\NormalTok{(cohorts))) \{}
\NormalTok{    missing }\OtherTok{\textless{}{-}} \FunctionTok{paste}\NormalTok{(}\FunctionTok{c}\NormalTok{(}\StringTok{"age"}\NormalTok{, }\StringTok{"fast\_OM"}\NormalTok{, }\StringTok{"slow\_OM"}\NormalTok{, }
                       \StringTok{"mineral"}\NormalTok{, }\StringTok{"root\_mass"}\NormalTok{, }
                       \StringTok{"layer\_top"}\NormalTok{, }\StringTok{"layer\_bottom"}\NormalTok{)[}\SpecialCharTok{!}\FunctionTok{c}\NormalTok{(}\StringTok{"age"}\NormalTok{, }\StringTok{"fast\_OM"}\NormalTok{,}
                                                       \StringTok{"slow\_OM"}\NormalTok{, }\StringTok{"mineral"}\NormalTok{, }
                                                       \StringTok{"root\_mass"}\NormalTok{, }\StringTok{"layer\_top"}\NormalTok{,}
                                                       \StringTok{"layer\_bottom"}\NormalTok{,}
                                            \StringTok{"cumCohortVol"}\NormalTok{)}\SpecialCharTok{\%in\%}\FunctionTok{names}\NormalTok{(cohorts)],}
                     \AttributeTok{collapse =} \StringTok{", "}\NormalTok{)}
    \FunctionTok{stop}\NormalTok{(}\FunctionTok{paste}\NormalTok{(}\StringTok{"Initial cohorts table is missing "}\NormalTok{, missing, }\StringTok{"."}\NormalTok{, }\AttributeTok{sep=}\StringTok{""}\NormalTok{))}
\NormalTok{  \}}
  
  \FunctionTok{return}\NormalTok{(}\FunctionTok{list}\NormalTok{(cohorts,}
\NormalTok{              bio\_table,}
\NormalTok{              initSediment))}
\NormalTok{\}}
\end{Highlighting}
\end{Shaded}

\hypertarget{filldepthcellswithroots}{%
\subsection{\texorpdfstring{\texttt{fillDepthCellsWithRoots()}}{fillDepthCellsWithRoots()}}\label{filldepthcellswithroots}}

\begin{Shaded}
\begin{Highlighting}[]
\NormalTok{fillDepthCellsWithRoots }\OtherTok{\textless{}{-}} \ControlFlowTok{function}\NormalTok{(}\AttributeTok{depthMins =} \DecValTok{0}\SpecialCharTok{:}\DecValTok{149}\NormalTok{, }
                                    \AttributeTok{depthMaxs =} \DecValTok{1}\SpecialCharTok{:}\DecValTok{150}\NormalTok{,}
                                    \AttributeTok{totalRootBmass =} \DecValTok{3000}\NormalTok{,}
                                    \AttributeTok{rootShape =} \StringTok{\textquotesingle{}linear\textquotesingle{}}\NormalTok{,}
                                    \AttributeTok{rootDepthMax =} \DecValTok{30}\NormalTok{,}
                                    \AttributeTok{customFunction =} \ControlFlowTok{function}\NormalTok{(x) \{ }
                                      \FunctionTok{exp}\NormalTok{(}\SpecialCharTok{{-}}\NormalTok{((x}\SpecialCharTok{{-}}\NormalTok{maxRootDepth)}\SpecialCharTok{\^{}}\DecValTok{2}\SpecialCharTok{/}\NormalTok{lambdaRoot}\SpecialCharTok{\^{}}\DecValTok{2}\NormalTok{)) }
\NormalTok{                                    \},}
                                    \AttributeTok{customParams =} \FunctionTok{list}\NormalTok{(}\AttributeTok{lambdaRoot =} \DecValTok{33}\NormalTok{,}
                                                        \AttributeTok{maxRootDepth =} \DecValTok{30}\NormalTok{)}
\NormalTok{) \{}
  

  
  \CommentTok{\# Calculate root mass for each min to max depth inteval.}
  \ControlFlowTok{if}\NormalTok{ (rootShape }\SpecialCharTok{==} \StringTok{"linear"}\NormalTok{) \{ }\CommentTok{\# if we\textquotesingle{}re calculating a linear root mass shape}
    
    \CommentTok{\# Calculate intercept of linear function}
\NormalTok{    rootMassInt }\OtherTok{\textless{}{-}} \DecValTok{2} \SpecialCharTok{*}\NormalTok{ totalRootBmass }\SpecialCharTok{/}\NormalTok{ rootDepthMax}
    
    \CommentTok{\# If the depth inteval is above the rooting depth}
    \CommentTok{\# calculate root mass as integral between min to max depth.}
    \CommentTok{\# If it\textquotesingle{}s below rootin depth, root mass is 0.}
\NormalTok{    rootMassInCell }\OtherTok{\textless{}{-}} \FunctionTok{ifelse}\NormalTok{(depthMins }\SpecialCharTok{\textless{}}\NormalTok{ rootDepthMax,}
\NormalTok{                             rootMassInt }\SpecialCharTok{*}\NormalTok{ (depthMaxs }\SpecialCharTok{{-}}\NormalTok{ depthMins) }\SpecialCharTok{{-}}
\NormalTok{                               rootMassInt }\SpecialCharTok{*}\NormalTok{ (depthMaxs }\SpecialCharTok{\^{}} \DecValTok{2} \SpecialCharTok{{-}}\NormalTok{ depthMins }\SpecialCharTok{\^{}} \DecValTok{2}\NormalTok{) }\SpecialCharTok{/}
\NormalTok{                               (}\DecValTok{2} \SpecialCharTok{*}\NormalTok{ rootDepthMax),}
                             \DecValTok{0}\NormalTok{)}
    
\NormalTok{  \} }\ControlFlowTok{else} \ControlFlowTok{if}\NormalTok{ (rootShape }\SpecialCharTok{==} \StringTok{"exponential"}\NormalTok{) \{ }
    
    \CommentTok{\# if we\textquotesingle{}re calculating an exponential root mass shape}
    \CommentTok{\# Define an asymtote that is a small number }
\NormalTok{    rootBmassAsymtote }\OtherTok{\textless{}{-}} \FunctionTok{log}\NormalTok{(}\FloatTok{0.05}\NormalTok{) }\SpecialCharTok{/}\NormalTok{ rootDepthMax}
    
    \CommentTok{\# Calculate intercept of the exponential function}
\NormalTok{    rootMassInt }\OtherTok{\textless{}{-}} \SpecialCharTok{{-}}\FloatTok{0.95} \SpecialCharTok{*}\NormalTok{ totalRootBmass }\SpecialCharTok{*}\NormalTok{ rootBmassAsymtote }\SpecialCharTok{/} 
\NormalTok{      (}\DecValTok{1} \SpecialCharTok{{-}} \FunctionTok{exp}\NormalTok{(rootBmassAsymtote }\SpecialCharTok{*}\NormalTok{ rootDepthMax))}
    
    \CommentTok{\#  calculate root mass as integral between min to max in two steps.}
\NormalTok{    e1 }\OtherTok{\textless{}{-}} \FunctionTok{exp}\NormalTok{(rootBmassAsymtote }\SpecialCharTok{*}\NormalTok{ depthMaxs) }\CommentTok{\# step 1}
    
    \CommentTok{\# step 2}
\NormalTok{    rootMassInCell }\OtherTok{\textless{}{-}}\NormalTok{ (rootMassInt }\SpecialCharTok{/}\NormalTok{ rootBmassAsymtote) }\SpecialCharTok{*} 
\NormalTok{      (e1 }\SpecialCharTok{{-}} \FunctionTok{exp}\NormalTok{(depthMins }\SpecialCharTok{*}\NormalTok{ rootBmassAsymtote))}
    
\NormalTok{  \} }\ControlFlowTok{else} \ControlFlowTok{if}\NormalTok{ (rootShape }\SpecialCharTok{==} \StringTok{"custom"}\NormalTok{) \{ }\CommentTok{\# If another custom function,}
    
    \CommentTok{\# Go through the list and make sure each parameter is an object in memory}
    \ControlFlowTok{for}\NormalTok{ (i }\ControlFlowTok{in} \DecValTok{1}\SpecialCharTok{:}\FunctionTok{length}\NormalTok{(customParams)) \{}
\NormalTok{      paramName }\OtherTok{\textless{}{-}} \FunctionTok{names}\NormalTok{(customParams[i])}
      \FunctionTok{assign}\NormalTok{(paramName, customParams[[paramName]])}
\NormalTok{    \}}
    
    \CommentTok{\# Iterate through depth series and integrate function at min and max depth}
\NormalTok{    rootDensity }\OtherTok{\textless{}{-}} \FunctionTok{c}\NormalTok{() }\CommentTok{\# blank vector for storing outputs}
    \ControlFlowTok{for}\NormalTok{ (i }\ControlFlowTok{in} \DecValTok{1}\SpecialCharTok{:}\FunctionTok{length}\NormalTok{(depthMaxs)) \{}
\NormalTok{      rootMassCell\_i }\OtherTok{\textless{}{-}}\FunctionTok{integrate}\NormalTok{(customFunction, depthMins[i], depthMaxs[i]}
\NormalTok{      )}\SpecialCharTok{$}\NormalTok{value}
\NormalTok{      rootDensity }\OtherTok{\textless{}{-}} \FunctionTok{c}\NormalTok{(rootDensity, rootMassCell\_i)}
\NormalTok{    \}}
    
    \CommentTok{\# normalize and mutliply by total biomass}
\NormalTok{    rootMassInCell }\OtherTok{\textless{}{-}}\NormalTok{ rootDensity }\SpecialCharTok{/} \FunctionTok{sum}\NormalTok{(rootDensity) }\SpecialCharTok{*}\NormalTok{ totalRootBmass}
    
\NormalTok{  \}}
  
  \CommentTok{\# Return a data frame with depth intervals and roots in cells}
  \FunctionTok{return}\NormalTok{(}\FunctionTok{data.frame}\NormalTok{(}\AttributeTok{depthMin =}\NormalTok{ depthMins, }
                    \AttributeTok{depthMax =}\NormalTok{ depthMaxs, }
                    \AttributeTok{rootBiomass\_gPerM2 =}\NormalTok{ rootMassInCell))}
\NormalTok{\}}
\end{Highlighting}
\end{Shaded}

\hypertarget{floodtimelinear}{%
\subsection{\texorpdfstring{\texttt{floodTimeLinear()}}{floodTimeLinear()}}\label{floodtimelinear}}

\begin{Shaded}
\begin{Highlighting}[]
\NormalTok{floodTimeLinear }\OtherTok{\textless{}{-}} \ControlFlowTok{function}\NormalTok{(z, datumHigh, datumLow, }\AttributeTok{tidalCycleLength =} \DecValTok{1}\NormalTok{) \{}
\NormalTok{  floodFract }\OtherTok{\textless{}{-}} \FunctionTok{ifelse}\NormalTok{(z }\SpecialCharTok{\textgreater{}=}\NormalTok{ datumHigh, }\DecValTok{0}\NormalTok{, }
                       \FunctionTok{ifelse}\NormalTok{(z }\SpecialCharTok{\textless{}=}\NormalTok{ datumLow, }\DecValTok{1}\NormalTok{, }
\NormalTok{                              (datumHigh}\SpecialCharTok{{-}}\NormalTok{z)}\SpecialCharTok{/}\NormalTok{(datumHigh}\SpecialCharTok{{-}}\NormalTok{datumLow)))}
\NormalTok{  floodTime }\OtherTok{\textless{}{-}}\NormalTok{ floodFract }\SpecialCharTok{*}\NormalTok{ tidalCycleLength}
  \FunctionTok{return}\NormalTok{(floodTime)}
\NormalTok{\}}
\end{Highlighting}
\end{Shaded}

\hypertarget{floodtimetrig}{%
\subsection{\texorpdfstring{\texttt{floodTimeTrig()}}{floodTimeTrig()}}\label{floodtimetrig}}

\begin{Shaded}
\begin{Highlighting}[]
\CommentTok{\# Second function if flood time is calculated trigonometrically}
\NormalTok{floodTimeTrig }\OtherTok{\textless{}{-}} \ControlFlowTok{function}\NormalTok{(z, datumHigh, datumLow, }\AttributeTok{tidalCycleLength =} \DecValTok{1}\NormalTok{) \{}
  
  \CommentTok{\# If elevation is above the tidal range indation time is 0}
\NormalTok{  datumHigh }\OtherTok{\textless{}{-}} \FunctionTok{ifelse}\NormalTok{(z}\SpecialCharTok{\textgreater{}=}\NormalTok{datumHigh, z, datumHigh)}
  
  \CommentTok{\# If elevation is below inundation time is a full tidal cycle}
\NormalTok{  datumLow }\OtherTok{\textless{}{-}} \FunctionTok{ifelse}\NormalTok{(z}\SpecialCharTok{\textless{}=}\NormalTok{datumLow, z, datumLow)}
  
  \CommentTok{\# Rising time over cell = 6.21 (A/pi {-} 1) where A = 2* pi {-} cos{-}1 [2 (height}
  \CommentTok{\# of cell – meanLowWater) / (meanHighWater – meanLowWater) {-} 1] radians}
\NormalTok{  A1 }\OtherTok{\textless{}{-}} \DecValTok{2} \SpecialCharTok{*}\NormalTok{ pi }\SpecialCharTok{{-}} \FunctionTok{acos}\NormalTok{(}\DecValTok{2} \SpecialCharTok{*}\NormalTok{ (z}\SpecialCharTok{{-}}\NormalTok{datumLow) }\SpecialCharTok{/}\NormalTok{ (datumHigh}\SpecialCharTok{{-}}\NormalTok{datumLow) }\SpecialCharTok{{-}} \DecValTok{1}\NormalTok{)}
\NormalTok{  risingTime }\OtherTok{\textless{}{-}}\NormalTok{ tidalCycleLength}\SpecialCharTok{/}\DecValTok{2} \SpecialCharTok{*}\NormalTok{ (A1}\SpecialCharTok{/}\NormalTok{pi }\SpecialCharTok{{-}} \DecValTok{1}\NormalTok{)}
  
  \CommentTok{\# Falling time over cell = 6.21 (A/pi {-} 1) where A = 2* {-} cos{-}1 [2 (height of}
  \CommentTok{\# cell – meanHighWater) / (meanLowWater – meanHighWater) {-} 1] radians}
\NormalTok{  A2 }\OtherTok{\textless{}{-}} \DecValTok{2} \SpecialCharTok{*}\NormalTok{ pi }\SpecialCharTok{{-}} \FunctionTok{acos}\NormalTok{(}\DecValTok{2} \SpecialCharTok{*}\NormalTok{ (z}\SpecialCharTok{{-}}\NormalTok{datumHigh) }\SpecialCharTok{/}\NormalTok{ (datumLow}\SpecialCharTok{{-}}\NormalTok{datumHigh) }\SpecialCharTok{{-}} \DecValTok{1}\NormalTok{)}
\NormalTok{  fallingTime }\OtherTok{\textless{}{-}}\NormalTok{ tidalCycleLength}\SpecialCharTok{/}\DecValTok{2} \SpecialCharTok{*}\NormalTok{ (A2}\SpecialCharTok{/}\NormalTok{pi }\SpecialCharTok{{-}} \DecValTok{1}\NormalTok{)}
  
  \CommentTok{\# If between inundation time = abs (time rising {-} 6.21) + time falling}
\NormalTok{  inundationTime }\OtherTok{\textless{}{-}} \FunctionTok{abs}\NormalTok{(risingTime }\SpecialCharTok{{-}}\NormalTok{ tidalCycleLength}\SpecialCharTok{/}\DecValTok{2}\NormalTok{) }\SpecialCharTok{+}\NormalTok{ fallingTime}
  
  \FunctionTok{return}\NormalTok{(inundationTime)}
\NormalTok{\}}
\end{Highlighting}
\end{Shaded}

\hypertarget{massliveroots}{%
\subsection{\texorpdfstring{\texttt{massLiveRoots()}}{massLiveRoots()}}\label{massliveroots}}

\begin{Shaded}
\begin{Highlighting}[]
\NormalTok{massLiveRoots }\OtherTok{\textless{}{-}} \ControlFlowTok{function}\NormalTok{(layerBottom, layerTop, }
\NormalTok{                          totalRootMassPerArea, }
\NormalTok{                          rootDepthMax,}
                          \AttributeTok{soilLength=}\DecValTok{1}\NormalTok{, }\AttributeTok{soilWidth=}\DecValTok{1}\NormalTok{, }
\NormalTok{                          shape,}
                          \AttributeTok{expDecayRatePerMaxDepth =} \FunctionTok{log}\NormalTok{(}\FloatTok{0.05}\NormalTok{),}
\NormalTok{                          ...)\{}
  
\NormalTok{  totalRootMass }\OtherTok{\textless{}{-}}\NormalTok{ soilLength}\SpecialCharTok{*}\NormalTok{soilWidth}\SpecialCharTok{*}\NormalTok{totalRootMassPerArea}
  
  \ControlFlowTok{if}\NormalTok{ (totalRootMass }\SpecialCharTok{==} \DecValTok{0}\NormalTok{) \{}
\NormalTok{    rootMass }\OtherTok{\textless{}{-}} \FunctionTok{rep}\NormalTok{(}\DecValTok{0}\NormalTok{, }\FunctionTok{length}\NormalTok{(layerBottom))}
    \FunctionTok{return}\NormalTok{(rootMass)}
\NormalTok{  \} }\ControlFlowTok{else}\NormalTok{ \{}
    
\NormalTok{    layerBottom[}\FunctionTok{is.na}\NormalTok{(layerBottom)] }\OtherTok{\textless{}{-}} \DecValTok{0}
\NormalTok{    layerTop[}\FunctionTok{is.na}\NormalTok{(layerTop)] }\OtherTok{\textless{}{-}} \DecValTok{0}
    
    \ControlFlowTok{if}\NormalTok{(}\FunctionTok{any}\NormalTok{(layerBottom }\SpecialCharTok{\textless{}}\NormalTok{ layerTop))\{}
      \CommentTok{\#print(data.frame(layerBottom, layerTop))}
      \FunctionTok{stop}\NormalTok{(}\StringTok{\textquotesingle{}Bad layer definition.\textquotesingle{}}\NormalTok{)}
\NormalTok{    \}}
    
    \DocumentationTok{\#\#reset the layers that are beyond the root inputs to have 0 depths}
\NormalTok{    layerBottom[layerBottom }\SpecialCharTok{\textgreater{}}\NormalTok{ rootDepthMax] }\OtherTok{\textless{}{-}}\NormalTok{ rootDepthMax}
\NormalTok{    layerTop[layerTop }\SpecialCharTok{\textgreater{}}\NormalTok{ rootDepthMax] }\OtherTok{\textless{}{-}}\NormalTok{ rootDepthMax}
    
    \ControlFlowTok{if}\NormalTok{(shape }\SpecialCharTok{==} \StringTok{\textquotesingle{}linear\textquotesingle{}}\NormalTok{)\{}
\NormalTok{      slope }\OtherTok{\textless{}{-}} \SpecialCharTok{{-}}\DecValTok{2} \SpecialCharTok{*}\NormalTok{ totalRootMass }\SpecialCharTok{/}\NormalTok{ (rootDepthMax}\SpecialCharTok{\^{}}\DecValTok{2}\NormalTok{)}
\NormalTok{      intercept }\OtherTok{\textless{}{-}} \DecValTok{2} \SpecialCharTok{*}\NormalTok{ totalRootMass }\SpecialCharTok{/}\NormalTok{ rootDepthMax}
      \CommentTok{\#mass = integral(mass\_per\_depth, depth)}
\NormalTok{      rootMass }\OtherTok{\textless{}{-}}\NormalTok{ intercept}\SpecialCharTok{*}\NormalTok{(layerBottom}\SpecialCharTok{{-}}\NormalTok{layerTop) }\SpecialCharTok{+}
\NormalTok{        slope}\SpecialCharTok{/}\DecValTok{2}\SpecialCharTok{*}\NormalTok{(layerBottom }\SpecialCharTok{\^{}}\DecValTok{2}\SpecialCharTok{{-}}\NormalTok{layerTop}\SpecialCharTok{\^{}}\DecValTok{2}\NormalTok{)}
      
\NormalTok{    \}}\ControlFlowTok{else}\NormalTok{\{}
      \ControlFlowTok{if}\NormalTok{(shape }\SpecialCharTok{==} \StringTok{\textquotesingle{}exponential\textquotesingle{}}\NormalTok{)\{}
\NormalTok{        b }\OtherTok{\textless{}{-}}\NormalTok{ expDecayRatePerMaxDepth }\SpecialCharTok{/}\NormalTok{ rootDepthMax }
        \CommentTok{\#convert from total profile to per cm}
\NormalTok{        a }\OtherTok{\textless{}{-}}\NormalTok{ totalRootMass }\SpecialCharTok{*}\NormalTok{ (}\DecValTok{1} \SpecialCharTok{/}\NormalTok{ b }\SpecialCharTok{*} \FunctionTok{exp}\NormalTok{(b }\SpecialCharTok{*}\NormalTok{ rootDepthMax) }\SpecialCharTok{{-}}  
                                \FunctionTok{exp}\NormalTok{(b }\SpecialCharTok{*}\NormalTok{ rootDepthMax) }\SpecialCharTok{*}\NormalTok{ rootDepthMax }\SpecialCharTok{{-}} \DecValTok{1} \SpecialCharTok{/}\NormalTok{ b)}\SpecialCharTok{\^{}{-}}\DecValTok{1}
\NormalTok{        m }\OtherTok{\textless{}{-}}\NormalTok{ a }\SpecialCharTok{*} \FunctionTok{exp}\NormalTok{(b }\SpecialCharTok{*}\NormalTok{ rootDepthMax)}
\NormalTok{        rootMass }\OtherTok{\textless{}{-}}\NormalTok{ a }\SpecialCharTok{/}\NormalTok{ b }\SpecialCharTok{*}\NormalTok{ (}\FunctionTok{exp}\NormalTok{(b }\SpecialCharTok{*}\NormalTok{ layerBottom)}\SpecialCharTok{{-}}\FunctionTok{exp}\NormalTok{(b }\SpecialCharTok{*}\NormalTok{ layerTop)) }\SpecialCharTok{+}
\NormalTok{          m}\SpecialCharTok{*}\NormalTok{(layerTop }\SpecialCharTok{{-}}\NormalTok{ layerBottom)}
        
\NormalTok{      \}}\ControlFlowTok{else}\NormalTok{\{}
        
        \FunctionTok{stop}\NormalTok{(}\FunctionTok{paste}\NormalTok{(}\StringTok{\textquotesingle{}Unknown shape specified:\textquotesingle{}}\NormalTok{, shape))}
\NormalTok{      \}}
\NormalTok{    \}}
    \FunctionTok{return}\NormalTok{(rootMass)}
\NormalTok{  \}}
  
  
  
\NormalTok{\}}
\end{Highlighting}
\end{Shaded}

\hypertarget{predictbiomass}{%
\subsection{\texorpdfstring{\texttt{predictBiomass()}}{predictBiomass()}}\label{predictbiomass}}

\begin{Shaded}
\begin{Highlighting}[]
\NormalTok{predictBiomass }\OtherTok{\textless{}{-}} \ControlFlowTok{function}\NormalTok{(}\AttributeTok{z=}\DecValTok{0}\NormalTok{, }\AttributeTok{bMax=}\DecValTok{2500}\NormalTok{, }\AttributeTok{zVegMax=}\DecValTok{3}\NormalTok{, }\AttributeTok{zVegMin=}\SpecialCharTok{{-}}\DecValTok{1}\NormalTok{, }\AttributeTok{zVegPeak=}\ConstantTok{NA}\NormalTok{) \{}

  \CommentTok{\# Stop the function if there are invalid parameters}
  \ControlFlowTok{if}\NormalTok{ ( (bMax }\SpecialCharTok{\textless{}} \DecValTok{0}\NormalTok{) }\SpecialCharTok{|} \CommentTok{\# negative peak biomss}
       \CommentTok{\# or elevations that don\textquotesingle{}t make sense}
\NormalTok{       (zVegMax }\SpecialCharTok{\textless{}=} \FunctionTok{max}\NormalTok{(zVegMin, zVegPeak, }\AttributeTok{na.rm=}\NormalTok{T)) }\SpecialCharTok{|}
\NormalTok{       ((zVegMin }\SpecialCharTok{\textgreater{}=} \FunctionTok{min}\NormalTok{(zVegMax, zVegPeak, }\AttributeTok{na.rm=}\NormalTok{T)))}
\NormalTok{       ) \{}
    \FunctionTok{stop}\NormalTok{(}\StringTok{"invalid biomass parameters"}\NormalTok{)}
\NormalTok{  \}}
  
  \CommentTok{\# If there is no peak elevation for vegetation parabola is symmetric.}
  \ControlFlowTok{if}\NormalTok{ (}\FunctionTok{is.na}\NormalTok{(zVegPeak)) \{ }
    
    \CommentTok{\# Elevation of the veg. peak is halfway between the min and max limits}
\NormalTok{    zVegPeak}\OtherTok{\textless{}{-}}\NormalTok{(zVegMax}\SpecialCharTok{+}\NormalTok{zVegMin)}\SpecialCharTok{/}\DecValTok{2}
  
    \CommentTok{\# From bmax, min, and max elevation limits, solve for parameters of a}
    \CommentTok{\# parabola.}
\NormalTok{    a }\OtherTok{\textless{}{-}} \SpecialCharTok{{-}}\NormalTok{((}\SpecialCharTok{{-}}\NormalTok{zVegMin }\SpecialCharTok{*}\NormalTok{ bMax }\SpecialCharTok{{-}}\NormalTok{ zVegMax }\SpecialCharTok{*}\NormalTok{ bMax) }\SpecialCharTok{/}
\NormalTok{             ((zVegMin }\SpecialCharTok{{-}}\NormalTok{ zVegPeak) }\SpecialCharTok{*}\NormalTok{ (}\SpecialCharTok{{-}}\NormalTok{zVegMax }\SpecialCharTok{+}\NormalTok{ zVegPeak)))}
\NormalTok{    b }\OtherTok{\textless{}{-}} \SpecialCharTok{{-}}\NormalTok{(bMax }\SpecialCharTok{/}\NormalTok{ ((zVegMin }\SpecialCharTok{{-}}\NormalTok{ zVegPeak) }\SpecialCharTok{*}\NormalTok{ (}\SpecialCharTok{{-}}\NormalTok{zVegMax }\SpecialCharTok{+}\NormalTok{ zVegPeak)))}
\NormalTok{    c }\OtherTok{\textless{}{-}}\NormalTok{ (zVegMin }\SpecialCharTok{*}\NormalTok{ zVegMax }\SpecialCharTok{*}\NormalTok{ bMax) }\SpecialCharTok{/}
\NormalTok{      ((zVegMin }\SpecialCharTok{{-}}\NormalTok{ zVegPeak) }\SpecialCharTok{*}\NormalTok{ (zVegMax }\SpecialCharTok{{-}}\NormalTok{ zVegPeak))}
    
    \CommentTok{\# Apply parabolic function to elevation to calulate above ground biomass.}
\NormalTok{    agb }\OtherTok{\textless{}{-}}\NormalTok{ a}\SpecialCharTok{*}\NormalTok{z }\SpecialCharTok{+}\NormalTok{ b}\SpecialCharTok{*}\NormalTok{z}\SpecialCharTok{\^{}}\DecValTok{2} \SpecialCharTok{+}\NormalTok{ c}
  
  \CommentTok{\# If elevation of peak biomass is specified the curve can be more flexible ...}
\NormalTok{  \} }\ControlFlowTok{else}\NormalTok{ \{}
    \CommentTok{\# ... but we need to split it into two curves.}
    
    \CommentTok{\# For the curve applied on the \textquotesingle{}upper end\textquotesingle{} create a new minimum elevation}
    \CommentTok{\# mirroring  upper elevation limit accross the peak biomass elevation.}
\NormalTok{    zVegMin\_up }\OtherTok{\textless{}{-}}\NormalTok{ zVegPeak}\SpecialCharTok{{-}}\NormalTok{((zVegMax}\SpecialCharTok{{-}}\NormalTok{zVegPeak)) }
    
    \CommentTok{\# For the curve applied at the \textquotesingle{}lower end\textquotesingle{}, same. Create new maximum}
    \CommentTok{\# elevation tolerance mirroring, lower elevation limit accros the peak}
    \CommentTok{\# biomass elevation.}
\NormalTok{    zVegMax\_low }\OtherTok{\textless{}{-}}\NormalTok{zVegPeak}\SpecialCharTok{+}\NormalTok{((zVegPeak}\SpecialCharTok{{-}}\NormalTok{zVegMin))}
    
    \CommentTok{\# Solve for the parameters of the upper curve.}
\NormalTok{    a\_up }\OtherTok{\textless{}{-}} \SpecialCharTok{{-}}\NormalTok{((}\SpecialCharTok{{-}}\NormalTok{zVegMin\_up }\SpecialCharTok{*}\NormalTok{ bMax }\SpecialCharTok{{-}}\NormalTok{ zVegMax }\SpecialCharTok{*}\NormalTok{ bMax) }\SpecialCharTok{/} 
\NormalTok{                ((zVegMin\_up }\SpecialCharTok{{-}}\NormalTok{ zVegPeak) }\SpecialCharTok{*}\NormalTok{ (}\SpecialCharTok{{-}}\NormalTok{zVegMax }\SpecialCharTok{+}\NormalTok{ zVegPeak)))}
\NormalTok{    b\_up }\OtherTok{\textless{}{-}} \SpecialCharTok{{-}}\NormalTok{(bMax }\SpecialCharTok{/}\NormalTok{ ((zVegMin\_up }\SpecialCharTok{{-}}\NormalTok{ zVegPeak) }\SpecialCharTok{*}\NormalTok{ (}\SpecialCharTok{{-}}\NormalTok{zVegMax }\SpecialCharTok{+}\NormalTok{ zVegPeak)))}
\NormalTok{    c\_up }\OtherTok{\textless{}{-}}\NormalTok{ (zVegMin\_up }\SpecialCharTok{*}\NormalTok{ zVegMax }\SpecialCharTok{*}\NormalTok{ bMax) }\SpecialCharTok{/} 
\NormalTok{      ((zVegMin\_up }\SpecialCharTok{{-}}\NormalTok{ zVegPeak) }\SpecialCharTok{*}\NormalTok{ (zVegMax }\SpecialCharTok{{-}}\NormalTok{ zVegPeak))}
    
    \CommentTok{\# Solve for the parametrs of the lower curve.}
\NormalTok{    a\_low }\OtherTok{\textless{}{-}} \SpecialCharTok{{-}}\NormalTok{((}\SpecialCharTok{{-}}\NormalTok{zVegMin }\SpecialCharTok{*}\NormalTok{ bMax }\SpecialCharTok{{-}}\NormalTok{ zVegMax\_low }\SpecialCharTok{*}\NormalTok{ bMax) }\SpecialCharTok{/} 
\NormalTok{                 ((zVegMin }\SpecialCharTok{{-}}\NormalTok{ zVegPeak) }\SpecialCharTok{*}\NormalTok{ (}\SpecialCharTok{{-}}\NormalTok{zVegMax\_low }\SpecialCharTok{+}\NormalTok{ zVegPeak)))}
\NormalTok{    b\_low }\OtherTok{\textless{}{-}} \SpecialCharTok{{-}}\NormalTok{(bMax }\SpecialCharTok{/}\NormalTok{ ((zVegMin }\SpecialCharTok{{-}}\NormalTok{ zVegPeak) }\SpecialCharTok{*}\NormalTok{ (}\SpecialCharTok{{-}}\NormalTok{zVegMax\_low }\SpecialCharTok{+}\NormalTok{ zVegPeak)))}
\NormalTok{    c\_low }\OtherTok{\textless{}{-}}\NormalTok{ (zVegMin }\SpecialCharTok{*}\NormalTok{ zVegMax\_low }\SpecialCharTok{*}\NormalTok{ bMax) }\SpecialCharTok{/} 
\NormalTok{      ((zVegMin }\SpecialCharTok{{-}}\NormalTok{ zVegPeak) }\SpecialCharTok{*}\NormalTok{ (zVegMax\_low }\SpecialCharTok{{-}}\NormalTok{ zVegPeak))}
    
    \CommentTok{\# If elevation is above the specified peak biomass elevation, apply the}
    \CommentTok{\# upper curve, if it\textquotesingle{}s under apply the lower curve}
\NormalTok{    agb }\OtherTok{\textless{}{-}} \FunctionTok{ifelse}\NormalTok{(z}\SpecialCharTok{\textgreater{}}\NormalTok{zVegPeak, a\_up}\SpecialCharTok{*}\NormalTok{z }\SpecialCharTok{+}\NormalTok{ b\_up}\SpecialCharTok{*}\NormalTok{z}\SpecialCharTok{\^{}}\DecValTok{2} \SpecialCharTok{+}
\NormalTok{                    c\_up, a\_low}\SpecialCharTok{*}\NormalTok{z }\SpecialCharTok{+}\NormalTok{ b\_low}\SpecialCharTok{*}\NormalTok{z}\SpecialCharTok{\^{}}\DecValTok{2} \SpecialCharTok{+}\NormalTok{ c\_low)}
    
\NormalTok{  \}}
  
  \CommentTok{\# Recast any negative values as 0, since biomass can\textquotesingle{}t be negative.}
\NormalTok{  agb }\OtherTok{\textless{}{-}} \FunctionTok{ifelse}\NormalTok{(agb}\SpecialCharTok{\textgreater{}}\DecValTok{0}\NormalTok{,agb,}\DecValTok{0}\NormalTok{) }
  \FunctionTok{return}\NormalTok{(agb) }\CommentTok{\# Return biomass as a function of elevation.}
  
\NormalTok{\}}
\end{Highlighting}
\end{Shaded}

\hypertarget{predictlunarnodalcycle}{%
\subsection{\texorpdfstring{\texttt{predictLunarNodalCycle()}}{predictLunarNodalCycle()}}\label{predictlunarnodalcycle}}

\begin{Shaded}
\begin{Highlighting}[]
\NormalTok{predictLunarNodalCycle }\OtherTok{\textless{}{-}} \ControlFlowTok{function}\NormalTok{(year, floodElv, meanSeaLevelDatum,}
\NormalTok{                                   meanSeaLevel, lunarNodalAmp) \{}
  \CommentTok{\# Build meanHighWater lines based on meanSeaLevel, long{-}term tidal range and}
  \CommentTok{\# lunar nodal amplitude}
\NormalTok{  meanHighWater }\OtherTok{\textless{}{-}}\NormalTok{ meanSeaLevel }\SpecialCharTok{+}\NormalTok{ (floodElv}\SpecialCharTok{{-}}\NormalTok{meanSeaLevelDatum) }\SpecialCharTok{+}
\NormalTok{    (lunarNodalAmp }\SpecialCharTok{*}\NormalTok{ (}\FunctionTok{sin}\NormalTok{(}\DecValTok{2}\SpecialCharTok{*}\NormalTok{pi}\SpecialCharTok{*}\NormalTok{(year}\DecValTok{{-}1983}\NormalTok{)}\SpecialCharTok{/}\FloatTok{18.61}\NormalTok{)))}
  \FunctionTok{return}\NormalTok{(meanHighWater)}
\NormalTok{\}}
\end{Highlighting}
\end{Shaded}

\hypertarget{runcohortmem}{%
\subsection{\texorpdfstring{\texttt{runCohortMem()}}{runCohortMem()}}\label{runcohortmem}}

\begin{Shaded}
\begin{Highlighting}[]
\NormalTok{runCohortMem }\OtherTok{\textless{}{-}} \ControlFlowTok{function}\NormalTok{(startYear, }\AttributeTok{endYear=}\NormalTok{startYear}\SpecialCharTok{+}\DecValTok{99}\NormalTok{, relSeaLevelRiseInit,}
\NormalTok{                         relSeaLevelRiseTotal, initElv,}
\NormalTok{                         meanSeaLevel, }\AttributeTok{meanSeaLevelDatum=}\NormalTok{meanSeaLevel[}\DecValTok{1}\NormalTok{],}
\NormalTok{                         meanHighWaterDatum,}
                         \AttributeTok{meanHighHighWaterDatum=}\ConstantTok{NA}\NormalTok{,}
                         \AttributeTok{meanHighHighWaterSpringDatum=}\ConstantTok{NA}\NormalTok{, }
\NormalTok{                         suspendedSediment, lunarNodalAmp, }
                         \AttributeTok{lunarNodalPhase=}\FloatTok{2011.181}\NormalTok{,}
                         \AttributeTok{nFloods =} \FloatTok{705.79}\NormalTok{, }\AttributeTok{floodTime.fn =}\NormalTok{ floodTimeLinear,}
\NormalTok{                         bMax, zVegMin, zVegMax, zVegPeak, plantElevationType,}
\NormalTok{                         rootToShoot, rootTurnover, }\AttributeTok{abovegroundTurnover=}\ConstantTok{NA}\NormalTok{,}
                         \AttributeTok{speciesCode=}\ConstantTok{NA}\NormalTok{, rootDepthMax, }\AttributeTok{shape=}\StringTok{"linear"}\NormalTok{,}
\NormalTok{                         omDecayRate, recalcitrantFrac, captureRate,}
                         \AttributeTok{omToOcParams =} \FunctionTok{list}\NormalTok{(}\AttributeTok{B0=}\DecValTok{0}\NormalTok{, }\AttributeTok{B1=}\FloatTok{0.48}\NormalTok{),}
                         \AttributeTok{omPackingDensity=}\FloatTok{0.085}\NormalTok{, }\AttributeTok{mineralPackingDensity=}\FloatTok{1.99}\NormalTok{,}
                         \AttributeTok{rootPackingDensity=}\NormalTok{omPackingDensity,}
                         \AttributeTok{initialCohorts=}\ConstantTok{NA}\NormalTok{,}
                         \AttributeTok{uplandCohorts=}\ConstantTok{NA}\NormalTok{,}
                         \AttributeTok{supertidalCohorts=}\ConstantTok{NA}\NormalTok{,}
                         \AttributeTok{supertidalSedimentInput=}\ConstantTok{NA}\NormalTok{,}
\NormalTok{                         ...) \{}
  
  \CommentTok{\# Make sure tidyverse is there}
  \DocumentationTok{\#\#TODO move this to the package description}
  \FunctionTok{require}\NormalTok{(tidyverse, }\AttributeTok{quietly =} \ConstantTok{TRUE}\NormalTok{)}
  
  \CommentTok{\# Build scenario curve}
\NormalTok{  scenario }\OtherTok{\textless{}{-}} \FunctionTok{buildScenarioCurve}\NormalTok{(}\AttributeTok{startYear=}\NormalTok{startYear, }\AttributeTok{endYear=}\NormalTok{endYear, }
                                 \AttributeTok{meanSeaLevel=}\NormalTok{meanSeaLevel, }
                                 \AttributeTok{relSeaLevelRiseInit=}\NormalTok{relSeaLevelRiseInit, }
                                 \AttributeTok{relSeaLevelRiseTotal=}\NormalTok{relSeaLevelRiseTotal, }
                                 \AttributeTok{suspendedSediment=}\NormalTok{suspendedSediment)}
  
  \CommentTok{\# add high tides}
\NormalTok{  scenario }\OtherTok{\textless{}{-}} \FunctionTok{buildHighTideScenario}\NormalTok{(scenario, }
                                    \AttributeTok{meanSeaLevelDatum=}\NormalTok{meanSeaLevelDatum, }
                                    \AttributeTok{meanHighWaterDatum=}\NormalTok{meanHighWaterDatum, }
                                    \AttributeTok{meanHighHighWaterDatum=}
\NormalTok{                                      meanHighHighWaterDatum, }
                                    \AttributeTok{meanHighHighWaterSpringDatum=}
\NormalTok{                                      meanHighHighWaterSpringDatum, }
                                    \AttributeTok{lunarNodalAmp=}\NormalTok{lunarNodalAmp,}
                                    \AttributeTok{lunarNodalPhase=}\NormalTok{lunarNodalPhase)}
  
  \CommentTok{\# Add blank colums for attributes we will add later}
\NormalTok{  scenario}\SpecialCharTok{$}\NormalTok{surfaceElevation }\OtherTok{\textless{}{-}} \FunctionTok{as.numeric}\NormalTok{(}\FunctionTok{rep}\NormalTok{(}\ConstantTok{NA}\NormalTok{, }\FunctionTok{nrow}\NormalTok{(scenario)))}
\NormalTok{  scenario}\SpecialCharTok{$}\NormalTok{speciesCode }\OtherTok{\textless{}{-}} \FunctionTok{as.character}\NormalTok{(}\FunctionTok{rep}\NormalTok{(}\ConstantTok{NA}\NormalTok{, }\FunctionTok{nrow}\NormalTok{(scenario)))}
\NormalTok{  scenario}\SpecialCharTok{$}\NormalTok{rootToShoot }\OtherTok{\textless{}{-}} \FunctionTok{as.numeric}\NormalTok{(}\FunctionTok{rep}\NormalTok{(}\ConstantTok{NA}\NormalTok{, }\FunctionTok{nrow}\NormalTok{(scenario)))}
\NormalTok{  scenario}\SpecialCharTok{$}\NormalTok{rootTurnover }\OtherTok{\textless{}{-}} \FunctionTok{as.numeric}\NormalTok{(}\FunctionTok{rep}\NormalTok{(}\ConstantTok{NA}\NormalTok{, }\FunctionTok{nrow}\NormalTok{(scenario)))}
\NormalTok{  scenario}\SpecialCharTok{$}\NormalTok{abovegroundTurnover }\OtherTok{\textless{}{-}} \FunctionTok{as.numeric}\NormalTok{(}\FunctionTok{rep}\NormalTok{(}\ConstantTok{NA}\NormalTok{, }\FunctionTok{nrow}\NormalTok{(scenario)))}
\NormalTok{  scenario}\SpecialCharTok{$}\NormalTok{rootDepthMax }\OtherTok{\textless{}{-}} \FunctionTok{as.numeric}\NormalTok{(}\FunctionTok{rep}\NormalTok{(}\ConstantTok{NA}\NormalTok{, }\FunctionTok{nrow}\NormalTok{(scenario)))}
\NormalTok{  scenario}\SpecialCharTok{$}\NormalTok{aboveground\_biomass }\OtherTok{\textless{}{-}} \FunctionTok{as.numeric}\NormalTok{(}\FunctionTok{rep}\NormalTok{(}\ConstantTok{NA}\NormalTok{, }\FunctionTok{nrow}\NormalTok{(scenario)))}
\NormalTok{  scenario}\SpecialCharTok{$}\NormalTok{belowground\_biomass }\OtherTok{\textless{}{-}} \FunctionTok{as.numeric}\NormalTok{(}\FunctionTok{rep}\NormalTok{(}\ConstantTok{NA}\NormalTok{, }\FunctionTok{nrow}\NormalTok{(scenario)))}
\NormalTok{  scenario}\SpecialCharTok{$}\NormalTok{mineral }\OtherTok{\textless{}{-}} \FunctionTok{as.numeric}\NormalTok{(}\FunctionTok{rep}\NormalTok{(}\ConstantTok{NA}\NormalTok{, }\FunctionTok{nrow}\NormalTok{(scenario)))}

  \CommentTok{\# Set initial conditions}
\NormalTok{  initialConditions }\OtherTok{\textless{}{-}} \FunctionTok{determineInitialCohorts}\NormalTok{(}\AttributeTok{initElv=}\NormalTok{initElv,}
                         \AttributeTok{meanSeaLevel=}\NormalTok{scenario}\SpecialCharTok{$}\NormalTok{meanSeaLevel[}\DecValTok{1}\NormalTok{], }
                         \AttributeTok{meanHighWater=}\NormalTok{scenario}\SpecialCharTok{$}\NormalTok{meanHighWater[}\DecValTok{1}\NormalTok{], }
                         \AttributeTok{meanHighHighWater=}\NormalTok{scenario}\SpecialCharTok{$}\NormalTok{meanHighHighWater[}\DecValTok{1}\NormalTok{], }
                         \AttributeTok{meanHighHighWaterSpring=}
\NormalTok{                           scenario}\SpecialCharTok{$}\NormalTok{meanHighHighWaterSpring[}\DecValTok{1}\NormalTok{],}
                         \AttributeTok{suspendedSediment=}\NormalTok{scenario}\SpecialCharTok{$}\NormalTok{suspendedSediment[}\DecValTok{1}\NormalTok{],}
                         \AttributeTok{nFloods =}\NormalTok{ nFloods, }\AttributeTok{floodTime.fn =}\NormalTok{ floodTime.fn,}
                         \AttributeTok{bMax=}\NormalTok{bMax, }\AttributeTok{zVegMin=}\NormalTok{zVegMin, }\AttributeTok{zVegMax=}\NormalTok{zVegMax,}
                         \AttributeTok{zVegPeak=}\NormalTok{zVegPeak,}
                         \AttributeTok{plantElevationType=}\NormalTok{plantElevationType,}
                         \AttributeTok{rootToShoot=}\NormalTok{rootToShoot, }\AttributeTok{rootTurnover=}\NormalTok{rootTurnover, }
                         \AttributeTok{rootDepthMax=}\NormalTok{rootDepthMax, }\AttributeTok{shape=}\NormalTok{shape,}
                         \AttributeTok{abovegroundTurnover=}\NormalTok{abovegroundTurnover,}
                         \AttributeTok{omDecayRate=}\NormalTok{omDecayRate, }
                         \AttributeTok{recalcitrantFrac=}\NormalTok{recalcitrantFrac, }
                         \AttributeTok{captureRate=}\NormalTok{captureRate,}
                         \AttributeTok{omPackingDensity=}\NormalTok{omPackingDensity, }
                         \AttributeTok{mineralPackingDensity=}\NormalTok{mineralPackingDensity,}
                         \AttributeTok{rootPackingDensity=}\NormalTok{omPackingDensity,}
                         \AttributeTok{speciesCode=}\NormalTok{speciesCode,}
                         \AttributeTok{initialCohorts=}\NormalTok{initialCohorts,}
                         \AttributeTok{uplandCohorts=}\NormalTok{uplandCohorts,}
                         \AttributeTok{supertidalCohorts=}\NormalTok{supertidalCohorts,}
                         \AttributeTok{supertidalSedimentInput=}\NormalTok{supertidalSedimentInput)}
\NormalTok{  cohorts }\OtherTok{\textless{}{-}}\NormalTok{ initialConditions[[}\DecValTok{1}\NormalTok{]]}

  \CommentTok{\# Add initial conditions to annual time step tracker}
\NormalTok{  scenario}\SpecialCharTok{$}\NormalTok{surfaceElevation[}\DecValTok{1}\NormalTok{] }\OtherTok{\textless{}{-}}\NormalTok{ initElv}
\NormalTok{  scenario[}\DecValTok{1}\NormalTok{,}\FunctionTok{names}\NormalTok{(scenario) }\SpecialCharTok{\%in\%} \FunctionTok{names}\NormalTok{(initialConditions[[}\DecValTok{2}\NormalTok{]])] }\OtherTok{\textless{}{-}}
\NormalTok{    initialConditions[[}\DecValTok{2}\NormalTok{]]}
\NormalTok{  scenario}\SpecialCharTok{$}\NormalTok{mineral[}\DecValTok{1}\NormalTok{] }\OtherTok{\textless{}{-}}\NormalTok{ initialConditions[[}\DecValTok{3}\NormalTok{]]}
  
\NormalTok{  cohorts}\SpecialCharTok{$}\NormalTok{year }\OtherTok{\textless{}{-}} \FunctionTok{rep}\NormalTok{(scenario}\SpecialCharTok{$}\NormalTok{year[}\DecValTok{1}\NormalTok{], }\FunctionTok{nrow}\NormalTok{(cohorts))}
  
  \CommentTok{\# Preallocate memory for cohort tracking}
\NormalTok{  nInitialCohorts }\OtherTok{\textless{}{-}} \FunctionTok{nrow}\NormalTok{(cohorts)}
\NormalTok{  nScenarioYears }\OtherTok{\textless{}{-}} \FunctionTok{nrow}\NormalTok{(scenario)}
\NormalTok{  initCohortRows }\OtherTok{\textless{}{-}}\NormalTok{ nInitialCohorts }\SpecialCharTok{*}\NormalTok{ nScenarioYears}
\NormalTok{  newCohortRows }\OtherTok{\textless{}{-}} \FunctionTok{sum}\NormalTok{(}\DecValTok{1}\SpecialCharTok{:}\NormalTok{nScenarioYears)}
\NormalTok{  totalRows }\OtherTok{\textless{}{-}}\NormalTok{ initCohortRows}\SpecialCharTok{+}\NormalTok{newCohortRows}
  
\NormalTok{  trackCohorts }\OtherTok{\textless{}{-}} \FunctionTok{data.frame}\NormalTok{(}\AttributeTok{age=}\FunctionTok{as.numeric}\NormalTok{(}\FunctionTok{rep}\NormalTok{(}\ConstantTok{NA}\NormalTok{, totalRows)),}
                             \AttributeTok{fast\_OM=}\FunctionTok{as.numeric}\NormalTok{(}\FunctionTok{rep}\NormalTok{(}\ConstantTok{NA}\NormalTok{, totalRows)),}
                             \AttributeTok{slow\_OM=}\FunctionTok{as.numeric}\NormalTok{(}\FunctionTok{rep}\NormalTok{(}\ConstantTok{NA}\NormalTok{, totalRows)),}
                             \AttributeTok{respired\_OM=}\FunctionTok{as.numeric}\NormalTok{(}\FunctionTok{rep}\NormalTok{(}\ConstantTok{NA}\NormalTok{, totalRows)),}
                             \AttributeTok{mineral=}\FunctionTok{as.numeric}\NormalTok{(}\FunctionTok{rep}\NormalTok{(}\ConstantTok{NA}\NormalTok{, totalRows)), }
                             \AttributeTok{root\_mass=}\FunctionTok{as.numeric}\NormalTok{(}\FunctionTok{rep}\NormalTok{(}\ConstantTok{NA}\NormalTok{, totalRows)),}
                             \AttributeTok{layer\_top=}\FunctionTok{as.numeric}\NormalTok{(}\FunctionTok{rep}\NormalTok{(}\ConstantTok{NA}\NormalTok{, totalRows)),}
                             \AttributeTok{layer\_bottom=}\FunctionTok{as.numeric}\NormalTok{(}\FunctionTok{rep}\NormalTok{(}\ConstantTok{NA}\NormalTok{, totalRows)),}
                             \AttributeTok{cumCohortVol=}\FunctionTok{as.numeric}\NormalTok{(}\FunctionTok{rep}\NormalTok{(}\ConstantTok{NA}\NormalTok{, totalRows)),}
                             \AttributeTok{year=}\FunctionTok{as.integer}\NormalTok{(}\FunctionTok{rep}\NormalTok{(}\ConstantTok{NA}\NormalTok{, totalRows)))}
  
  \CommentTok{\# add initial set of cohorts}
\NormalTok{  trackCohorts[}\DecValTok{1}\SpecialCharTok{:}\NormalTok{nInitialCohorts,] }\OtherTok{\textless{}{-}}\NormalTok{ cohorts}
  
  \CommentTok{\# create variables to keep track of cohorts added to the full cohort tracker}
\NormalTok{  cohortsNewRowMin }\OtherTok{\textless{}{-}}\NormalTok{ nInitialCohorts }\SpecialCharTok{+} \DecValTok{1}
  
  \CommentTok{\# Calculate the unmoving bottom of the profile as a consistent reference point}
\NormalTok{  profileBottomElv }\OtherTok{\textless{}{-}}\NormalTok{ initElv }\SpecialCharTok{{-}} \FunctionTok{max}\NormalTok{(cohorts}\SpecialCharTok{$}\NormalTok{layer\_bottom)}
  
  \CommentTok{\# Convert real growing elevations to dimensionless growing elevations}
  \ControlFlowTok{if}\NormalTok{ (}\SpecialCharTok{!}\NormalTok{ plantElevationType }\SpecialCharTok{\%in\%} \FunctionTok{c}\NormalTok{(}\StringTok{"dimensionless"}\NormalTok{, }\StringTok{"zStar"}\NormalTok{, }\StringTok{"Z*"}\NormalTok{, }\StringTok{"zstar"}\NormalTok{)) \{}
\NormalTok{    zStarVegMin }\OtherTok{\textless{}{-}} \FunctionTok{convertZToZstar}\NormalTok{(zVegMin, meanHighWaterDatum,}
\NormalTok{                                   meanSeaLevelDatum)}
\NormalTok{    zStarVegMax }\OtherTok{\textless{}{-}} \FunctionTok{convertZToZstar}\NormalTok{(zVegMax, meanHighWaterDatum,}
\NormalTok{                                   meanSeaLevelDatum)}
\NormalTok{    zStarVegPeak }\OtherTok{\textless{}{-}} \FunctionTok{convertZToZstar}\NormalTok{(zVegPeak, meanHighWaterDatum,}
\NormalTok{                                    meanSeaLevelDatum)}
\NormalTok{  \} }\ControlFlowTok{else}\NormalTok{ \{}
\NormalTok{    zStarVegMin }\OtherTok{\textless{}{-}}\NormalTok{ zVegMin}
\NormalTok{    zStarVegMax }\OtherTok{\textless{}{-}}\NormalTok{ zVegMax}
\NormalTok{    zStarVegPeak }\OtherTok{\textless{}{-}}\NormalTok{ zVegPeak}
\NormalTok{  \}}
  
  \CommentTok{\# Fourth, add one cohort for each year in the scenario}
  \CommentTok{\# Iterate through scenario table}
  \ControlFlowTok{for}\NormalTok{ (i }\ControlFlowTok{in} \DecValTok{2}\SpecialCharTok{:}\FunctionTok{nrow}\NormalTok{(scenario)) \{}
    
    \CommentTok{\# Calculate surface elevation relative to datum}
\NormalTok{    surfaceElvZStar }\OtherTok{\textless{}{-}} \FunctionTok{convertZToZstar}\NormalTok{(}\AttributeTok{z=}\NormalTok{scenario}\SpecialCharTok{$}\NormalTok{surfaceElevation[i}\DecValTok{{-}1}\NormalTok{], }
                                       \AttributeTok{meanHighWater=}\NormalTok{scenario}\SpecialCharTok{$}\NormalTok{meanHighWater[i], }
                                       \AttributeTok{meanSeaLevel=}\NormalTok{scenario}\SpecialCharTok{$}\NormalTok{meanSeaLevel[i])}
    
    \CommentTok{\# Calculate dynamic above ground biomass}
\NormalTok{    bio\_table }\OtherTok{\textless{}{-}} \FunctionTok{runMultiSpeciesBiomass}\NormalTok{(}\AttributeTok{z=}\NormalTok{surfaceElvZStar, }\AttributeTok{bMax=}\NormalTok{bMax,}
                                        \AttributeTok{zVegMax=}\NormalTok{zStarVegMax, }
                                        \AttributeTok{zVegMin=}\NormalTok{zStarVegMin,}
                                        \AttributeTok{zVegPeak=}\NormalTok{zStarVegPeak,}
                                        \AttributeTok{rootToShoot=}\NormalTok{rootToShoot,}
                                        \AttributeTok{rootTurnover=}\NormalTok{rootTurnover, }
                                        \AttributeTok{abovegroundTurnover=}\NormalTok{abovegroundTurnover, }
                                        \AttributeTok{rootDepthMax=}\NormalTok{rootDepthMax, }
                                        \AttributeTok{speciesCode=}\NormalTok{speciesCode)    }
    
    \CommentTok{\# Calculate Mineral pool}
\NormalTok{    dynamicMineralPool }\OtherTok{\textless{}{-}} \FunctionTok{deliverSediment}\NormalTok{(}\AttributeTok{z=}\NormalTok{scenario}\SpecialCharTok{$}\NormalTok{surfaceElevation[i}\DecValTok{{-}1}\NormalTok{], }
                            \AttributeTok{suspendedSediment=}\NormalTok{scenario}\SpecialCharTok{$}\NormalTok{suspendedSediment[i], }
                            \AttributeTok{meanSeaLevel=}\NormalTok{scenario}\SpecialCharTok{$}\NormalTok{meanSeaLevel[i], }
                            \AttributeTok{meanHighWater=}\NormalTok{scenario}\SpecialCharTok{$}\NormalTok{meanHighWater[i], }
                            \AttributeTok{meanHighHighWater =}\NormalTok{ scenario}\SpecialCharTok{$}\NormalTok{meanHighHighWater[i], }
                            \AttributeTok{meanHighHighWaterSpring =} 
\NormalTok{                              scenario}\SpecialCharTok{$}\NormalTok{meanHighHighWaterSpring[i], }
                            \AttributeTok{captureRate=}\NormalTok{captureRate,}
                            \AttributeTok{nFloods=}\NormalTok{nFloods,}
                            \AttributeTok{floodTime.fn=}\NormalTok{floodTime.fn)}
    
   
    \CommentTok{\# Add a the new inorganic sediment cohort,}
    \CommentTok{\# add live roots, and age the organic matter}
\NormalTok{    cohorts }\OtherTok{\textless{}{-}} \FunctionTok{addCohort}\NormalTok{(cohorts, }
                         \AttributeTok{totalRootMassPerArea=}\NormalTok{bio\_table}\SpecialCharTok{$}\NormalTok{belowground\_biomass[}\DecValTok{1}\NormalTok{],}
                         \AttributeTok{rootDepthMax=}\NormalTok{bio\_table}\SpecialCharTok{$}\NormalTok{rootDepthMax[}\DecValTok{1}\NormalTok{], }
                         \AttributeTok{rootTurnover =}\NormalTok{ bio\_table}\SpecialCharTok{$}\NormalTok{rootTurnover[}\DecValTok{1}\NormalTok{],}
                         \AttributeTok{omDecayRate =} \FunctionTok{list}\NormalTok{(}\AttributeTok{fast=}\NormalTok{omDecayRate, }\AttributeTok{slow=}\DecValTok{0}\NormalTok{),}
                         \AttributeTok{rootOmFrac=}\FunctionTok{list}\NormalTok{(}\AttributeTok{fast=}\DecValTok{1}\SpecialCharTok{{-}}\NormalTok{recalcitrantFrac, }\AttributeTok{slow=}\NormalTok{recalcitrantFrac),}
                         \AttributeTok{packing=}\FunctionTok{list}\NormalTok{(}\AttributeTok{organic=}\NormalTok{omPackingDensity,}
                                      \AttributeTok{mineral=}\NormalTok{mineralPackingDensity), }
                         \AttributeTok{rootDensity=}\NormalTok{rootPackingDensity, }\AttributeTok{shape=}\NormalTok{shape,}
                         \AttributeTok{mineralInput =}\NormalTok{ dynamicMineralPool)}
    
    \CommentTok{\# Add the calendar year to the cohort table so that we can track each cohort}
    \CommentTok{\# over each year}
\NormalTok{    cohorts}\SpecialCharTok{$}\NormalTok{year }\OtherTok{\textless{}{-}} \FunctionTok{rep}\NormalTok{(scenario}\SpecialCharTok{$}\NormalTok{year[i], }\FunctionTok{nrow}\NormalTok{(cohorts))}
    
    \CommentTok{\# add new set of cohorts, to the table of all the cohorts we are tracking}
    \CommentTok{\# over each year}
\NormalTok{    cohortsNewRowMax }\OtherTok{\textless{}{-}}\NormalTok{ cohortsNewRowMin }\SpecialCharTok{+} \FunctionTok{nrow}\NormalTok{(cohorts) }\SpecialCharTok{{-}} \DecValTok{1}
\NormalTok{    trackCohorts[cohortsNewRowMin}\SpecialCharTok{:}\NormalTok{cohortsNewRowMax,] }\OtherTok{\textless{}{-}}\NormalTok{ cohorts}
    
    \CommentTok{\# Keep track of the number of rows in the cohorts table}
\NormalTok{    cohortsNewRowMin }\OtherTok{\textless{}{-}}\NormalTok{ cohortsNewRowMin }\SpecialCharTok{+} \FunctionTok{nrow}\NormalTok{(cohorts) }\SpecialCharTok{+} \DecValTok{1}
    
    \CommentTok{\# Add annual variables to annual time step summary table}
\NormalTok{    scenario}\SpecialCharTok{$}\NormalTok{surfaceElevation[i] }\OtherTok{\textless{}{-}}\NormalTok{ profileBottomElv }\SpecialCharTok{+}
      \FunctionTok{max}\NormalTok{(cohorts}\SpecialCharTok{$}\NormalTok{layer\_bottom, }\AttributeTok{na.rm=}\NormalTok{T)}
\NormalTok{    scenario[i,}\FunctionTok{names}\NormalTok{(scenario) }\SpecialCharTok{\%in\%} \FunctionTok{names}\NormalTok{(bio\_table)] }\OtherTok{\textless{}{-}}\NormalTok{ bio\_table}
\NormalTok{    scenario}\SpecialCharTok{$}\NormalTok{mineral[i] }\OtherTok{\textless{}{-}}\NormalTok{ dynamicMineralPool}
    
\NormalTok{  \}}
  
  \CommentTok{\# Calculate C sequestration rate from cohorts table and add it to scenario}
  \CommentTok{\# table}
\NormalTok{  \{}
    \CommentTok{\# To convert OM to OC}
    \CommentTok{\# If parmeter list is 2 long then simple linear correlation}
    \ControlFlowTok{if}\NormalTok{ (}\FunctionTok{length}\NormalTok{(omToOcParams) }\SpecialCharTok{==} \DecValTok{2}\NormalTok{) \{}
\NormalTok{      omToOc }\OtherTok{\textless{}{-}} \ControlFlowTok{function}\NormalTok{(om, }\AttributeTok{B0=}\NormalTok{omToOcParams}\SpecialCharTok{$}\NormalTok{B0, }\AttributeTok{B1=}\NormalTok{omToOcParams}\SpecialCharTok{$}\NormalTok{B1) \{}
        \FunctionTok{return}\NormalTok{(B0 }\SpecialCharTok{+}\NormalTok{ om}\SpecialCharTok{*}\NormalTok{B1)\}}
\NormalTok{    \} }\ControlFlowTok{else} \ControlFlowTok{if}\NormalTok{ (}\FunctionTok{length}\NormalTok{(omToOcParams) }\SpecialCharTok{==} \DecValTok{3}\NormalTok{) \{}
      \CommentTok{\# If parameter list is 3 long, then it\textquotesingle{}s quadratic}
\NormalTok{      omToOc }\OtherTok{\textless{}{-}} \ControlFlowTok{function}\NormalTok{(om, }\AttributeTok{B0=}\NormalTok{omToOcParams}\SpecialCharTok{$}\NormalTok{B, }\AttributeTok{B1=}\NormalTok{omToOcParams}\SpecialCharTok{$}\NormalTok{B1,}
                         \AttributeTok{B2=}\NormalTok{omToOcParams}\SpecialCharTok{$}\NormalTok{B2) \{}\FunctionTok{return}\NormalTok{(B0 }\SpecialCharTok{+}\NormalTok{ om}\SpecialCharTok{*}\NormalTok{B1 }\SpecialCharTok{+}\NormalTok{ om}\SpecialCharTok{\^{}}\DecValTok{2}\SpecialCharTok{*}\NormalTok{B2)\}}
\NormalTok{    \} }\ControlFlowTok{else}\NormalTok{ \{}
      \CommentTok{\# If something else then trip an error message}
      \FunctionTok{stop}\NormalTok{(}\StringTok{"Invalid number of organic matter to}
\StringTok{           organic carbon conversion parameters,"}\NormalTok{)}
\NormalTok{    \}}
    
    \CommentTok{\# Remove NA values from cohorts}
\NormalTok{    trackCohorts }\OtherTok{\textless{}{-}} \FunctionTok{trimCohorts}\NormalTok{(trackCohorts)}
    
    \CommentTok{\# Apparent Carbon Burial Rate}
    \CommentTok{\# Carbon Sequestration Rate}
\NormalTok{    carbonFluxTab }\OtherTok{\textless{}{-}}\NormalTok{ trackCohorts }\SpecialCharTok{\%\textgreater{}\%}
      \CommentTok{\# Total organic matter per cohort}
\NormalTok{      dplyr}\SpecialCharTok{::}\FunctionTok{mutate}\NormalTok{(}\AttributeTok{total\_om\_perCoh =}\NormalTok{ fast\_OM }\SpecialCharTok{+}\NormalTok{ slow\_OM }\SpecialCharTok{+}\NormalTok{ root\_mass) }\SpecialCharTok{\%\textgreater{}\%}
\NormalTok{      dplyr}\SpecialCharTok{::}\FunctionTok{group\_by}\NormalTok{(year) }\SpecialCharTok{\%\textgreater{}\%}
      \CommentTok{\# Get the total cumulative observed and sequestered organic matter for the}
      \CommentTok{\# profile}
\NormalTok{      dplyr}\SpecialCharTok{::}\FunctionTok{summarise}\NormalTok{(}\AttributeTok{cumulativeTotalOm =} \FunctionTok{sum}\NormalTok{(total\_om\_perCoh),}
                       \AttributeTok{cumulativeSequesteredOm =} \FunctionTok{sum}\NormalTok{(slow\_OM)) }\SpecialCharTok{\%\textgreater{}\%} 
      \CommentTok{\# Caluclate fluxes by comparing total at time step i to time step i {-} 1}
\NormalTok{      dplyr}\SpecialCharTok{::}\FunctionTok{mutate}\NormalTok{(}\AttributeTok{omFlux =}\NormalTok{ cumulativeTotalOm }\SpecialCharTok{{-}} \FunctionTok{lag}\NormalTok{(cumulativeTotalOm),}
                    \AttributeTok{omSequestration =}\NormalTok{ cumulativeSequesteredOm }\SpecialCharTok{{-}}
                      \FunctionTok{lag}\NormalTok{(cumulativeSequesteredOm),}
                    \CommentTok{\# Convert organic matter to organic carbon using function}
                    \CommentTok{\# defined in previous step.}
                    \AttributeTok{cFlux =} \FunctionTok{omToOc}\NormalTok{(}\AttributeTok{om=}\NormalTok{omFlux),}
                    \AttributeTok{cSequestration =} \FunctionTok{omToOc}\NormalTok{(}\AttributeTok{om =}\NormalTok{ omSequestration))}
    
    \CommentTok{\# Join flux table to annual time step table}
\NormalTok{    scenario }\OtherTok{\textless{}{-}}\NormalTok{ scenario }\SpecialCharTok{\%\textgreater{}\%} 
\NormalTok{      dplyr}\SpecialCharTok{::}\FunctionTok{left\_join}\NormalTok{(carbonFluxTab)}
\NormalTok{  \}}
  
  \CommentTok{\# Return annual time steps and full set of cohorts}
\NormalTok{  outputsList }\OtherTok{\textless{}{-}} \FunctionTok{list}\NormalTok{(}\AttributeTok{annualTimeSteps =}\NormalTok{ scenario, }\AttributeTok{cohorts =}\NormalTok{ trackCohorts)}
  
  \FunctionTok{return}\NormalTok{(outputsList)}
\NormalTok{\}}
\end{Highlighting}
\end{Shaded}

\hypertarget{runmultispeciesbiomass}{%
\subsection{\texorpdfstring{\texttt{runMultiSpeciesBiomass()}}{runMultiSpeciesBiomass()}}\label{runmultispeciesbiomass}}

\begin{Shaded}
\begin{Highlighting}[]
\NormalTok{runMultiSpeciesBiomass }\OtherTok{\textless{}{-}} \ControlFlowTok{function}\NormalTok{(z, bMax, zVegMax, zVegMin, }\AttributeTok{zVegPeak=}\ConstantTok{NA}\NormalTok{,}
\NormalTok{                                   rootToShoot, rootTurnover,}
                                   \AttributeTok{abovegroundTurnover=}\ConstantTok{NA}\NormalTok{, rootDepthMax, }
                                   \AttributeTok{speciesCode=}\ConstantTok{NA}\NormalTok{, }\AttributeTok{competition.fn=}\ConstantTok{NA}\NormalTok{) \{}
   \CommentTok{\# If a custom competition function is not specified ...}
   \ControlFlowTok{if}\NormalTok{ (}\FunctionTok{is.na}\NormalTok{(competition.fn)) \{}
     \CommentTok{\# Generic competition function filters the inputed bio\_table maximum}
     \CommentTok{\# aboveground biomass}
\NormalTok{     competition.fn }\OtherTok{\textless{}{-}} \ControlFlowTok{function}\NormalTok{(bio\_table) \{}
       \FunctionTok{return}\NormalTok{(dplyr}\SpecialCharTok{::}\FunctionTok{filter}\NormalTok{(bio\_table,}
\NormalTok{               aboveground\_biomass }\SpecialCharTok{==} \FunctionTok{max}\NormalTok{(bio\_table}\SpecialCharTok{$}\NormalTok{aboveground\_biomass)))}
\NormalTok{     \}}
\NormalTok{   \}}
  
   \CommentTok{\# make a list of all the biological inputs}
\NormalTok{   bio\_inputs }\OtherTok{\textless{}{-}} \FunctionTok{list}\NormalTok{(bMax, zVegMax, zVegMin, zVegPeak, rootToShoot, }
\NormalTok{                     rootTurnover, abovegroundTurnover, rootDepthMax,}
\NormalTok{                     speciesCode)}
   
   \CommentTok{\# Get the length of each input}
\NormalTok{   input\_lengths }\OtherTok{\textless{}{-}} \FunctionTok{sapply}\NormalTok{(bio\_inputs, length)}
   
   \CommentTok{\# All the lengths either need to be 1 or equal to the number of species that}
   \CommentTok{\# are inputted}
   \ControlFlowTok{if}\NormalTok{ (}\FunctionTok{all}\NormalTok{(input\_lengths }\SpecialCharTok{==} \DecValTok{1} \SpecialCharTok{|}\NormalTok{ input\_lengths}\SpecialCharTok{==}\FunctionTok{max}\NormalTok{(input\_lengths))) \{}
     \CommentTok{\# If species codes are not specified ... }
     \ControlFlowTok{if}\NormalTok{ (}\FunctionTok{all}\NormalTok{(}\FunctionTok{is.na}\NormalTok{(speciesCode)) }\SpecialCharTok{|} \FunctionTok{length}\NormalTok{(speciesCode)}\SpecialCharTok{==}\FunctionTok{max}\NormalTok{(input\_lengths)) \{}
       \ControlFlowTok{if}\NormalTok{ (}\FunctionTok{any}\NormalTok{(}\FunctionTok{is.na}\NormalTok{(speciesCode))) \{}
         \CommentTok{\# ... then assign them 1,2,3,etc.}
\NormalTok{         speciesCode }\OtherTok{\textless{}{-}} \FunctionTok{as.character}\NormalTok{(}\DecValTok{1}\SpecialCharTok{:}\FunctionTok{max}\NormalTok{(input\_lengths))}
\NormalTok{       \}}
       \CommentTok{\# Create a table with all the biological inputs}
\NormalTok{       bio\_table }\OtherTok{\textless{}{-}} \FunctionTok{data.frame}\NormalTok{(}\AttributeTok{speciesCode=}\NormalTok{speciesCode, }
                               \AttributeTok{stringsAsFactors =}\NormalTok{ F) }\SpecialCharTok{\%\textgreater{}\%}
         \CommentTok{\# Using mutate will add vectors for all mutliple values and repeat}
         \CommentTok{\# single values so all the columns will be the same length.}
\NormalTok{         dplyr}\SpecialCharTok{::}\FunctionTok{mutate}\NormalTok{(}\AttributeTok{bMax =}\NormalTok{ bMax,}
                       \AttributeTok{zVegMax=}\NormalTok{zVegMax,}
                       \AttributeTok{zVegPeak=}\NormalTok{zVegPeak,}
                       \AttributeTok{zVegMin=}\NormalTok{zVegMin,}
                       \AttributeTok{rootToShoot=}\NormalTok{rootToShoot, }
                       \AttributeTok{rootTurnover=}\NormalTok{rootTurnover,}
                       \AttributeTok{abovegroundTurnover=}\NormalTok{abovegroundTurnover,}
                       \AttributeTok{rootDepthMax=}\NormalTok{rootDepthMax)}
\NormalTok{     \} }\ControlFlowTok{else}\NormalTok{ \{}
       \FunctionTok{stop}\NormalTok{(}\StringTok{"Species codes either need to be blank, }
\StringTok{            or have the same number as biomass inputs."}\NormalTok{)}
\NormalTok{     \}}
\NormalTok{   \} }\ControlFlowTok{else}\NormalTok{ \{}
     \FunctionTok{stop}\NormalTok{(}\StringTok{"The number of biomass inputs either need to be}
\StringTok{          the same length for multiple species or singular."}\NormalTok{)}
\NormalTok{   \}}
   
   \CommentTok{\# Run the parabolic biomass function on the table}
\NormalTok{   bio\_table}\SpecialCharTok{$}\NormalTok{aboveground\_biomass }\OtherTok{\textless{}{-}} \FunctionTok{mapply}\NormalTok{(predictBiomass, }\AttributeTok{z=}\NormalTok{z, }
                                          \AttributeTok{bMax=}\NormalTok{bio\_table}\SpecialCharTok{$}\NormalTok{bMax,}
                                          \AttributeTok{zVegMax =}\NormalTok{ bio\_table}\SpecialCharTok{$}\NormalTok{zVegMax, }
                                          \AttributeTok{zVegMin =}\NormalTok{ bio\_table}\SpecialCharTok{$}\NormalTok{zVegMin, }
                                          \AttributeTok{zVegPeak =}\NormalTok{ bio\_table}\SpecialCharTok{$}\NormalTok{zVegPeak)}
   
\NormalTok{   bio\_table}\SpecialCharTok{$}\NormalTok{belowground\_biomass }\OtherTok{\textless{}{-}}\NormalTok{ bio\_table}\SpecialCharTok{$}\NormalTok{aboveground\_biomass }\SpecialCharTok{*}
\NormalTok{     bio\_table}\SpecialCharTok{$}\NormalTok{rootToShoot}
   
   \CommentTok{\# If all aboveground biomass values are 0 ...}
   \ControlFlowTok{if}\NormalTok{ (}\FunctionTok{all}\NormalTok{(bio\_table}\SpecialCharTok{$}\NormalTok{aboveground\_biomass }\SpecialCharTok{==} \DecValTok{0}\NormalTok{)) \{}
     \CommentTok{\# ... then make all bio params 0 and changes species name to unvegetated.}
\NormalTok{     bio\_table[,}\DecValTok{6}\SpecialCharTok{:}\DecValTok{9}\NormalTok{] }\OtherTok{\textless{}{-}} \DecValTok{0}
\NormalTok{     bio\_table[,}\DecValTok{1}\NormalTok{] }\OtherTok{\textless{}{-}} \StringTok{"unvegetated"}
\NormalTok{     bio\_table }\OtherTok{\textless{}{-}}\NormalTok{ bio\_table[}\DecValTok{1}\NormalTok{,]}
\NormalTok{     bio\_table}
\NormalTok{   \}}
   
   \CommentTok{\# If the returned dataframe is more than one row long ...}
   \ControlFlowTok{if}\NormalTok{ (}\FunctionTok{nrow}\NormalTok{(bio\_table)}\SpecialCharTok{\textgreater{}}\DecValTok{1}\NormalTok{) \{}
     
     \CommentTok{\# ... then run the competition function.}
\NormalTok{     bio\_table }\OtherTok{\textless{}{-}} \FunctionTok{competition.fn}\NormalTok{(bio\_table)}
  
\NormalTok{     \}}
     
   \CommentTok{\# If more than one agb have exactly the same value ... }
   \ControlFlowTok{if}\NormalTok{ (}\FunctionTok{nrow}\NormalTok{(bio\_table)}\SpecialCharTok{\textgreater{}}\DecValTok{1}\NormalTok{) \{}
     \CommentTok{\# ... then simplify the table to being one row.}
\NormalTok{     bio\_table }\OtherTok{\textless{}{-}}\NormalTok{ bio\_table }\SpecialCharTok{\%\textgreater{}\%} 
\NormalTok{       dplyr}\SpecialCharTok{::}\FunctionTok{group\_by}\NormalTok{() }\SpecialCharTok{\%\textgreater{}\%} 
       \CommentTok{\# Group the species name into a single string ...}
\NormalTok{       dplyr}\SpecialCharTok{::}\FunctionTok{summarise}\NormalTok{(}\AttributeTok{speciesCode=}\FunctionTok{paste}\NormalTok{(bio\_table}\SpecialCharTok{$}\NormalTok{speciesCode,}\AttributeTok{collapse=}\StringTok{"; "}\NormalTok{), }
                        \CommentTok{\# ... and average all the parameters.}
                        \AttributeTok{rootToShoot=}\FunctionTok{mean}\NormalTok{(rootToShoot), }
                        \AttributeTok{rootTurnover=}\FunctionTok{mean}\NormalTok{(rootTurnover),}
                        \AttributeTok{rootDepthMax=}\FunctionTok{mean}\NormalTok{(rootDepthMax),}
                        \AttributeTok{aboveground\_biomass=}\FunctionTok{first}\NormalTok{(aboveground\_biomass)) }\SpecialCharTok{\%\textgreater{}\%} 
\NormalTok{       dplyr}\SpecialCharTok{::}\FunctionTok{ungroup}\NormalTok{()}
\NormalTok{   \}}
   
\NormalTok{   bio\_table }\OtherTok{\textless{}{-}}\NormalTok{ bio\_table[, }\SpecialCharTok{!} \FunctionTok{names}\NormalTok{(bio\_table) }\SpecialCharTok{\%in\%} \FunctionTok{c}\NormalTok{(}\StringTok{"bMax"}\NormalTok{,}\StringTok{"zVegMax"}\NormalTok{,}
                                                      \StringTok{"zVegPeak"}\NormalTok{,}\StringTok{"zVegMin"}\NormalTok{)]}

   \CommentTok{\# Return the data frame.}
   \FunctionTok{return}\NormalTok{(bio\_table)}
\NormalTok{\}}
\end{Highlighting}
\end{Shaded}

\hypertarget{runtoequilibrium}{%
\subsection{\texorpdfstring{\texttt{runToEquilibrium()}}{runToEquilibrium()}}\label{runtoequilibrium}}

\begin{Shaded}
\begin{Highlighting}[]
\NormalTok{runToEquilibrium }\OtherTok{\textless{}{-}} \ControlFlowTok{function}\NormalTok{(}\AttributeTok{minAge =} \DecValTok{50}\NormalTok{, }\AttributeTok{maxAge =} \DecValTok{12000}\NormalTok{,}
\NormalTok{                             minDepth,}
                             \AttributeTok{recordEvolution =} \ConstantTok{FALSE}\NormalTok{,}
                             \AttributeTok{relTol =} \FloatTok{1e{-}6}\NormalTok{, }\AttributeTok{absTol =} \FloatTok{1e{-}8}\NormalTok{, ...)\{}
  
  \CommentTok{\#initalize things to empty}

\NormalTok{  cohortProfile }\OtherTok{\textless{}{-}} \FunctionTok{data.frame}\NormalTok{(}\AttributeTok{age=}\ConstantTok{NA}\NormalTok{, }\AttributeTok{fast\_OM=}\ConstantTok{NA}\NormalTok{, }\AttributeTok{slow\_OM=}\ConstantTok{NA}\NormalTok{, }
                              \AttributeTok{respired\_OM=}\ConstantTok{NA}\NormalTok{,}
                              \AttributeTok{mineral=}\ConstantTok{NA}\NormalTok{, }\AttributeTok{root\_mass=}\ConstantTok{NA}\NormalTok{,}
                              \AttributeTok{layer\_top=}\ConstantTok{NA}\NormalTok{, }\AttributeTok{layer\_bottom=}\ConstantTok{NA}\NormalTok{)}
  
  \ControlFlowTok{if}\NormalTok{(recordEvolution)\{}
\NormalTok{    record.ls }\OtherTok{\textless{}{-}} \FunctionTok{list}\NormalTok{(cohortProfile)}
\NormalTok{  \}}
  \ControlFlowTok{for}\NormalTok{(ii }\ControlFlowTok{in} \DecValTok{2}\SpecialCharTok{:}\NormalTok{maxAge)\{}
    \ControlFlowTok{if}\NormalTok{(recordEvolution)\{}
\NormalTok{      record.ls[[}\FunctionTok{sprintf}\NormalTok{(}\StringTok{\textquotesingle{}Yr\%d\textquotesingle{}}\NormalTok{, ii)]] }\OtherTok{\textless{}{-}}\NormalTok{ cohortProfile}
\NormalTok{    \}}
    \CommentTok{\#oldCohort \textless{}{-} cohortProfile}
\NormalTok{    cohortProfile }\OtherTok{\textless{}{-}} \FunctionTok{addCohort}\NormalTok{(}\AttributeTok{massPools =}\NormalTok{ cohortProfile, ...)}
    
    \DocumentationTok{\#\#have the last layer OM pools stabilized?}
    \ControlFlowTok{if}\NormalTok{((ii }\SpecialCharTok{\textgreater{}}\NormalTok{ minAge) }\SpecialCharTok{\&}\NormalTok{ (}\FunctionTok{max}\NormalTok{(cohortProfile}\SpecialCharTok{$}\NormalTok{layer\_bottom)}\SpecialCharTok{\textgreater{}}\NormalTok{minDepth))\{}
      \ControlFlowTok{if}\NormalTok{((}\FunctionTok{abs}\NormalTok{(}\FunctionTok{diff}\NormalTok{(cohortProfile}\SpecialCharTok{$}\NormalTok{fast\_OM[ii}\SpecialCharTok{{-}}\FunctionTok{c}\NormalTok{(}\DecValTok{1}\SpecialCharTok{:}\DecValTok{2}\NormalTok{)] }\SpecialCharTok{+}
\NormalTok{                   cohortProfile}\SpecialCharTok{$}\NormalTok{slow\_OM[ii}\SpecialCharTok{{-}}\FunctionTok{c}\NormalTok{(}\DecValTok{1}\SpecialCharTok{:}\DecValTok{2}\NormalTok{)])) }\SpecialCharTok{\textless{}}\NormalTok{ absTol }\SpecialCharTok{|}
         \FunctionTok{abs}\NormalTok{(}\FunctionTok{diff}\NormalTok{(cohortProfile}\SpecialCharTok{$}\NormalTok{fast\_OM[ii}\SpecialCharTok{{-}}\FunctionTok{c}\NormalTok{(}\DecValTok{1}\SpecialCharTok{:}\DecValTok{2}\NormalTok{)] }\SpecialCharTok{+}
\NormalTok{                  cohortProfile}\SpecialCharTok{$}\NormalTok{slow\_OM[ii}\SpecialCharTok{{-}}\FunctionTok{c}\NormalTok{(}\DecValTok{1}\SpecialCharTok{:}\DecValTok{2}\NormalTok{)] ) }\SpecialCharTok{/}
\NormalTok{             (cohortProfile}\SpecialCharTok{$}\NormalTok{fast\_OM[ii}\DecValTok{{-}1}\NormalTok{] }\SpecialCharTok{+}
\NormalTok{              cohortProfile}\SpecialCharTok{$}\NormalTok{slow\_OM[ii}\DecValTok{{-}1}\NormalTok{])) }\SpecialCharTok{\textless{}}\NormalTok{ relTol))\{}
        \ControlFlowTok{break}
\NormalTok{      \}}
\NormalTok{    \}}
\NormalTok{  \}}
  
  \ControlFlowTok{if}\NormalTok{(recordEvolution)\{}
    \FunctionTok{return}\NormalTok{(record.ls)}
\NormalTok{  \}}\ControlFlowTok{else}\NormalTok{\{}
    \FunctionTok{return}\NormalTok{(cohortProfile)}
\NormalTok{  \}}
\NormalTok{\}}
\end{Highlighting}
\end{Shaded}

\hypertarget{sedimentinputs}{%
\subsection{\texorpdfstring{\texttt{sedimentInputs()}}{sedimentInputs()}}\label{sedimentinputs}}

\begin{Shaded}
\begin{Highlighting}[]
                           \CommentTok{\# Suspended Sediment Concentration, mg per liter}
\NormalTok{sedimentInputs }\OtherTok{\textless{}{-}} \ControlFlowTok{function}\NormalTok{(suspendedSediment, }
                           \CommentTok{\#mean tide height above marsh}
\NormalTok{                           meanTidalHeight, }
                           \AttributeTok{nTidesPerYear =} \DecValTok{704}\NormalTok{,}
                           \AttributeTok{soilLength=}\DecValTok{1}\NormalTok{, }\AttributeTok{soilWidth=}\DecValTok{1}\NormalTok{, ...)\{ }
  \CommentTok{\#only add sediment to the marsh if the mean high water is above the marsh}
  \CommentTok{\#elevation}
  \ControlFlowTok{if}\NormalTok{(meanTidalHeight }\SpecialCharTok{\textless{}} \DecValTok{0}\NormalTok{)\{}
    \FunctionTok{return}\NormalTok{(}\DecValTok{0}\NormalTok{) }
\NormalTok{  \}}
  \CommentTok{\# convert mg/l to grams/cm\^{}3}
\NormalTok{  annSediment }\OtherTok{\textless{}{-}}\NormalTok{ (suspendedSediment }\SpecialCharTok{*} \FloatTok{1e{-}6}\NormalTok{)  }\SpecialCharTok{*} 
    \CommentTok{\# Cumulative water volume}
\NormalTok{    nTidesPerYear }\SpecialCharTok{*}\NormalTok{ (meanTidalHeight }\SpecialCharTok{*} \FloatTok{0.5} \SpecialCharTok{*}\NormalTok{ soilLength }\SpecialCharTok{*}\NormalTok{ soilWidth) }
  
  \ControlFlowTok{if}\NormalTok{ (annSediment }\SpecialCharTok{\textless{}} \DecValTok{0}\NormalTok{) \{}
    \FunctionTok{return}\NormalTok{(}\DecValTok{0}\NormalTok{)}
\NormalTok{  \} }\ControlFlowTok{else}\NormalTok{ \{}
    \FunctionTok{return}\NormalTok{(annSediment)}
\NormalTok{  \}}
\NormalTok{\}}
\end{Highlighting}
\end{Shaded}

\hypertarget{simulatesetdata}{%
\subsection{\texorpdfstring{\texttt{simulateSetData()}}{simulateSetData()}}\label{simulatesetdata}}

\begin{Shaded}
\begin{Highlighting}[]
\NormalTok{simulateSetData }\OtherTok{\textless{}{-}} \ControlFlowTok{function}\NormalTok{(cohorts, scenario, }\AttributeTok{markerHorizonYear=}\ConstantTok{NA}\NormalTok{) \{}
  \CommentTok{\# For the scenario, For each set of cohorts measure the minimum depth of the}
  \CommentTok{\# cohort nearest to the horizon age without going older Accretion rate is}
  \CommentTok{\# mimium depth / (time t {-} marker horizon year)}
\NormalTok{  scenario }\OtherTok{\textless{}{-}}\NormalTok{ scenario }\SpecialCharTok{\%\textgreater{}\%} 
\NormalTok{    dplyr}\SpecialCharTok{::}\FunctionTok{mutate}\NormalTok{(}\AttributeTok{netElevationChange =}\NormalTok{ surfaceElevation }\SpecialCharTok{{-}} \FunctionTok{lag}\NormalTok{(surfaceElevation))}
  
  \ControlFlowTok{if}\NormalTok{ (}\SpecialCharTok{!} \FunctionTok{is.na}\NormalTok{(markerHorizonYear)) \{}
\NormalTok{    cohortsMh }\OtherTok{\textless{}{-}}\NormalTok{ cohorts }\SpecialCharTok{\%\textgreater{}\%} 
\NormalTok{      dplyr}\SpecialCharTok{::}\FunctionTok{filter}\NormalTok{(}\FunctionTok{complete.cases}\NormalTok{(.)) }\SpecialCharTok{\%\textgreater{}\%} 
\NormalTok{      dplyr}\SpecialCharTok{::}\FunctionTok{group\_by}\NormalTok{(year) }\SpecialCharTok{\%\textgreater{}\%} 
\NormalTok{      dplyr}\SpecialCharTok{::}\FunctionTok{filter}\NormalTok{(year }\SpecialCharTok{\textgreater{}=}\NormalTok{ markerHorizonYear) }\SpecialCharTok{\%\textgreater{}\%}
\NormalTok{      dplyr}\SpecialCharTok{::}\FunctionTok{mutate}\NormalTok{(}\AttributeTok{index =} \FunctionTok{length}\NormalTok{(age)}\SpecialCharTok{:}\DecValTok{1}\NormalTok{)}
    
\NormalTok{    mhYear }\OtherTok{\textless{}{-}}\NormalTok{ cohortsMh}\SpecialCharTok{$}\NormalTok{index[}\DecValTok{1}\NormalTok{]}
    
\NormalTok{    cohortsMh }\OtherTok{\textless{}{-}}\NormalTok{ cohortsMh }\SpecialCharTok{\%\textgreater{}\%} 
\NormalTok{      dplyr}\SpecialCharTok{::}\FunctionTok{filter}\NormalTok{(index }\SpecialCharTok{==}\NormalTok{ mhYear) }\SpecialCharTok{\%\textgreater{}\%}
\NormalTok{      dplyr}\SpecialCharTok{::}\FunctionTok{mutate}\NormalTok{(}\AttributeTok{accretionRate =}\NormalTok{ layer\_top }\SpecialCharTok{/}\NormalTok{ age) }\SpecialCharTok{\%\textgreater{}\%} 
\NormalTok{      dplyr}\SpecialCharTok{::}\FunctionTok{select}\NormalTok{(year, accretionRate) }\CommentTok{\# \%\textgreater{}\% }
      \CommentTok{\# dplyr::rename(years = year)}
    
\NormalTok{    cohortsMh}\SpecialCharTok{$}\NormalTok{accretionRate[}\DecValTok{1}\NormalTok{] }\OtherTok{\textless{}{-}} \DecValTok{0} 
    
\NormalTok{    scenario }\OtherTok{\textless{}{-}}\NormalTok{ scenario }\SpecialCharTok{\%\textgreater{}\%} 
\NormalTok{      dplyr}\SpecialCharTok{::}\FunctionTok{left\_join}\NormalTok{(cohortsMh, }\AttributeTok{by=}\StringTok{"year"}\NormalTok{)}
\NormalTok{  \}}
  \FunctionTok{return}\NormalTok{(scenario)}
\NormalTok{\}}
\end{Highlighting}
\end{Shaded}

\hypertarget{simulatesoilcore}{%
\subsection{\texorpdfstring{\texttt{simulateSoilCore()}}{simulateSoilCore()}}\label{simulatesoilcore}}

\begin{Shaded}
\begin{Highlighting}[]
\NormalTok{simulateSoilCore }\OtherTok{\textless{}{-}} \ControlFlowTok{function}\NormalTok{(cohorts, coreYear, }\AttributeTok{coreDepth=}\DecValTok{100}\NormalTok{, }
                             \AttributeTok{coreMaxs=}\DecValTok{1}\SpecialCharTok{:}\NormalTok{coreDepth, }\AttributeTok{coreMins=}\NormalTok{coreMaxs}\DecValTok{{-}1}\NormalTok{,}
                             \AttributeTok{omToOcParams =} \FunctionTok{list}\NormalTok{(}\AttributeTok{B0=}\DecValTok{0}\NormalTok{, }\AttributeTok{B1=}\FloatTok{0.48}\NormalTok{),}
                             \AttributeTok{omPackingDensity=}\FloatTok{0.085}\NormalTok{, }\AttributeTok{mineralPackingDensity=}\FloatTok{1.99}\NormalTok{,}
                             \AttributeTok{rootPackingDensity=}\NormalTok{omPackingDensity) \{}
  
  \CommentTok{\# Filter only to cohorts in the core year}
\NormalTok{  cohortsInCoreYear }\OtherTok{\textless{}{-}}\NormalTok{ cohorts }\SpecialCharTok{\%\textgreater{}\%} 
\NormalTok{    dplyr}\SpecialCharTok{::}\FunctionTok{filter}\NormalTok{(year}\SpecialCharTok{==}\NormalTok{coreYear) }\SpecialCharTok{\%\textgreater{}\%}
\NormalTok{    dplyr}\SpecialCharTok{::}\FunctionTok{select}\NormalTok{(}\SpecialCharTok{{-}}\NormalTok{year)}
  
  \CommentTok{\# If the profile is shorter than the intended core ...}
  \ControlFlowTok{if}\NormalTok{ (}\FunctionTok{max}\NormalTok{(coreMaxs) }\SpecialCharTok{\textgreater{}} \FunctionTok{max}\NormalTok{(cohortsInCoreYear}\SpecialCharTok{$}\NormalTok{layer\_bottom, }\AttributeTok{na.rm=}\NormalTok{T)) \{}
\NormalTok{    coreMins }\OtherTok{\textless{}{-}}\NormalTok{ coreMins[coreMins}\SpecialCharTok{\textless{}} \FunctionTok{max}\NormalTok{(cohortsInCoreYear}\SpecialCharTok{$}\NormalTok{layer\_bottom, }\AttributeTok{na.rm=}\NormalTok{T)]}
\NormalTok{    coreMaxs }\OtherTok{\textless{}{-}}\NormalTok{ coreMaxs[}\DecValTok{1}\SpecialCharTok{:}\FunctionTok{length}\NormalTok{(coreMins)]}
\NormalTok{    coreMaxs[}\FunctionTok{length}\NormalTok{(coreMaxs)] }\OtherTok{\textless{}{-}} \FunctionTok{max}\NormalTok{(cohortsInCoreYear}\SpecialCharTok{$}\NormalTok{layer\_bottom, }\AttributeTok{na.rm=}\NormalTok{T)}
\NormalTok{  \}}
  \CommentTok{\# }
  
  \CommentTok{\# To convert OM to OC}
  \CommentTok{\# If parmeter list is 2 long then simple linear correlation}
  \ControlFlowTok{if}\NormalTok{ (}\FunctionTok{length}\NormalTok{(omToOcParams) }\SpecialCharTok{==} \DecValTok{2}\NormalTok{) \{}
\NormalTok{    omToOc }\OtherTok{\textless{}{-}} \ControlFlowTok{function}\NormalTok{(om, }\AttributeTok{B0=}\NormalTok{omToOcParams}\SpecialCharTok{$}\NormalTok{B0, }\AttributeTok{B1=}\NormalTok{omToOcParams}\SpecialCharTok{$}\NormalTok{B1) \{}
      \FunctionTok{return}\NormalTok{(B0 }\SpecialCharTok{+}\NormalTok{ om}\SpecialCharTok{*}\NormalTok{B1)\}}
\NormalTok{  \} }\ControlFlowTok{else} \ControlFlowTok{if}\NormalTok{ (}\FunctionTok{length}\NormalTok{(omToOcParams) }\SpecialCharTok{==} \DecValTok{3}\NormalTok{) \{}
    \CommentTok{\# If parameter list is 3 long, then it\textquotesingle{}s quadratic}
\NormalTok{    omToOc }\OtherTok{\textless{}{-}} \ControlFlowTok{function}\NormalTok{(om, }\AttributeTok{B0=}\NormalTok{omToOcParams}\SpecialCharTok{$}\NormalTok{B, }\AttributeTok{B1=}\NormalTok{omToOcParams}\SpecialCharTok{$}\NormalTok{B1,}
                       \AttributeTok{B2=}\NormalTok{omToOcParams}\SpecialCharTok{$}\NormalTok{B2) \{}\FunctionTok{return}\NormalTok{(B0 }\SpecialCharTok{+}\NormalTok{ om}\SpecialCharTok{*}\NormalTok{B1 }\SpecialCharTok{+}\NormalTok{ om}\SpecialCharTok{\^{}}\DecValTok{2}\SpecialCharTok{*}\NormalTok{B2)\}}
\NormalTok{  \} }\ControlFlowTok{else}\NormalTok{ \{}
    \CommentTok{\# If something else then trip an error message}
\FunctionTok{stop}\NormalTok{(}\StringTok{"Invalid number of organic matter to organic carbon conversion parameters,"}\NormalTok{)}
\NormalTok{  \}}
  
  \CommentTok{\# Convert cohorts to age{-}depth profile}
\NormalTok{  coreYearAgeDepth }\OtherTok{\textless{}{-}} \FunctionTok{convertProfileAgeToDepth}\NormalTok{(}\AttributeTok{ageCohort=}\NormalTok{cohortsInCoreYear,}
                                                \AttributeTok{layerTop=}\NormalTok{coreMins,}
                                                \AttributeTok{layerBottom=}\NormalTok{coreMaxs)}
  
\NormalTok{  coreYearAgeDepth }\OtherTok{\textless{}{-}}\NormalTok{ coreYearAgeDepth }\SpecialCharTok{\%\textgreater{}\%}  
\NormalTok{    dplyr}\SpecialCharTok{::}\FunctionTok{filter}\NormalTok{(}\FunctionTok{complete.cases}\NormalTok{(.)) }\SpecialCharTok{\%\textgreater{}\%}
\NormalTok{    dplyr}\SpecialCharTok{::}\FunctionTok{mutate}\NormalTok{(}\AttributeTok{om\_fraction =}\NormalTok{ (fast\_OM}\SpecialCharTok{+}\NormalTok{slow\_OM}\SpecialCharTok{+}\NormalTok{root\_mass)}\SpecialCharTok{/}
\NormalTok{                    (fast\_OM}\SpecialCharTok{+}\NormalTok{slow\_OM}\SpecialCharTok{+}\NormalTok{root\_mass}\SpecialCharTok{+}\NormalTok{mineral),}
                  \AttributeTok{dry\_bulk\_density =}\NormalTok{ (}\DecValTok{1}\SpecialCharTok{/}\NormalTok{(}
\NormalTok{                    (om\_fraction}\SpecialCharTok{/}\NormalTok{omPackingDensity) }\SpecialCharTok{+}
\NormalTok{                      ((}\DecValTok{1}\SpecialCharTok{{-}}\NormalTok{om\_fraction)}\SpecialCharTok{/}\NormalTok{mineralPackingDensity))}
\NormalTok{                    ),}
                  \AttributeTok{oc\_fraction =} \FunctionTok{omToOc}\NormalTok{(om\_fraction)}
\NormalTok{                  )}
  
  \CommentTok{\# Add to the list of outputs}
  \FunctionTok{return}\NormalTok{(coreYearAgeDepth) }
\NormalTok{\}}
\end{Highlighting}
\end{Shaded}

\hypertarget{trimcohorts}{%
\subsection{\texorpdfstring{\texttt{trimCohorts}()`}{trimCohorts()`}}\label{trimcohorts}}

\begin{Shaded}
\begin{Highlighting}[]
\NormalTok{trimCohorts }\OtherTok{\textless{}{-}} \ControlFlowTok{function}\NormalTok{(cohorts) \{}
\NormalTok{  cohorts }\OtherTok{\textless{}{-}}\NormalTok{ cohorts }\SpecialCharTok{\%\textgreater{}\%}
\NormalTok{    dplyr}\SpecialCharTok{::}\FunctionTok{filter}\NormalTok{(cumCohortVol}\SpecialCharTok{!=}\DecValTok{0}\NormalTok{)}
  \FunctionTok{return}\NormalTok{(cohorts)}
\NormalTok{\}}
\end{Highlighting}
\end{Shaded}
